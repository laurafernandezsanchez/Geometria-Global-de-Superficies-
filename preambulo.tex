% --- PAQUETES DE IDIOMA Y CODIFICACIÓN ---
\usepackage[utf8]{inputenc}
\usepackage[T1]{fontenc}
\usepackage[spanish,es-tabla]{babel}

% --- PAQUETES MATEMÁTICOS ---
\usepackage{amsmath, amsthm, amssymb, amsfonts}
\usepackage{mathtools}
\usepackage{physics} 
% ...existing code...
\usepackage{wasysym} 


% --- PAQUETES DE DISEÑO Y GRÁFICOS ---
\usepackage{geometry}
\geometry{left=2.5cm, right=2.5cm, top=3cm, bottom=3cm}
\usepackage{xcolor}
\usepackage{enumitem}
% Configuración de enlaces bonitos
\usepackage[colorlinks=true, linkcolor=mainblue, urlcolor=mainblue, citecolor=mainred]{hyperref}
\usepackage{graphicx} % Para imágenes

% --- CONFIGURACIÓN DE TCOLORBOX (Tus cajas bonitas) ---
\usepackage[most]{tcolorbox}

% Definición de colores
\definecolor{mainblue}{RGB}{0, 102, 204}
\definecolor{mainred}{RGB}{204, 0, 0}
\definecolor{maingreen}{RGB}{0, 153, 76}

% Cajas personalizadas
\newtcolorbox[auto counter, number within=section]{definicion}[2][]{
    colback=mainblue!5!white, colframe=mainblue!75!black, fonttitle=\bfseries,
    title=Definición~\thetcbcounter: #2, sharp corners=downhill, enhanced,
    attach boxed title to top left={yshift=-2mm, xshift=2mm},
    boxed title style={colback=mainblue!75!black}, #1
}

\newtcolorbox[auto counter, number within=section]{teorema}[2][]{
    colback=red!5!white, colframe=mainred!75!black, fonttitle=\bfseries,
    title=Teorema~\thetcbcounter: #2, sharp corners=downhill, enhanced,
    attach boxed title to top left={yshift=-2mm, xshift=2mm},
    boxed title style={colback=mainred!75!black}, #1
}
% --- CAJA PARA LEMAS (Lila claro) ---
\newtcolorbox[auto counter, number within=section]{lema}[2][]{
    colback=violet!5!white,          % Fondo lila muy suave
    colframe=violet!60!black,        % Borde violeta oscuro para contraste
    fonttitle=\bfseries,
    title=Lema~\thetcbcounter: #2, 
    sharp corners=downhill, 
    enhanced,
    attach boxed title to top left={yshift=-2mm, xshift=2mm},
    boxed title style={colback=violet!60!black},
    #1
}
% --- CAJA PARA OBSERVACIONES (Verde) ---
\newtcolorbox[auto counter, number within=section]{observacion}[2][]{
    colback=pink!5!white,           % Fondo verde muy suave
    colframe=pink!60!black,         % Borde verde oscuro
    fonttitle=\bfseries,
    title=Observación~\thetcbcounter: #2, 
    sharp corners=downhill, 
    enhanced,
    attach boxed title to top left={yshift=-2mm, xshift=2mm},
    boxed title style={colback=pink!60!black},
    breakable,
    #1
}

% --- CAJA PARA COROLARIOS (Azul Cerceta / Teal) ---
\newtcolorbox[auto counter, number within=section]{corolario}[2][]{
    colback=teal!5!white,            % Fondo azulado muy suave
    colframe=teal!60!black,          % Borde azul oscuro/pavesa
    fonttitle=\bfseries,
    title=Corolario~\thetcbcounter: #2, 
    sharp corners=downhill, 
    enhanced,
    attach boxed title to top left={yshift=-2mm, xshift=2mm},
    boxed title style={colback=teal!60!black},
    #1
}

\newtcolorbox[auto counter, number within=section]{proposicion}[2][]{
    colback=yellow!10!white,         % Amarillo muy claro para el fondo
    colframe=yellow!60!black,        % Amarillo oscuro/dorado para el borde
    fonttitle=\bfseries,
    title=Proposición~\thetcbcounter: #2, 
    sharp corners=downhill, 
    enhanced,
    attach boxed title to top left={yshift=-2mm, xshift=2mm},
    boxed title style={colback=yellow!60!black}, % Color del fondo del título
    #1
}

\newtcolorbox[auto counter, number within=section]{ejemplo}[2][]{
    colback=maingreen!5!white, colframe=maingreen!75!black, fonttitle=\bfseries,
    title=Ejemplo~\thetcbcounter: #2, breakable, enhanced, #1
}

% --- COMANDOS MATEMÁTICOS PERSONALIZADOS ---
\newcommand{\R}{\mathbb{R}}
\newcommand{\Ssf}{\mathbb{S}}
\newcommand{\Xfrak}{\mathfrak{X}}
\newcommand{\inner}[2]{\langle #1, #2 \rangle}

% --- CONFIGURACIÓN DE ENCABEZADO Y PIE DE PÁGINA ---
\usepackage{fancyhdr}
\setlength{\headheight}{15pt} % Evita advertencias de altura

% Activamos el estilo 'fancy'
\pagestyle{fancy}
\fancyhf{} % Borra los ajustes por defecto

% --- CONFIGURACIÓN DEL ENCABEZADO ---
\fancyhead[L]{\small \textbf{Geometría Global de Superficies}} 
\fancyhead[R]{\small \nouppercase{\leftmark}} % Derecha: Nombre del Capítulo actual

% --- CONFIGURACIÓN DEL PIE DE PÁGINA ---
\fancyfoot[L]{\hyperlink{indice}{\textbf{Ir al Índice}}} 
% Modificamos el centro para incluir el enlace:
\fancyfoot[C]{\thepage} 
\fancyfoot[R]{\small UMU -- 2025}

% --- LÍNEAS SEPARADORAS ---
\renewcommand{\headrulewidth}{0.4pt} % Grosor línea superior
\renewcommand{\footrulewidth}{0.4pt} % Grosor línea inferior
% --- CONFIGURACIÓN DE PÁRRAFOS ---
\setlength{\parindent}{0pt}  % Quita la sangría (el hueco al principio)
\setlength{\parskip}{0.8em}  % Añade espacio entre párrafos (opcional pero recomendado)

% Evita que LaTeX estire el espacio verticalmente para llenar la página
\raggedbottom