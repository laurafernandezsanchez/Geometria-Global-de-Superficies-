% Archivo: capitulos/tema2.tex

% -------------------------------------------------------------------------
\section{Curvas regulares a trozos y Variaciones}

Para estudiar las propiedades minimizantes de las geodésicas (que son las curvas más cortas localmente), necesitamos 
ampliar el espacio de curvas admisibles para permitir esquinas.

\begin{definicion}{Curva regular a trozos}
Una curva regular a trozos es una aplicación continua 
$\alpha: [a,b] \longrightarrow \mathbb{R}^n$ para la cual existe una 
partición $a=t_0 < t_1 < \dots < t_k = b$ tal que, para cada $i=1, \dots, k$, 
la restricción $\alpha_i := \alpha|_{[t_{i-1}, t_i]}$ es una curva regular diferenciable.
\end{definicion}

\textbf{Notación y Vértices:}
Denotaremos los límites laterales de la derivada en los puntos de la partición como:
\[
\alpha'_{-}(t_i) = \lim_{t \to t_i^{-}} \alpha'(t)=\alpha'_i(t_i), \quad \alpha'_{+}(t_i) = \lim_{t \to t_i^{+}} \alpha'(t)=\alpha'_{i+1}(t_i).
\]
Diremos que $\alpha(t_i)$ es un \textbf{vértice} de $\alpha$ si $\alpha'_{-}(t_i) \neq \alpha'_{+}(t_i)$.
\\
\textit{Observación:} Toda curva regular a trozos puede reparametrizarse por el arco (p.p.a.). Y en $t=a$ y $t=b$ solo existe una derivada, excepto si la curva es cerrada (y podría haber un vértice o cerrarse regularmente).
\begin{figure}
    \centering
    \includegraphics[scale=0.65]{imagenes/CurvaRegularTrozos.png}    
\end{figure}


\subsection{El concepto de Variación}
\begin{definicion}{Variación de una curva}
Sean $\alpha : [a, b] \longrightarrow S$ una curva regular a trozos en una superficie regular $S$ y $a = s_0 < s_1 < \dots < s_k = b$ una partición de $[a, b]$ tal que, para cada $i = 1, \dots, k$, $\alpha_i := \alpha|_{[s_{i-1}, s_i]}$ es una curva regular.

Una \textbf{variación} de $\alpha$ es una aplicación continua $\phi : [a, b] \times (-\varepsilon, \varepsilon) \longrightarrow S$, que es diferenciable en cada uno de los rectángulos $[s_{i-1}, s_i] \times (-\varepsilon, \varepsilon)$, y tal que:
\begin{enumerate}[label=\roman*)]
    \item $\phi(s, 0) = \phi_0(s) = \alpha(s)$ para todo $s \in [a, b]$;
    \item para todo $t \in (-\varepsilon, \varepsilon)$, $\alpha_t = \phi(\cdot, t) : [a, b] \longrightarrow S$ es una curva parametrizada regular a trozos, llamada \textbf{curva de la variación}.
\end{enumerate}

Además, se dice que $\phi$ es una \textbf{variación propia} (o con \textbf{extremos fijos}) cuando $\alpha_t(a) = \phi(a, t) = \alpha(a)$ y $\alpha_t(b) = \phi(b, t) = \alpha(b)$ para todo $t \in (-\varepsilon, \varepsilon)$.
\end{definicion}



\begin{figure}
    \centering
    \includegraphics[scale=0.4]{imagenes/Variación de Una Cruva.png}
\end{figure}

\begin{definicion}{Curvas Transversales y de Variación}
Dada una variación $\phi$ de $\alpha$, se llama \textbf{curvas transversales} a las curvas $\beta_s : (-\varepsilon, \varepsilon) \longrightarrow S$ definidas por $\beta_s(t) = \phi(s, t)$, para cada $s \in [a, b]$.
Se llama \textbf{curvas de la variación} a las curvas $\alpha_t : [a, b] \longrightarrow S$ definidas por $\alpha_t(s) = \phi(s, t)$, para cada $t \in (-\varepsilon, \varepsilon)$.    
\end{definicion}



\begin{definicion}{Campo Variacional}
Sea $\phi$ una variación de una curva regular a trozos $\alpha$. Se define el \textbf{campo variacional} de $\phi$ como:
\[
\beta'(s)=Z(s) = \frac{\partial \phi}{\partial t}(s, 0) \in \mathfrak{X}(\alpha).
\]
\end{definicion}    


\textbf{Tipos de variaciones:}
\begin{itemize}
    \item \textbf{Variación propia (extremos fijos):} Si $\alpha_t(a) = \alpha(a)$ y $\alpha_t(b) = \alpha(b)$ para todo $t$. Esto implica que $Z(a)=0$ y $Z(b)=0$.
    \item \textbf{Variación normal:} Si el campo variacional es ortogonal a la curva, $\inner{Z(s)}{\alpha'(s)} = 0$.
\end{itemize}


% -------------------------------------------------------------------------
\section{Fórmulas de Variación de la Longitud}
Una variación $\phi$ de $\alpha$ genera una familia uniparamétrica $\{\alpha_t\}_{t\in (-\epsilon,\epsilon)}$ de curvas en la superficie, cuya longitud es el llamado \textbf{funcional longitud:}
\[
L(t) := L_{a}^{b}(\alpha_{t}) = \sum_{i=1}^{k} L_{s_{i-1}}^{s_{i}}(\alpha_{t}) = \sum_{i=1}^{k} \int_{s_{i-1}}^{s_{i}} |\alpha_{t}^{\prime}(s)| \, ds.
\]
Ponemos los sumatorios porque tenemos una curva regular a trozos, y cada tramo es regular, así que podemos calcular la longitud de cada tramo y sumarlas.



\begin{teorema}{Primera fórmula de variación}
Sea $\alpha: [0,l] \longrightarrow S$ una curva regular a trozos p.p.a. y sea $Z$ el campo variacional de una variación $\phi$. Entonces:
\[
L'(0) = \left[ \inner{Z(s)}{\alpha'(s)} \right]_0^l - \sum_{i=1}^{k-1} \inner{Z(s_i)}{\Delta_i \alpha'} - \int_0^l \inner{Z(s)}{\frac{D\alpha'}{ds}(s)} ds,
\]
donde $\Delta_i \alpha' = \alpha'_{+}(s_i) - \alpha'_{-}(s_i)$ es el "salto" de velocidad en los vértices.
\end{teorema}

Si la variación es \textbf{propia} (extremos fijos), los términos de frontera desaparecen:
\[
L'(0) = - \sum_{i=1}^{k-1} \inner{Z(s_i)}{\Delta_i \alpha'} - \int_0^l \inner{Z(s)}{\frac{D\alpha'}{ds}(s)} ds.
\]
\begin{observacion}{Fórmula de variación para las geodésicas}
    Observemos que si $\alpha$ es una geodésica, entonces $\frac{D\alpha'}{ds} \equiv 0$. Por tanto, para cualquier 
    variación propia de $\alpha$, se cumple que que $L'(0) = - \sum_{i=1}^{k-1} \inner{Z(s_i)}{\Delta_i \alpha'}$. En particular, si $\alpha$ no tiene vértices, 
    entonces $L'(0) = 0$ para toda variación propia de $\alpha$. (Por ser geodésica, no tendrá vértices porque es regular)
    Será cierto el recíproco? Es decir, si $L'(0) = 0$ para toda variación propia de $\alpha$, entonces $\alpha$ es una geodésica?
\end{observacion}

\begin{proof}[Demostración de la primera variación de la longitud de arco]
Sea $\alpha: [0, l] \to S$ una curva p.p.a. y regular a trozos, con vértices en $0 = s_0 < s_1 < \dots < s_k = l$. Sea $\phi(s,t)$ una variación de $\alpha$, tal que $\alpha_t(s) = \phi(s,t)$ y el campo variacional es $Z(s) = \frac{\partial \phi}{\partial t}(s,0)$.

Tomamos un intervalo donde la curva es regular, $[s_{i-1}, s_i]$. Calculamos la longitud de arco de la curva variada en ese tramo, que denotaremos como $L_i(t)$:
\[
L_i(t) = \int_{s_{i-1}}^{s_i} |\alpha'_t(s)| \, ds = \int_{s_{i-1}}^{s_i} \left| \frac{\partial \phi}{\partial s}(s,t) \right| ds = \int_{s_{i-1}}^{s_i} \left\langle \frac{\partial \phi}{\partial s}, \frac{\partial \phi}{\partial s} \right\rangle^{\frac{1}{2}} ds
\]

Derivemos $L_i(t)$ respecto del parámetro de variación $t$:
\[
L'_i(t) = \int_{s_{i-1}}^{s_i} \frac{2 \left\langle \frac{\partial^2 \phi}{\partial s \partial t}, \frac{\partial \phi}{\partial s} \right\rangle}{2 \underbrace{\left\langle \frac{\partial \phi}{\partial s}, \frac{\partial \phi}{\partial s} \right\rangle^{\frac{1}{2}}}_{|\alpha'_t(s)|}} \, ds
\]

Sustituyendo en $t = 0$, y recordando que $\alpha_0(s) = \alpha(s)$ está p.p.a. (por lo que $|\alpha'_0(s)| = 1$):
\[
L'_i(0) = \int_{s_{i-1}}^{s_i} \frac{\left\langle \frac{\partial^2 \phi}{\partial s \partial t}(s,0), \frac{\partial \phi}{\partial s} \right\rangle}{\underbrace{|\alpha'_0(s)|}_{= 1}} \, ds = \int_{s_{i-1}}^{s_i} \left\langle \frac{\partial^2 \phi}{\partial s \partial t}(s,0), \frac{\partial \phi}{\partial s}(s,0) \right\rangle ds
\]

Por la regla del producto para la derivada de un producto escalar, se tiene:
\[
\frac{d}{ds} \left\langle \frac{\partial \phi}{\partial t}, \frac{\partial \phi}{\partial s} \right\rangle = \left\langle \frac{\partial^2 \phi}{\partial t \partial s}, \frac{\partial \phi}{\partial s} \right\rangle + \left\langle \frac{\partial \phi}{\partial t}, \frac{\partial^2 \phi}{\partial s^2} \right\rangle
\]
Despejando el primer sumando del lado derecho (y usando el Teorema de Schwarz $\frac{\partial^2 \phi}{\partial t \partial s} = \frac{\partial^2 \phi}{\partial s \partial t}$) e integrando:

\begin{align*}
L'_i(0) &= \int_{s_{i-1}}^{s_i} \left[ \frac{d}{ds} \left\langle \underbrace{\frac{\partial \phi}{\partial t}(s,0)}_{Z(s)}, \underbrace{\frac{\partial \phi}{\partial s}(s,0)}_{\alpha'(s)} \right\rangle - \left\langle \underbrace{\frac{\partial \phi}{\partial t}(s,0)}_{Z(s)}, \underbrace{\frac{\partial^2 \phi}{\partial s^2}(s,0)}_{\alpha''(s)} \right\rangle \right] ds \\
&= \left[ \langle Z(s), \alpha'(s) \rangle \right]_{s_{i-1}}^{s_i} - \int_{s_{i-1}}^{s_i} \langle Z(s), \alpha''(s) \rangle \, ds
\end{align*}

Sabemos que $\alpha''(s) = \frac{D\alpha'}{ds} + (\alpha'')^\perp$. Al hacer el producto escalar $\langle Z(s), \alpha''(s) \rangle$, como el campo variacional $Z(s)$ es un campo tangente a la superficie, su producto con la componente normal $(\alpha'')^\perp$ se anula. Por tanto, nos quedamos únicamente con la componente tangente:
\[
L'_i(0) = \left[ \langle Z(s), \alpha'(s) \rangle \right]_{s_{i-1}}^{s_i} - \int_{s_{i-1}}^{s_i} \left\langle Z(s), \frac{D\alpha'}{ds}(s) \right\rangle ds
\]

Entonces, sumando para todos los tramos $i = 1, \dots, k$, obtenemos la variación total de la longitud $L'(0) = \sum_{i=1}^k L'_i(0)$:
\begin{align*}
L'(0) &= \sum_{i=1}^k \left( \langle Z(s_i), \alpha'_-(s_i) \rangle - \langle Z(s_{i-1}), \alpha'_+(s_{i-1}) \rangle \right) - \underbrace{\sum_{i=1}^k \int_{s_{i-1}}^{s_i} \left\langle Z, \frac{D\alpha'}{ds} \right\rangle ds}_{\int_0^l \langle Z, \frac{D\alpha'}{ds} \rangle \, ds}
\end{align*}
Donde $\alpha'_-(s_i)$ y $\alpha'_+(s_{i-1})$ denotan las velocidades por la izquierda y por la derecha en los respectivos vértices, ya que la curva es solo regular a trozos.

Falta desarrollar la primera parte (la suma telescópica con saltos). Extrayendo los términos de los extremos $0$ y $l$, y reordenando los índices:
\begin{align*}
\sum_{i=1}^k \big( \langle Z(s_i), &\alpha'_-(s_i) \rangle - \langle Z(s_{i-1}), \alpha'_+(s_{i-1}) \rangle \big) \\
&= \langle Z(l), \alpha'(l) \rangle + \sum_{i=1}^{k-1} \langle Z(s_i), \alpha'_-(s_i) \rangle - \langle Z(0), \alpha'(0) \rangle - \sum_{i=2}^k \langle Z(s_{i-1}), \alpha'_+(s_{i-1}) \rangle \\
\intertext{Haciendo un cambio de índice en el último sumatorio para que coincida con el anterior:}
&= \langle Z(l), \alpha'(l) \rangle - \langle Z(0), \alpha'(0) \rangle + \sum_{i=1}^{k-1} \langle Z(s_i), \alpha'_-(s_i) \rangle - \sum_{i=1}^{k-1} \langle Z(s_i), \alpha'_+(s_i) \rangle \\
&= \left[ \langle Z, \alpha' \rangle \right]_0^l - \sum_{i=1}^{k-1} \left\langle Z(s_i), \underbrace{\alpha'_+(s_i) - \alpha'_-(s_i)}_{\Delta_i \alpha'} \right\rangle
\end{align*}

Sustituyendo esto en la expresión general de $L'(0)$, concluimos la demostración:
\[
L'(0) = - \int_0^l \left\langle Z, \frac{D\alpha'}{ds} \right\rangle ds + \left[ \langle Z, \alpha' \rangle \right]_0^l - \sum_{i=1}^{k-1} \langle Z(s_i), \Delta_i \alpha' \rangle
\]
\end{proof}



\begin{teorema}{Caracterización variacional de las geodésicas}
Una curva regular a trozos p.p.a. $\alpha$ es un segmento de geodésica si, y solo si, $L'(0) = 0$ 
para toda variación propia de $\alpha$.
\end{teorema}

Esto implica que las geodésicas son los \textbf{puntos críticos} del funcional longitud.

\begin{proposicion}{Existencia de variaciones para un campo dado}
Sean $\alpha : [0, \ell] \longrightarrow S$ una curva regular a trozos p.p.a. en una superficie regular $S$ y $0 = s_0 < s_1 < \dots < s_k = \ell$ una partición de $[0, \ell]$ 
tal que $\alpha|_{[s_{i-1}, s_i]}$ es regular para cada $i = 1, \dots, k$. 
Sea $Z$ un campo de vectores tangente cualquiera a lo largo de $\alpha$, continuo en $[0, \ell]$ y diferenciable en cada subintervalo $[s_{i-1}, s_i]$.
\begin{enumerate}[label=\roman*)]
    \item Entonces existe una variación $\phi$ de $\alpha$ cuyo campo variacional es $Z$.
    \item Si $Z(0) = \mathbf{0}$ y $Z(\ell) = \mathbf{0}$, se puede elegir $\phi$ de forma que sea variación propia.
\end{enumerate}
\end{proposicion}
\begin{proof}[Demostración de la proposición]
    Lo vamos a demostrar en la hoja de problemas. 
\end{proof}


\begin{proof}[Demostración del teorema de caracterización variacional de las geodésicas]
La demostración consta de dos implicaciones.

\vspace{0.3cm}
\noindent $\Longrightarrow]$ \textbf{Supongamos que $\alpha$ es geodésica.} 

Por definición de geodésica, la curva $\alpha$ es regular en todo su dominio (no tiene vértices) y su aceleración geodésica es nula, es decir, $\frac{D\alpha'}{ds} = 0$. 
Además, como estamos considerando una variación propia $\phi$, los extremos de la curva permanecen fijos durante la deformación, lo que implica que el campo variacional se anula en los extremos: $Z(0) = Z(l) = 0$. Al no existir vértices, los términos de salto $\Delta_i \alpha'$ también son nulos.
Sustituyendo estas condiciones en la fórmula de la primera variación de la longitud (obtenida en la proposición anterior), resulta de manera directa que:
\[
L'(0) = - \int_0^l \left\langle Z, \underbrace{\frac{D\alpha'}{ds}}_{=0} \right\rangle ds + \underbrace{\langle Z(l), \alpha'(l) \rangle}_{=0} - \underbrace{\langle Z(0), \alpha'(0) \rangle}_{=0} - \sum_{i=1}^{k-1} \langle Z(s_i), \underbrace{\Delta_i \alpha'}_{=0} \rangle = 0.
\]

\vspace{0.3cm}
\noindent $\Longleftarrow]$ \textbf{Supongamos que $L'(0) = 0$ para toda variación propia.}

Como la variación es propia, los términos de frontera desaparecen ($Z(0)=Z(l)=0$), y nuestra hipótesis de partida se reduce a la ecuación:
\begin{equation} \label{eq:hipotesis_variacion}
0 = L'(0) = - \sum_{i=1}^{k-1} \langle Z(s_i), \Delta_i \alpha' \rangle - \int_0^l \left\langle Z, \frac{D\alpha'}{ds} \right\rangle ds
\end{equation}
para cualquier campo variacional $Z$ a lo largo de $\alpha$ que se anule en los extremos. Y esto se tiene que cumplir para toda variación propia. 

La demostración de que $\alpha$ es geodésica la dividiremos en dos pasos:

\textbf{Paso 1: Demostrar que $\alpha$ es geodésica a trozos.} \\
Fijemos un intervalo de regularidad cualquiera $(s_{i-1}, s_i)$. Sea $f: [0, l] \to \mathbb{R}$ una función diferenciable tal que $f(s) > 0$ 
para todo $s \in (s_{i-1}, s_i)$ y $f(s) = 0$ en el resto del dominio $[0, l]$.

Construimos el siguiente campo vectorial a lo largo de $\alpha$:
\[
Z(s) = f(s) \frac{D\alpha'}{ds}(s) \in \mathfrak{X}(\alpha)
\]
Por propiedades conocidas de las variaciones, existe una variación propia $\phi$ de $\alpha$ que tiene a este $Z$ como campo variacional asociado. Evaluamos nuestra hipótesis \eqref{eq:hipotesis_variacion} para este campo $Z$ en particular:
\[
0 = - \sum_{j=1}^{k-1} \langle \underbrace{Z(s_j)}_{=0}, \Delta_j \alpha' \rangle - \int_0^l \left\langle f(s) \frac{D\alpha'}{ds}, \frac{D\alpha'}{ds} \right\rangle ds
\]
La sumatoria se anula porque la función $f$ vale cero en todos los vértices $s_j$. La integral, al anularse el integrando fuera de $(s_{i-1}, s_i)$, se reduce a:
\[
0 = - \int_{s_{i-1}}^{s_i} f(s) \left| \frac{D\alpha'}{ds} \right|^2 ds
\]
Dado que $f(s) > 0$ en el intervalo de integración y la norma al cuadrado es siempre no negativa ($\left| \frac{D\alpha'}{ds} \right|^2 \ge 0$), la única forma de que la integral de una función no negativa sea cero es que el integrando sea idénticamente nulo casi por todas partes. Al ser las funciones continuas, deducimos que:
\[
\frac{D\alpha'}{ds} = 0 \quad \text{en } (s_{i-1}, s_i).
\]
Como este razonamiento es válido para cualquier subintervalo, concluimos que $\alpha$ es una geodésica a trozos.

\vspace{0.3cm}
\textbf{Paso 2: Demostrar que no existen vértices.} \\
Ahora debemos probar que $\alpha$ es globalmente regular, es decir, que no hay cambios bruscos de dirección en los vértices: $\alpha'_+(s_i) = \alpha'_-(s_i)$, o equivalentemente, $\Delta_i \alpha' = 0$ para todo $i$.

Fijemos un vértice arbitrario $s_i$. Mediante el uso de particiones de la unidad y transporte paralelo, es posible construir un campo vectorial diferenciable $Z \in \mathfrak{X}(\alpha)$ tal que en dicho vértice tome exactamente el valor del salto, $Z(s_i) = \Delta_i \alpha'$, y que se anule en todos los demás vértices, $Z(s_j) = 0$ para todo $j \neq i$, así como en los extremos de la curva.
Tomando la variación propia asociada a este nuevo campo $Z$ y aplicando de nuevo la hipótesis \eqref{eq:hipotesis_variacion}:
\[
0 = - \sum_{j=1}^{k-1} \langle Z(s_j), \Delta_j \alpha' \rangle - \int_0^l \left\langle Z, \underbrace{\frac{D\alpha'}{ds}}_{=0} \right\rangle ds
\]
La integral se anula completamente porque, según demostramos en el Paso 1, $\frac{D\alpha'}{ds} = 0$ en el interior de cada subintervalo. La sumatoria colapsa a un único término, correspondiente al vértice $s_i$ que hemos fijado:
\[
0 = - \langle Z(s_i), \Delta_i \alpha' \rangle = - \langle \Delta_i \alpha', \Delta_i \alpha' \rangle = - |\Delta_i \alpha'|^2
\]
Esto implica que $|\Delta_i \alpha'|^2 = 0$, de donde se deduce ineludiblemente que $\Delta_i \alpha' = 0$.

Al ser nulo el salto en cada vértice, la derivada $\alpha'$ es continua en todo el dominio $[0, l]$. Por lo tanto, la curva está formada por un único "trozo" regular. Queda así demostrado que $\alpha$ es, en efecto, una geodésica global.
\end{proof}

\begin{definicion}{Variación normal}
Sea $\alpha : [0, \ell] \longrightarrow S$ una curva regular a trozos en una superficie regular $S$. Una variación de $\alpha$ se dice \textbf{normal} si su campo variacional $Z$ es normal a $\alpha$, esto es, $\langle Z(s), \alpha'(s) \rangle = 0$.
\end{definicion}

\begin{corolario}{}
Sea $\alpha: [a,b] \longrightarrow S$ una curva regular p.p.a. en una superficie regular $S$. Entonces $\alpha$ es un segmento de geodésica de $S$ si, y solo si, $L'(0) = 0$ para toda variación \textbf{\textcolor{red}{normal}} $\phi$ de la curva $\alpha$.
\end{corolario}

\begin{observacion}{Justificación del uso de variaciones normales}
En el teorema general de variaciones se habla de cualquier variación, pero aquí se puede exigir únicamente que sea una variación \textbf{normal}. La justificación matemática (como indican las notas manuscritas) es la siguiente:

Si para la demostración construimos el campo variacional ortogonal clásico:
\[
Z(s) = f(s) \frac{D\alpha'}{ds}
\]
podemos asegurar que este campo \textbf{es un campo normal} a la curva $\alpha$.

\textbf{Demostración de este hecho:}
Como la curva $\alpha$ está parametrizada por longitud de arco (p.p.a.), su vector velocidad tiene norma constante igual a 1. Es decir:
\[
\langle \alpha', \alpha' \rangle = 1
\]
Si derivamos esta expresión a lo largo de la curva (usando la derivada covariante):
\[
\frac{d}{ds} \langle \alpha', \alpha' \rangle = 2 \left\langle \alpha', \frac{D\alpha'}{ds} \right\rangle = 0 \implies \left\langle \alpha', \frac{D\alpha'}{ds} \right\rangle = 0
\]
Multiplicando por la función escalar $f(s)$:
\[
\left\langle \alpha', f(s) \frac{D\alpha'}{ds} \right\rangle = 0 \implies \langle \alpha', Z(s) \rangle = 0
\]

\textbf{Conclusión:}
Como el producto escalar del campo $Z$ con el vector tangente $\alpha'$ es cero, significa que \textit{``la parte tangencial se va''}. El campo $Z$ no tiene componente en la dirección de la curva, por lo que es un campo estrictamente normal, y la variación asociada $\phi$ será una variación normal.
\end{observacion}

\begin{teorema}{Segunda fórmula de variación}
Sea $\gamma: [0,l] \longrightarrow S$ una geodésica p.p.a. (por tanto, $L'(0)=0$). Para una variación propia y normal con campo $Z$, la segunda derivada de la longitud es:
\[
L''(0) = \int_0^l \left[ \left| \frac{DZ}{ds} \right|^2 - K(\gamma(s)) |Z(s)|^2 \right] ds.
\]
\end{teorema}
\textit{Nota:} Aquí aparece explícitamente la curvatura de Gauss $K$. Si $K < 0$, entonces $L''(0) > 0$, lo que sugiere que en superficies de curvatura negativa las geodésicas minimizan la longitud (son mínimos locales estables).

% -------------------------------------------------------------------------
% --- SECCIÓN: INTEGRACIÓN EN SUPERFICIES ---

\section{Integración en superficies}

\begin{definicion}{Elemento de área}
Sea $S \subset \mathbb{R}^{3}$ una superficie regular y orientada, con aplicación de Gauss $N$. Se denomina \textbf{\textcolor{mainred}{elemento de área}} de $S$ en un punto $p \in S$ a la aplicación $dA(p) : T_{p}S \times T_{p}S \longrightarrow \mathbb{R}$ dada por $dA(p)(\mathbf{v}, \mathbf{w}) = \det(\mathbf{v}, \mathbf{w}, N(p))$.
\end{definicion}

\begin{observacion}{MUCHO OJO}
    No confundir con una diferencial o con el operador forma (Operador Weingarten).\\
    $dA(p)$ es una forma bilineal antisimétrica.\\
    $dA(p)(X_{u}(q), X_{v}(q)) = \sqrt{EG - F^{2}}(q)$.
\end{observacion}

\begin{proof}[Demostración de la tercera observación]
Sea $S$ una superficie regular orientada y sea $X: U \subset \mathbb{R}^2 \longrightarrow S$ una parametrización de $S$ en un entorno del punto $p = X(q)$, donde $q = (u,v) \in U$.

Por definición topológica y algebraica, el elemento de área $dA$ (la 2-forma de área asociada a la métrica y a la orientación de $S$) evaluado en un par de vectores tangentes $\vec{w}_1, \vec{w}_2 \in T_pS$ devuelve el volumen con signo del paralelepípedo formado por $\vec{w}_1, \vec{w}_2$ y el vector normal unitario $N(p)$. Es decir, se define mediante el producto mixto:
$$dA(p)(\vec{w}_1, \vec{w}_2) = \langle \vec{w}_1 \times \vec{w}_2, N(p) \rangle$$

Para demostrar la observación, evaluamos esta 2-forma en los vectores de la base coordenada del plano tangente, $\{X_u(q), X_v(q)\}$:
$$dA(p)(X_u(q), X_v(q)) = \langle X_u(q) \times X_v(q), N(p) \rangle$$

Sabemos que el campo normal unitario $N(p)$ que orienta la superficie de forma compatible con la parametrización está dado precisamente por el producto vectorial normalizado de la base coordenada:
$$N(p) = \frac{X_u(q) \times X_v(q)}{|X_u(q) \times X_v(q)|}$$

Sustituyendo la expresión de $N(p)$ en nuestra evaluación de la 2-forma:
$$dA(p)(X_u(q), X_v(q)) = \left\langle X_u(q) \times X_v(q), \frac{X_u(q) \times X_v(q)}{|X_u(q) \times X_v(q)|} \right\rangle$$

Por la bilinealidad (o linealidad en este caso) del producto escalar, podemos extraer el escalar del denominador:
$$dA(p)(X_u(q), X_v(q)) = \frac{1}{|X_u(q) \times X_v(q)|} \langle X_u(q) \times X_v(q), X_u(q) \times X_v(q) \rangle$$

Recordando que $\langle \vec{v}, \vec{v} \rangle = |\vec{v}|^2$ para cualquier vector en $\mathbb{R}^3$, el numerador se convierte en el cuadrado de la norma:
$$dA(p)(X_u(q), X_v(q)) = \frac{|X_u(q) \times X_v(q)|^2}{|X_u(q) \times X_v(q)|} = |X_u(q) \times X_v(q)|$$

Físicamente y geométricamente, esto corrobora que el valor de la 2-forma sobre la base es el área del paralelogramo tangente que expanden $X_u$ y $X_v$. Ahora, para conectarlo con la geometría intrínseca de la superficie, utilizamos la \textbf{Identidad de Lagrange}:
$$|\vec{a} \times \vec{b}|^2 = |\vec{a}|^2 |\vec{b}|^2 - \langle \vec{a}, \vec{b} \rangle^2$$

Aplicando esta identidad de álgebra vectorial a nuestros vectores tangentes:
$$|X_u(q) \times X_v(q)|^2 = |X_u(q)|^2 |X_v(q)|^2 - \langle X_u(q), X_v(q) \rangle^2$$

A continuación, recordamos la definición de los coeficientes de la Primera Forma Fundamental:
\begin{itemize}
    \item $E(q) = \langle X_u(q), X_u(q) \rangle = |X_u(q)|^2$
    \item $F(q) = \langle X_u(q), X_v(q) \rangle$
    \item $G(q) = \langle X_v(q), X_v(q) \rangle = |X_v(q)|^2$
\end{itemize}

Sustituyendo directamente estos coeficientes métricos en la identidad de Lagrange desarrollada:
$$|X_u(q) \times X_v(q)|^2 = E(q)G(q) - (F(q))^2$$

Tomando la raíz cuadrada positiva (puesto que la norma representa un área geométrica real, siempre no negativa):
$$|X_u(q) \times X_v(q)| = \sqrt{EG - F^2}(q)$$

Por lo tanto, enlazando el principio con el final del desarrollo, queda formalmente demostrado que:
$$dA(p)(X_u(q), X_v(q)) = \sqrt{EG - F^2}(q)$$
\end{proof}

\begin{observacion}{Relación de áreas mediante la diferencial}
Con la diferencial "pasamos" del plano a la superficie:



Si tomamos un pequeño rectángulo $R$ de lados $\Delta u, \Delta v$ en el dominio de los parámetros a partir de un punto $q$, la diferencial linealiza esta transformación. Se tiene para la dirección $u$:
\[
dX_q((\Delta u, 0)) = \Delta u \underbrace{dX_q(1,0)}_{X_u}
\]

El área del paralelogramo $R'$ formado en el plano tangente $T_pS$ por la imagen de los lados del rectángulo se calcula mediante la norma del producto vectorial:
\[
A(R') = |dX_q(\Delta u, 0) \times dX_q(0, \Delta v)| = \Delta u \Delta v |X_u \times X_v| = \underbrace{\Delta u \Delta v}_{A(R)} \sqrt{EG-F^2}
\]

Al pasar al límite (haciendo que el rectángulo sea infinitamente pequeño), el área del paralelogramo tangente $A(R')$ y el área de la región curva real sobre la superficie $A(\tilde{R})$ se igualan, obteniéndose el siguiente límite:
\[
\lim_{\Delta u, \Delta v \to 0} \frac{A(\tilde{R})}{A(R)} = \sqrt{EG-F^2}
\]

\textbf{Conclusión:}
Las áreas entre el plano de parámetros y la superficie están relacionadas (distorsionadas) localmente mediante el factor $\sqrt{EG-F^2}$. Esto es lo que justifica que, al integrar, el elemento diferencial de área sea $dA = \sqrt{EG-F^2} \, du \, dv$.
\end{observacion}

\begin{center}
    \includegraphics[width=0.7\textwidth]{Imagenes/AreaSuperficies.png}
\end{center}

%%%% AREA DE UNA REGION %%%

\begin{definicion}{Área de una región}
El área de una región $R \subset S$ viene dada por la expresión
\[
A(R) = \iint_{X^{-1}(R)} \sqrt{EG - F^{2}} \, du \, dv,
\]
donde $(U, X)$ es cualquier parametrización tal que $X(U)$ contiene a $R$.
\end{definicion}

\begin{definicion}{Soporte de una función}
Sea $f : S \longrightarrow \mathbb{R}$ una función real definida sobre una superficie regular orientada $S$. El \textbf{\textcolor{mainred}{soporte}} (compacto) de $f$ es el conjunto
\[
\text{sop}(f) = \text{cl}\{p \in S : f(p) \neq 0\}.
\]
\end{definicion}




\begin{observacion}{El problema de integrar funciones sobre superficies}
Si se tiene una función escalar $f: S \longrightarrow \mathbb{R}$ definida sobre una superficie, nos planteamos cómo calcular su integral:
\[
\int_S f \, dA \quad \text{?}
\]

``Nos vamos'' al plano utilizando una parametrización local $X: \mathcal{U} \subset \mathbb{R}^2 \longrightarrow S$. Podríamos intentar definir la integral simplemente arrastrando la función:
\[
\int_S f \, dA \overset{?}{=} \iint_{\mathcal{U}} (f \circ X)(u,v) \, du \, dv
\]
Se puede hacer esto, pero... \textbf{¿es una buena definición?} Es decir, ¿depende de la parametrización?

\textbf{¡Pues es una castaña de definición!} ¿Por qué?

Si tenemos dos cartas distintas:
Sea $\phi = \bar{X}^{-1} \circ X : \mathcal{U} \longrightarrow \bar{\mathcal{U}}$ el cambio de coordenadas, tal que $(\bar{u}, \bar{v}) = \phi(u,v)$. 

Aplicando el Teorema del Cambio de Variable para integrales dobles, se tiene:
\[
\iint_{\bar{\mathcal{U}}} (f \circ \bar{X})(\bar{u}, \bar{v}) \, d\bar{u} \, d\bar{v} = \iint_{\phi^{-1}(\bar{\mathcal{U}})} (f \circ \bar{X})(\phi(u,v)) \, |J\phi| \, du \, dv
\]

Sabiendo que $\phi^{-1}(\bar{\mathcal{U}}) = \mathcal{U}$ y que la composición es $(f \circ \bar{X}) \circ \phi = f \circ (\bar{X} \circ \bar{X}^{-1} \circ X) = f \circ X$, sustituimos y nos queda:
\[
\iint_{\bar{\mathcal{U}}} (f \circ \bar{X})(\bar{u}, \bar{v}) \, d\bar{u} \, d\bar{v} = \iint_{\mathcal{U}} \underbrace{(f \circ \bar{X})(\phi(u,v))}_{f \circ X} \underbrace{|J\phi|}_{\substack{\text{pero esto se} \\ \text{queda ahí}}} \, du \, dv \neq \iint_{\mathcal{U}} (f \circ X)(u,v) \, du \, dv
\]

Entonces \textbf{no es una buena definición}, debe aparecer el elemento de área por algún lado, luego...
\end{observacion}

\begin{definicion}{Integral de $f$ en $S$ (Caso Local)}
Sea $f : S \longrightarrow \mathbb{R}$ una función con soporte compacto, definida sobre una superficie regular orientada $S$, tal que existe una parametrización $(U, X)$ de $S$ de forma que $\text{sop}(f) \subset X(U)$. Se define la \textbf{\textcolor{mainred}{integral de $f$ en $S$}} como
\[
\int_{S} f \, dA := \iint_{U} (f \circ X)(u, v) \sqrt{EG - F^{2}}(u, v) \, du \, dv
\]
(siempre y cuando la integral sobre $U \subset \mathbb{R}^{2}$ esté definida).
\end{definicion}
Hay que ver que esta definición no depende de la parametrización. Para ello, basta con aplicar el Teorema del Cambio de Variable y la fórmula de transformación de los coeficientes métricos $E, F, G$ bajo cambios de coordenadas. Lo veremos en la hoja de problemas.
Ahora bien, \textbf{¿Y si el soporte de $f$ no está contenido en un entorno coordenado?}

\begin{definicion}{Partición de la unidad}
Una \textbf{\textcolor{mainred}{partición}} (diferenciable) \textbf{\textcolor{mainred}{de la unidad}} en una superficie regular $S$ es una colección de funciones (diferenciables) $f_{i} : S \longrightarrow \mathbb{R}$, $i = 1, \dots, n$,
 tales que $0 \le f_{i} \le 1$, $i = 1, \dots, n$ y $\sum_{i=1}^{n} f_{i} \equiv 1$. 
 La partición $\{f_{i}\}_{i=1}^{n}$ está \textbf{subordinada a un cubrimiento abierto} de $S$, $\{V_{1}, \dots, V_{n}\}$, si $\text{sop}(f_{i}) \subset V_{i}$, $i = 1, \dots, n$.\\
Recordemos que un \textbf{cubrimiento} de una superficie es coger un  montón de parametrizaciones de forma que al final los 
entornos coordenados (imagen de cada una de nuestras parametrizaciones) me lo cubran todo.
\end{definicion}

\begin{observacion}{Particiones de la unidad y soporte de funciones}
\begin{itemize}
    \item Dado un cubrimiento finito $\{V_1, \ldots, V_n\}$ por abiertos de $S$, siempre existe una partición (diferenciable) de la unidad subordinada a éste.
    \item Sea $\{g, g_1, \ldots, g_n\}$ una partición de la unidad subordinada al cubrimiento $\{S \setminus \text{sop}(f), X_1(U_1), \ldots, X_n(U_n)\}$, y sea $f_i := f g_i$. Entonces $f = \sum_{i=1}^n f_i$ y $\text{sop}(f_i) \subset X_i(U_i)$ para todo $i = 1, \ldots, n$.
\end{itemize}

\vspace{0.3cm}
\textbf{Justificación (Desarrollo lógico de los soportes):}

Por definición de partición de la unidad subordinada al cubrimiento dado, sabemos dónde se localiza el soporte de cada función de la partición:
\begin{itemize}
    \item Soporte de $g_1$ en $X_1(U_1)$.
    \item Soporte de $g_2$ en $X_2(U_2)$.
    \item $\vdots$
    \item Soporte de $g$ en $S \setminus \text{sop}(f)$.
\end{itemize}

Dado que las funciones de la partición de la unidad suman $1$ en todo punto, podemos reescribir la función $f$ en un punto $p$ de la siguiente manera:
\[
f(p) = f(p) \cdot 1 = f(p) \left( g(p) + \sum g_i(p) \right) = \underbrace{f(p)g(p)}_{0} + \sum \underbrace{f \cdot g_i}_{f_i}(p) = \sum f_i(p)
\]

\textbf{¿Por qué el término $f(p) \cdot g(p) = 0$?}
\begin{itemize}
    \item Si $f(p) \neq 0 \implies p \in \text{sop}(f)$. Pero por construcción, sabemos que $\text{sop}(g) \subset S \setminus \text{sop}(f)$, lo que implica que $p$ está fuera del soporte de $g$, y por tanto $g(p) = 0$.
    \item (Si $f(p) = 0$ ya está, el producto es trivialmente cero).
\end{itemize}

Finalmente, para ver que cada $f_i$ se queda dentro de su respectivo entorno coordenado, analizamos su soporte:
Y se tiene: 
\[
\text{sop}(f_i) = \text{sop}(f \cdot g_i) = (\text{sop}(f)) \cap (\text{sop}(g_i)) \subset X_i(U_i) \quad \square
\]
\end{observacion}

\begin{definicion}{Integral de $f$ en $S$ (Caso Global)}
Se define la \textbf{\textcolor{mainred}{integral de $f$ en $S$}} como
\[
\int_{S} f \, dA := \sum_{i=1}^{n} \int_{S} f_{i} \, dA.
\]
\end{definicion}

% --- OBSERVACIONES SOBRE EL CÁLCULO DE ÁREAS ---

\begin{tcolorbox}[colback=maingreen!5!white, colframe=maingreen!75!black, title=Observaciones sobre la extensión del área]
Es importante notar que, por definición, no podemos calcular directamente el área de superficies enteras que no estén acotadas (como el cilindro o el cono) utilizando una sola región $R$. 

\begin{itemize}
    \item \textbf{Superficies compactas:} En estos casos, las superficies son regiones $R$ en sí mismas, lo que facilita el cálculo global del área.
    \item \textbf{Casos especiales:} Existen superficies no acotadas, como la \textbf{pseudoesfera}, donde el cálculo del área sí es posible, aunque constituye un caso excepcional.
    \item \textbf{Estrategia de cálculo:} Lo ideal es emplear parametrizaciones que cubran la superficie salvo un conjunto de \textbf{medida nula}. De esta forma, evitamos la necesidad de recurrir explícitamente a particiones de la unidad muy complejas.
\end{itemize}
\end{tcolorbox}

% --- EJEMPLO: ÁREA DE LA ESFERA ---

\begin{ejemplo}{Cálculo del área de la esfera $\mathbb{S}^2$}
Queremos calcular el área total de la esfera de radio $r$. Para ello, integramos la función constante $f \equiv 1$ sobre toda la superficie:
\[
A(\mathbb{S}^2) = \int_{\mathbb{S}^2} 1 \, dA.
\]
Como $sop(1) = \mathbb{S}^2$, necesitaríamos una parametrización que cubra toda la esfera. Dado que esto es topológicamente imposible con una sola carta, utilizamos una parametrización que deje fuera únicamente un conjunto de medida nula (los meridianos/polos):

Sea la parametrización en coordenadas geográficas:
\[
X(\theta, \varphi) = (r \cos\theta \sin\varphi, \, r \sin\theta \sin\varphi, \, r \cos\varphi)
\]
con $(\theta, \varphi) \in (0, 2\pi) \times (0, \pi)$. 

Calculando los coeficientes de la primera forma fundamental, obtenemos el elemento de área:
\[
\sqrt{EG - F^2} = r^2 \sin\varphi.
\]
*(Nota: En tus apuntes aparece $\sin\theta$, dependiendo de la asignación de ángulos en la parametrización)*.

Finalmente, el área resulta:
\[
A(\mathbb{S}^2) = \int_{0}^{\pi} \int_{0}^{2\pi} r^2 \sin\varphi \, d\theta \, d\varphi = 4\pi r^2.
\]
\end{ejemplo}

\begin{teorema}{Teorema del cambio de variable}
Sean $S_{1}$ y $S_{2}$ dos superficies regulares, conexas y orientadas, cuyos elementos de área representamos por $dA_{1}$ y $dA_{2}$, respectivamente. Sea $f : S_{2} \longrightarrow \mathbb{R}$ una función con soporte compacto. Si $\phi : S_{1} \longrightarrow S_{2}$ es un difeomorfismo, entonces
\[
\int_{S_{2}} f \, dA_{2} = \int_{S_{1}} (f \circ \phi) \, |\det(d\phi)| \, dA_{1}.
\]
\end{teorema}
\begin{proof}
    No hay demostración
\end{proof}
\begin{proposicion}{Corolario}
Sea $S$ una superficie regular orientada, $N$ su aplicación de Gauss y $p \in S$ con $K(p) \neq 0$. Si $V \subset S$ es un entorno de $p$ donde $N|_{V} : V \longrightarrow N(V)$ es un difeomorfismo, entonces
\[
A(N(V)) = \int_{V} |K| \, dA,
\]
siempre y cuando esta integral tome un valor finito.
\end{proposicion}

\textit{Observación: Si $K(p) \neq 0$, siempre puedo encontrar el entorno de $p$ donde tengo el difeomorfismo.}


% --- SECCIÓN: VARIACIONES DEL ÁREA ---

\section{Variaciones del área. Las superficies minimales}

\begin{definicion}{Variación normal de una superficie}
Sean $U \subset \mathbb{R}^{2}$ un abierto y $X : U \longrightarrow \mathbb{R}^{3}$ una aplicación tales que $X(U)$ es una superficie regular. Si $D \subset \mathbb{R}^{2}$ es un disco abierto cuya clausura $\text{cl } D \subset U$, entonces $X(D)$ es una superficie regular para la cual $X|_{D}$ es una parametrización.

Sea $h : \text{cl } D \longrightarrow \mathbb{R}$ una función diferenciable que se anula en la frontera de $D$. Se llama \textbf{\textcolor{mainred}{variación normal}} de $X(D)$ determinada por $h$ a la aplicación $\phi : \text{cl } D \times (-\varepsilon, \varepsilon) \longrightarrow \mathbb{R}^{3}$ definida por
\[
\phi(u, v, t) = X(u, v) + t h(u, v) N(X(u, v)),
\]
donde $N$ es el vector normal unitario y $\varepsilon > 0$ es suficientemente pequeño para que la aplicación $X^{t} : D \longrightarrow \mathbb{R}^{3}$ dada por $X^{t}(u, v) = \phi(u, v, t)$ sea una parametrización de la superficie regular $X^{t}(D)$ para todo $t \in (-\varepsilon, \varepsilon)$.
\end{definicion}


Una variación $\phi$ de $X(D)$ genera una familia uniparamétrica $\{X^{t}(D)\}_{t \in (-\varepsilon, \varepsilon)}$ de superficies regulares. El área de estas superficies da lugar al llamado \textbf{\textcolor{mainred}{funcional área}}:
\[
A(t) = A(X^{t}(D)).
\]

\begin{observacion}{}
Todo esto realmente es sencillo. 
Cogemos un disco $D$ (por comodidad) con la clausura del disco contenida en el abierto $\mathcal{U}$ (es decir, $\bar{D} \subset \mathcal{U}$). 
Se tiene que $\mathbf{X}(D)$ es una superficie regular y la restricción $\mathbf{X}|_D$ es una parametrización. Le creamos una \textbf{variación normal} a esta superficie $\mathbf{X}(D)$.

\textbf{Analogía geométrica:} Es como una cama elástica, cada ``posición'' o deformación de la tela sería una variación normal de la superficie.

Esa pequeña variación nos la da el parámetro $t \in (-\varepsilon, \varepsilon)$ (se varía estrictamente en la dirección del vector normal). 

Y todas esas parametrizaciones deformadas nos dan una \textbf{familia} de superficies.
\end{observacion}

\begin{center}
    \includegraphics[width=0.7\textwidth]{Imagenes/VariacionNormalSup.png}
\end{center}

% --- DEFINICIONES DE SUPERFICIES MINIMALES ---

\begin{definicion}{Caracterización Variacional (I)}
Una superficie regular $S$ es \textbf{\textcolor{mainred}{minimal}} si, para cada punto de la misma, existe un entorno $\Omega$ tal que éste es la superficie de menor área entre todas aquellas que tienen como frontera la frontera de $\Omega$.
\end{definicion}

\begin{definicion}{Ecuación en Derivadas Parciales (II)}
Una superficie es \textbf{\textcolor{mainred}{minimal}} si la función diferenciable $f$ que determina su gráfica verifica la \textbf{ecuación de Euler-Lagrange}:
\[
f_{xx}(1+f_{y}^{2}) - 2f_{xy}f_{x}f_{y} + f_{yy}(1+f_{x}^{2}) = 0.
\]
\end{definicion}

\begin{definicion}{Curvatura Media (III)}
Una superficie es \textbf{\textcolor{mainred}{minimal}} si su \textbf{curvatura media} se anula en todo punto, esto es, si $H \equiv 0$.
\end{definicion}

\begin{definicion}{Aplicación de Gauss (IV)}
Se llama \textbf{\textcolor{mainred}{superficie minimal}} a aquella cuya aplicación de Gauss es \textbf{conforme}\footnote{conserva ángulos.} y no es una esfera.
\end{definicion}

\begin{definicion}{Parametrización Isoterma (V)}
La superficie determinada por una parametrización \textbf{\textcolor{mainred}{isoterma}} $X(u,v) = (x(u,v), y(u,v), z(u,v))$ (es decir, tal que $E=G$ y $F=0$) es \textbf{\textcolor{mainred}{minimal}} si las funciones coordenadas son \textbf{armónicas}, esto es, si:
\[
\Delta x = \Delta y = \Delta z = 0.
\]
\end{definicion}

\begin{definicion}{Membrana Jabonosa (VI)}
Se denomina \textbf{\textcolor{mainred}{superficie minimal}} aquella que se puede reproducir mediante una \textbf{membrana jabonosa}.
\end{definicion}

















\begin{ejemplo}{Ejemplos clásicos}
\begin{itemize}
    \item \textbf{El Plano:} La solución trivial ($K=H=0$).
    \item \textbf{El Catenoide:} Generada al rotar una catenaria. (Euler, 1740: única minimal de revolución).
    \item \textbf{El Helicoide:} Superficie reglada generada por una recta que rota y sube (escalera de caracol). (Meusnier, 1770).
    \item \textbf{Superficies de Scherk:} Definidas implícitamente, por ejemplo, $e^z \cos x - \cos y = 0$.
\end{itemize}
Es interesante notar que el helicoide puede deformarse isométricamente en un catenoide pasando por una familia continua de superficies minimales.
\end{ejemplo}
\subsubsection{Superficies minimales como puntos críticos del área}

\begin{proposicion}{Existencia de funciones meseta}
Sean $D$ un disco abierto de $\mathbb{R}^2$ y $q \in D$. Entonces, para todo $\varepsilon > 0$ tal que $D(q, \varepsilon) \subset D$, existe una función diferenciable $\tilde{h} : D \longrightarrow \mathbb{R}$, llamada \textbf{\textcolor{mainred}{función meseta}}, verificando que:
\begin{itemize}
    \item $\text{sop}(\tilde{h}) \subset D(q, \varepsilon)$,
    \item $0 \le \tilde{h} \le 1$,
    \item $\tilde{h} \equiv 1$ en el disco $D(q, \varepsilon/2)$.
\end{itemize}
\end{proposicion}

\begin{teorema}{Superficies minimales como puntos críticos del área}
Sean $U \subset \mathbb{R}^2$ un abierto y $X : U \longrightarrow \mathbb{R}^3$ una aplicación tal que $X(U)$ es una superficie regular. Sea $D$ un disco abierto en $\mathbb{R}^2$ cuya clausura $\text{cl } D \subset U$. Entonces, $X(D)$ es una superficie minimal si, y solo si, $A'(0) = 0$ para cualquier variación normal $\phi$ de $X(D)$.
\end{teorema}


\begin{proof}
Dada la variación normal $\phi(u, v, t) = X(u, v) + t h(u, v) N(u, v)$, el funcional área es:
\[
A(t) = \iint_{D} \sqrt{E_t G_t - F_t^2} \, du \, dv.
\]
Para calcular $A'(0)$, primero aproximamos los coeficientes de la primera forma fundamental de la variación $X^t$. Tras omitir términos de orden $t^2$ que desaparecerán al derivar y evaluar en $t=0$, obtenemos:
\begin{align*}
E_t &= E - 2th e + t^2 R_1, \\
F_t &= F - 2th f + t^2 R_2, \\
G_t &= G - 2th g + t^2 R_3.
\end{align*}
Sustituyendo en el discriminante y utilizando la definición de curvatura media $H = \frac{gE + eG - 2fF}{2(EG - F^2)}$, se llega a:
\[
\sqrt{E_t G_t - F_t^2} = \sqrt{EG - F^2} \sqrt{1 - 4thH + t^2 \tilde{R}}.
\]
Derivando bajo el signo integral y evaluando en $t=0$:
\[
A'(0) = -2 \iint_{D} h H \sqrt{EG - F^2} \, du \, dv = -2 \int_{X(D)} h H \, dA.
\]

\textbf{($\implies$)} Si $X(D)$ es minimal, $H \equiv 0$, por lo que $A'(0) = 0$ para toda variación.

\textbf{($\impliedby$)} Supongamos que $A'(0) = 0$ para toda $h$, pero existe un punto $p = X(q)$ tal que $H(p) \neq 0$ (p.ej. $H(p) > 0$). Por continuidad, $H > 0$ en un entorno $D(q, r)$. 
Tomamos como función de la variación una \textbf{función meseta} $\tilde{h}$ con soporte en dicho disco. Entonces:
\[
A'(0) = -2 \iint_{D(q,r)} \tilde{h} H \sqrt{EG - F^2} \, du \, dv.
\]
Separando la integral en $D(q, r) \setminus D(q, r')$ y $D(q, r')$, donde $\tilde{h} \equiv 1$:
\begin{itemize}
    \item En la zona exterior, la integral es $\le 0$ (ya que $\tilde{h}, H, \sqrt{\dots}$ son $\ge 0$).
    \item En la zona interior, la integral es estrictamente $< 0$ (pues $H > 0$ y $\tilde{h} = 1$).
\end{itemize}
Esto implica $A'(0) < 0$, lo cual contradice la hipótesis $A'(0) = 0$. Por tanto, $H$ debe ser cero en todo punto.
\end{proof}

