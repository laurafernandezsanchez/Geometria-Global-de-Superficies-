\chapter*{Anexo: Ecuaciones Diferenciales}
\addcontentsline{toc}{chapter}{Anexo: Ecuaciones Diferenciales}


\section*{Ecuación lineal orden $n$ con coeficientes constantes}

Consideremos la EDO lineal homogénea de orden $n$ con coeficientes constantes,
\begin{equation}
    x^{n)} + a_{n-1}x^{n-1)} + \cdots + a_1 x' + a_0 x = 0, 
\end{equation}
con $a_{n-1}, \dots, a_0 \in \mathbb{R}$.

\begin{definicion}{Polinomio característico}
Dada la EDO lineal homogénea de orden $n$ se define su \textbf{polinomio característico} como el polinomio
\begin{equation}
    p(\lambda) = a_0 + a_1\lambda + a_2\lambda^2 + \cdots + a_{n-2}\lambda^{n-2} + a_{n-1}\lambda^{n-1} + \lambda^n. \tag{3.23}
\end{equation}

\end{definicion}


\begin{teorema}{Sistema fundamental de soluciones}
Un sistema fundamental de soluciones de (3.21) está formado por
\begin{equation}
    \{ t^k e^{at} \cos(bt), \, t^k e^{at} \sin(bt) \}, \tag{3.24}
\end{equation}
con $a+bi$, $b \ge 0$, recorriendo el conjunto de los valores propios y siendo $k = 0, \dots, m_a(\lambda) - 1$.

\end{teorema}


\subsection*{Ecuación lineal completa de orden $n$ con coeficientes constantes}

Queda sólo pendiente considerar de forma particular la EDO no homogénea
\begin{equation}
    x^{n)} + a_{n-1}x^{n-1)} + \cdots + a_1 x' + a_0 x = b(t). 
\end{equation}


\begin{teorema}{Método de coeficientes indeterminados}
Supongamos $b(t) = e^{at}(p(t)\cos(bt) + q(t)\sin(bt))$, donde $p, q$ son polinomios de grado a lo sumo $k \ge 0$. Sea $\lambda = a + bi$.

\begin{enumerate}
    \item Si $\lambda$ \textbf{no es raíz} del polinomio característico (3.23), entonces (3.25) tiene una solución particular de la forma
    \[
    e^{at}(r(t)\cos(bt) + s(t)\sin(bt)),
    \]
    con $r, s$ polinomios de grado a lo sumo $k$.

    \item Si $\lambda$ \textbf{es raíz} del polinomio característico (3.23) de multiplicidad $l$, entonces (3.25) tiene una solución particular de la forma
    \[
    t^l e^{at}(r(t)\cos(bt) + s(t)\sin(bt)),
    \]
    con $r, s$ polinomios de grado a lo sumo $k$.
\end{enumerate}

\end{teorema}

\begin{teorema}{Principio de superposición para EDOs}
Consideremos la EDO
\begin{equation}
    x^{n)} + a_{n-1}(t)x^{n-1)} + \cdots + a_1(t)x' + a_0(t)x = b_1(t) + b_2(t) + \cdots + b_r(t), \tag{3}
\end{equation}
con $b_1, \dots, b_r : I \subset \mathbb{R} \to \mathbb{R}$ continuas. Para cada $k = 1, \dots, r$, sea $x_k$ solución particular de
\[
    x^{n)} + a_{n-1}(t)x^{n-1)} + \cdots + a_1(t)x' + a_0(t)x = b_k(t).
\]
Entonces, $x_p = x_1 + \cdots + x_r$ es solución particular de (3).

\end{teorema}