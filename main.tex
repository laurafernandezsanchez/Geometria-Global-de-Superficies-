\documentclass[a4paper,11pt]{book} % Usamos 'book' para tener capítulos

% Importamos la configuración desde el otro archivo
% --- PAQUETES DE IDIOMA Y CODIFICACIÓN ---
\usepackage[utf8]{inputenc}
\usepackage[T1]{fontenc}
\usepackage[spanish,es-tabla]{babel}

% --- PAQUETES MATEMÁTICOS ---
\usepackage{amsmath, amsthm, amssymb, amsfonts}
\usepackage{mathtools}
\usepackage{physics} 
% ...existing code...
\usepackage{wasysym} 


% --- PAQUETES DE DISEÑO Y GRÁFICOS ---
\usepackage{geometry}
\geometry{left=2.5cm, right=2.5cm, top=3cm, bottom=3cm}
\usepackage{xcolor}
\usepackage{enumitem}
% Configuración de enlaces bonitos
\usepackage[colorlinks=true, linkcolor=mainblue, urlcolor=mainblue, citecolor=mainred]{hyperref}
\usepackage{graphicx} % Para imágenes

% --- CONFIGURACIÓN DE TCOLORBOX (Tus cajas bonitas) ---
\usepackage[most]{tcolorbox}

% Definición de colores
\definecolor{mainblue}{RGB}{0, 102, 204}
\definecolor{mainred}{RGB}{204, 0, 0}
\definecolor{maingreen}{RGB}{0, 153, 76}

% Cajas personalizadas
\newtcolorbox[auto counter, number within=section]{definicion}[2][]{
    colback=mainblue!5!white, colframe=mainblue!75!black, fonttitle=\bfseries,
    title=Definición~\thetcbcounter: #2, sharp corners=downhill, enhanced,
    attach boxed title to top left={yshift=-2mm, xshift=2mm},
    boxed title style={colback=mainblue!75!black}, #1
}

\newtcolorbox[auto counter, number within=section]{teorema}[2][]{
    colback=red!5!white, colframe=mainred!75!black, fonttitle=\bfseries,
    title=Teorema~\thetcbcounter: #2, sharp corners=downhill, enhanced,
    attach boxed title to top left={yshift=-2mm, xshift=2mm},
    boxed title style={colback=mainred!75!black}, #1
}
% --- CAJA PARA LEMAS (Lila claro) ---
\newtcolorbox[auto counter, number within=section]{lema}[2][]{
    colback=violet!5!white,          % Fondo lila muy suave
    colframe=violet!60!black,        % Borde violeta oscuro para contraste
    fonttitle=\bfseries,
    title=Lema~\thetcbcounter: #2, 
    sharp corners=downhill, 
    enhanced,
    attach boxed title to top left={yshift=-2mm, xshift=2mm},
    boxed title style={colback=violet!60!black},
    #1
}
% --- CAJA PARA OBSERVACIONES (Verde) ---
\newtcolorbox[auto counter, number within=section]{observacion}[2][]{
    colback=pink!5!white,           % Fondo verde muy suave
    colframe=pink!60!black,         % Borde verde oscuro
    fonttitle=\bfseries,
    title=Observación~\thetcbcounter: #2, 
    sharp corners=downhill, 
    enhanced,
    attach boxed title to top left={yshift=-2mm, xshift=2mm},
    boxed title style={colback=pink!60!black},
    breakable,
    #1
}

% --- CAJA PARA COROLARIOS (Azul Cerceta / Teal) ---
\newtcolorbox[auto counter, number within=section]{corolario}[2][]{
    colback=teal!5!white,            % Fondo azulado muy suave
    colframe=teal!60!black,          % Borde azul oscuro/pavesa
    fonttitle=\bfseries,
    title=Corolario~\thetcbcounter: #2, 
    sharp corners=downhill, 
    enhanced,
    attach boxed title to top left={yshift=-2mm, xshift=2mm},
    boxed title style={colback=teal!60!black},
    #1
}

\newtcolorbox[auto counter, number within=section]{proposicion}[2][]{
    colback=yellow!10!white,         % Amarillo muy claro para el fondo
    colframe=yellow!60!black,        % Amarillo oscuro/dorado para el borde
    fonttitle=\bfseries,
    title=Proposición~\thetcbcounter: #2, 
    sharp corners=downhill, 
    enhanced,
    attach boxed title to top left={yshift=-2mm, xshift=2mm},
    boxed title style={colback=yellow!60!black}, % Color del fondo del título
    #1
}

\newtcolorbox[auto counter, number within=section]{ejemplo}[2][]{
    colback=maingreen!5!white, colframe=maingreen!75!black, fonttitle=\bfseries,
    title=Ejemplo~\thetcbcounter: #2, breakable, enhanced, #1
}

% --- COMANDOS MATEMÁTICOS PERSONALIZADOS ---
\newcommand{\R}{\mathbb{R}}
\newcommand{\Ssf}{\mathbb{S}}
\newcommand{\Xfrak}{\mathfrak{X}}
\newcommand{\inner}[2]{\langle #1, #2 \rangle}

% --- CONFIGURACIÓN DE ENCABEZADO Y PIE DE PÁGINA ---
\usepackage{fancyhdr}
\setlength{\headheight}{15pt} % Evita advertencias de altura

% Activamos el estilo 'fancy'
\pagestyle{fancy}
\fancyhf{} % Borra los ajustes por defecto

% --- CONFIGURACIÓN DEL ENCABEZADO ---
\fancyhead[L]{\small \textbf{Geometría Global de Superficies}} 
\fancyhead[R]{\small \nouppercase{\leftmark}} % Derecha: Nombre del Capítulo actual

% --- CONFIGURACIÓN DEL PIE DE PÁGINA ---
\fancyfoot[L]{\hyperlink{indice}{\textbf{Ir al Índice}}} 
% Modificamos el centro para incluir el enlace:
\fancyfoot[C]{\thepage} 
\fancyfoot[R]{\small UMU -- 2025}

% --- LÍNEAS SEPARADORAS ---
\renewcommand{\headrulewidth}{0.4pt} % Grosor línea superior
\renewcommand{\footrulewidth}{0.4pt} % Grosor línea inferior
% --- CONFIGURACIÓN DE PÁRRAFOS ---
\setlength{\parindent}{0pt}  % Quita la sangría (el hueco al principio)
\setlength{\parskip}{0.8em}  % Añade espacio entre párrafos (opcional pero recomendado)

% Evita que LaTeX estire el espacio verticalmente para llenar la página
\raggedbottom

% --- DATOS DEL DOCUMENTO ---
\title{\textbf{Geometría Global de Superficies}}
\author{Notas de Clase de Laura Fernández}
\date{Curso 2025-2026}

\begin{document}

    % Archivo: portada.tex

\begin{titlepage}
    \centering
    \vspace*{1cm}
    
    % --- ENCADOZADO: UNIVERSIDAD / GRADO ---
    {\scshape\LARGE Universidad de Murcia \par} % O tu universidad
    \vspace{0.3cm}
    {\scshape\Large Grado en Matemáticas \par}
    
    \vspace{1cm}
    
    % --- TÍTULO CON CAJA DE COLOR ---
    % Usamos tcolorbox para un acabado profesional que combina con tus apuntes
    \begin{tcolorbox}[
        colback=mainblue!10!white,    % Fondo azul muy suave
        colframe=mainblue!80!black,   % Borde azul oscuro
        arc=2mm,                      % Bordes ligeramente redondeados
        boxrule=1.5mm,                % Grosor del borde
        halign=center,                % Texto centrado
        width=0.9\textwidth,          % Ancho de la caja
        top=1cm, bottom=1cm           % Espacio vertical interno
    ]
        {\Huge\bfseries\textcolor{mainblue!80!black}{Geometría Global de Superficies}} \\[0.5cm]
        {\Large\itshape Notas de Clase}
    \end{tcolorbox}
    
    \vspace{1cm}
    
    % --- SUBTÍTULO O DESCRIPCIÓN ---
    {\large Basado en las clases de Mª Ángeles Hernández Cifre \par}
    
    \vspace{1cm}
    
    % --- ESPACIO PARA LOGO 
    \includegraphics[width=0.7\textwidth]{Imagenes/logo.jpg}

    % --- PIE DE PÁGINA: AUTOR Y FECHA ---
    \rule{0.6\textwidth}{0.4pt} \\ % Línea horizontal fina
    %\vspace{0.2cm}
    {\Large\bfseries Laura Fernández Sánchez \par}
    \vspace{0.1cm}
    {\large Curso 2025 -- 2026 \par}
    
\end{titlepage}
    % --- AÑADE ESTA LÍNEA ---
    \hypertarget{indice}{} % <-- Esto define el punto de destino
    \tableofcontents

    % Aquí empieza el contenido. 
    % \include inserta el archivo y fuerza un salto de página (ideal para capítulos)

    \chapter{Geodésicas en superficies} 
    
% -------------------------------------------------------------------------
\section{Campos de vectores a lo largo de una curva}

Sea $S$ una superficie regular orientada con aplicación de Gauss $N:S \longrightarrow \mathbb{S}^2$ y sea $\alpha: I \longrightarrow S$ una curva diferenciable.

\begin{definicion}{Campo de vectores a lo largo de una curva}
Sea $S$ una superficie y $N:S\to \mathbb{S}^2$ su aplicación de Gauss. Para una curva $\alpha:I\to S$ diferenciable, \textbf{un campo de vectores a lo largo de $\alpha$} es una aplicación $V: I \longrightarrow \mathbb{R}^3$ tal que
\[
V(t) \in \mathbb{R}^3 = T_{\alpha(t)}S \oplus \text{span}\{N(t)\}.
\]
Se dice que $V$ es \textbf{diferenciable} si lo es como aplicación de $I$ a $\mathbb{R}^3$, es decir, $V \in C^\infty(I, \mathbb{R}^3)$.

Además, diremos que $V$ es \textbf{tangente} en $S$ a lo largo de $\alpha$ si $V(t) \in T_{\alpha(t)}S$ para todo $t \in I$. Denotaremos a la familia de campos tangentes y diferenciables a lo largo de $\alpha$ como $\Xfrak(\alpha)$.
\end{definicion}

\textbf{Observación:} Dado un campo de vectores $V$ cualquiera a lo largo de $\alpha$, siempre podemos descomponerlo en su parte tangencial y normal:
\[
V(t) = V(t)^\top + V(t)^\perp = V(t)^\top + \inner{V(t)}{N(t)}N(t).
\]
Por tanto, la componente tangencial viene dada por $V^\top = V - \inner{V}{N}N \in \Xfrak(\alpha)$.

% -------------------------------------------------------------------------
\section{La derivada covariante}
Si proyectamos la derivada usual de $\mathbb{R}^3$ sobre el plano tangente, obtenemos la derivada covariante. 

\begin{definicion}{Derivada Covariante}
Sea $V \in \Xfrak(\alpha)$ un campo tangente y diferenciable. Se define la derivada covariante (o intrínseca) de $V$ como la parte tangente de la derivada usual $V'(t)$:
\[
\frac{DV}{dt}(t) := V'(t)^\top = V'(t) - \inner{V'(t)}{N(t)}N(t) \in \Xfrak(\alpha).
\]
\end{definicion}

\begin{proposicion}{Carácter intrínseco}
$\frac{DV}{dt}$ es un concepto intrínseco; solo depende de la primera forma fundamental de $S$.
\end{proposicion}

\begin{proof}
Sea $\mathbf{X}: U \subset \mathbb{R}^2 \longrightarrow S$ una parametrización de la superficie y sea $\alpha: I \to S$ una curva tal que $\alpha(I) \subset \mathbf{X}(U)$. Podemos expresar la curva en coordenadas como $\alpha(t) = \mathbf{X}(u(t), v(t))$.

Sea $V \in \mathfrak{X}(\alpha)$ un campo tangente a lo largo de $\alpha$. Podemos expresarlo en la base del plano tangente $\{ \mathbf{X}_u, \mathbf{X}_v \}$ como:
\[
V(t) = a(t)\mathbf{X}_u(u(t), v(t)) + b(t)\mathbf{X}_v(u(t), v(t)).
\]
Para calcular la derivada covariante, primero calculamos la derivada usual $V'(t)$ usando la regla de la cadena y la regla del producto:
\[
V'(t) = a'\mathbf{X}_u + a(\mathbf{X}_{uu}u' + \mathbf{X}_{uv}v') + b'\mathbf{X}_v + b(\mathbf{X}_{vu}u' + \mathbf{X}_{vv}v').
\]
Ahora utilizamos las \textbf{Fórmulas de Gauss} para descomponer las segundas derivadas de la parametrización en sus partes tangencial y normal. Recordamos que:
\begin{align*}
\mathbf{X}_{uu} &= \Gamma_{11}^1 \mathbf{X}_u + \Gamma_{11}^2 \mathbf{X}_v + eN, \\
\mathbf{X}_{uv} &= \Gamma_{12}^1 \mathbf{X}_u + \Gamma_{12}^2 \mathbf{X}_v + fN, \\
\mathbf{X}_{vv} &= \Gamma_{22}^1 \mathbf{X}_u + \Gamma_{22}^2 \mathbf{X}_v + gN.
\end{align*}
Sustituyendo estas expresiones en la ecuación de $V'(t)$:
\begin{align*}
V'(t) &= a'\mathbf{X}_u + a \left[ u'(\Gamma_{11}^1 \mathbf{X}_u + \Gamma_{11}^2 \mathbf{X}_v + eN) + v'(\Gamma_{12}^1 \mathbf{X}_u + \Gamma_{12}^2 \mathbf{X}_v + fN) \right] \\
&\quad + b'\mathbf{X}_v + b \left[ u'(\Gamma_{12}^1 \mathbf{X}_u + \Gamma_{12}^2 \mathbf{X}_v + fN) + v'(\Gamma_{22}^1 \mathbf{X}_u + \Gamma_{22}^2 \mathbf{X}_v + gN) \right].
\end{align*}
Agrupando los términos tangenciales (coeficientes de $\mathbf{X}_u$ y $\mathbf{X}_v$) y los normales (coeficientes de $N$), obtenemos:
\begin{align*}
V'(t) &= \left[ a' + a u' \Gamma_{11}^1 + a v' \Gamma_{12}^1 + b u' \Gamma_{12}^1 + b v' \Gamma_{22}^1 \right] \mathbf{X}_u \\
&\quad + \left[ b' + a u' \Gamma_{11}^2 + a v' \Gamma_{12}^2 + b u' \Gamma_{12}^2 + b v' \Gamma_{22}^2 \right] \mathbf{X}_v \\
&\quad + \left[ a u' e + a v' f + b u' f + b v' g \right] N.
\end{align*}
Por definición, la derivada covariante $\frac{DV}{dt}$ es la proyección ortogonal de $V'(t)$ sobre el plano tangente $T_{\alpha(t)}S$. Por tanto, descartamos la componente en $N$ y nos queda:
\[
\frac{DV}{dt} = (\dots)\mathbf{X}_u + (\dots)\mathbf{X}_v.
\]
Observamos que esta expresión depende exclusivamente de $a, b, u, v$, sus primeras derivadas, y los \textbf{Símbolos de Christoffel} $\Gamma_{ij}^k$.

Como sabemos que los símbolos de Christoffel dependen únicamente de los coeficientes de la primera forma fundamental ($E, F, G$) y sus derivadas, concluimos que $\frac{DV}{dt}$ es un concepto intrínseco.
\end{proof}

\begin{proposicion}{Propiedades de la derivada covariante}
Sean $V, W \in \Xfrak(\alpha)$ y sea $f \in C^\infty(I, \mathbb{R})$. Entonces:
\begin{enumerate}
    \item $\displaystyle \frac{D}{dt}(V+W) = \frac{DV}{dt} + \frac{DW}{dt}$.
    
    \item $\displaystyle \frac{D}{dt}(fV) = f'V + f\frac{DV}{dt}$.
    
    \item $\displaystyle \inner{V}{W}' = \inner{\frac{DV}{dt}}{W} + \inner{V}{\frac{DW}{dt}}$.
\end{enumerate}
\end{proposicion}
\begin{proof}[Demostración]

 \textbf{ii Derivada del producto por una función:} \\
Por definición, $\frac{D}{dt}(fV)$ es la componente tangencial de la derivada usual $(fV)'$. Aplicando la regla de la cadena usual:
\[
\frac{D}{dt}(fV) = \left[ (fV)' \right]^\top = \left[ f'V + fV' \right]^\top.
\]
Usando la linealidad de la proyección tangencial:
\[
= (f'V)^\top + (fV')^\top.
\]
Como $V$ es un campo tangente, el vector $f'V$ es tangente a la superficie, por lo que su proyección es él mismo: $(f'V)^\top = f'V$. Por otro lado, $(fV')^\top = f(V')^\top = f\frac{DV}{dt}$.
Concluimos que:
\[
\frac{D}{dt}(fV) = f'V + f\frac{DV}{dt}.
\]

\textbf{iii Derivada del producto escalar:} \\
Calculamos la derivada del producto escalar usual en $\mathbb{R}^3$:
\[
\inner{V}{W}' = \inner{V'}{W} + \inner{V}{W'}.
\]
Para analizar el primer término, descomponemos $V'$ en su parte tangencial (la derivada covariante) y su parte normal:
\[
V' = \frac{DV}{dt} + (V')^\perp.
\]
Sustituyendo esto en el producto escalar:
\[
\inner{V'}{W} = \inner{\frac{DV}{dt} + (V')^\perp}{W} = \inner{\frac{DV}{dt}}{W} + \inner{(V')^\perp}{W}.
\]
Dado que $W$ es un campo tangente y $(V')^\perp$ es normal a la superficie (paralelo a $N$), son ortogonales, por lo que $\inner{(V')^\perp}{W} = 0$. Así obtenemos:
\[
\inner{V'}{W} = \inner{\frac{DV}{dt}}{W}.
\]
Aplicando un razonamiento análogo para el segundo término ($\inner{V}{W'} = \inner{V}{\frac{DW}{dt}}$), llegamos al resultado final:
\[
\inner{V}{W}' = \inner{\frac{DV}{dt}}{W} + \inner{V}{\frac{DW}{dt}}.
\]
\end{proof}

% -------------------------------------------------------------------------
\section{Campos paralelos y Transporte Paralelo}
\subsection{Campos paralelos}
\begin{definicion}{Campo paralelo}
Se dice que un campo de vectores $V \in \Xfrak(\alpha)$ es \textbf{paralelo} a lo largo de $\alpha$ si su derivada covariante es nula:
\[
\frac{DV}{dt} = 0.
\]
\end{definicion}

\begin{ejemplo}{El plano y la esfera}
\begin{itemize}
    \item En un plano $\Pi$, como el normal $N$ es constante, un campo es paralelo si y solo si $V$ es constante en el sentido usual (vectores paralelos euclídeos).
    \item En la esfera, a lo largo del ecuador $\alpha(t)=(\cos t, \sin t, 0)$, el campo $V_0(t)=(0,0,1)$ es paralelo. Además, el campo velocidad de cualquier circunferencia máxima es paralelo.
\end{itemize}
\end{ejemplo}


\begin{proposicion}{Propiedades de campos paralelos}
Sean $V, W \in \Xfrak(\alpha)$ campos paralelos.
\begin{enumerate}
    \item Si $a, b \in \mathbb{R}$, entonces $aV + bW$ es un campo paralelo (el espacio de campos paralelos es un espacio vectorial).
    \item El producto escalar $\inner{V}{W}$ es constante. En particular, la norma $|V|$ y el ángulo entre $V$ y $W$ son constantes a lo largo de la curva.
\end{enumerate}
\end{proposicion}

\begin{proof}
\mbox{} \\ % <--- Esto fuerza el salto de línea
\textbf{I.}
$\frac{D}{dt}(aV+bW)=[aV'+bW']^\top=a\frac{DV}{dt}+b\frac{DW}{dt}=0$\\
\\
\textbf{II.}
 $\inner{V}{W}'=\inner{\frac{DV}{dt}}{W} + \inner{V}{\frac{DW}{dt}} = 0 \implies \inner{V}{W} = c$

    
\end{proof}

\begin{teorema}{Existencia y unicidad de campos paralelos}
Sea $\alpha: I \longrightarrow S$ una curva diferenciable y sea $V_0 \in T_{\alpha(t_0)}S$ para cierto $t_0 \in I$. Entonces, existe un \textbf{único} campo paralelo $V \in \Xfrak(\alpha)$ tal que $V(t_0) = V_0$.
\end{teorema}
\begin{proof}
Consideremos una parametrización $\mathbf{X}(u,v)$ alrededor de la traza de la curva. Fijamos el vector inicial $V_0 \in T_{\alpha(t_0)}S$.

Queremos encontrar funciones escalares $a(t), b(t)$ tales que el campo $V(t) = a(t)\mathbf{X}_u + b(t)\mathbf{X}_v$ verifique:
\begin{enumerate}
    \item Condición inicial: $V(t_0) = V_0$, lo cual determina unívocamente $a(t_0)$ y $b(t_0)$.
    \item Condición de paralelismo: $\frac{DV}{dt} = 0$.
\end{enumerate}

Desarrollando la derivada covariante (utilizando la fórmula de la derivada de un campo en coordenadas que vimos anteriormente), la condición $\frac{DV}{dt} = 0$ implica que las componentes tangenciales deben anularse. Como $\{\mathbf{X}_u, \mathbf{X}_v\}$ es una base, sus coeficientes deben ser cero.

Esto nos lleva al siguiente \textbf{sistema de ecuaciones diferenciales ordinarias (EDO)} lineales de primer orden para las incógnitas $a(t)$ y $b(t)$:

\[
\begin{cases}
a' + a(u'\Gamma_{11}^1 + v'\Gamma_{12}^1) + b(u'\Gamma_{12}^1 + v'\Gamma_{22}^1) = 0, \\[0.5em]
b' + a(u'\Gamma_{11}^2 + v'\Gamma_{12}^2) + b(u'\Gamma_{12}^2 + v'\Gamma_{22}^2) = 0.
\end{cases}
\]

Este es un sistema lineal de la forma $Y'(t) = A(t)Y(t)$. Por el \textbf{Teorema de Existencia y Unicidad de soluciones para EDOs} (Picard-Lindelöf), dado que los coeficientes (que dependen de los símbolos de Christoffel y la curva) son diferenciables, existe una \textbf{única solución} $(a(t), b(t))$ definida en todo el intervalo $I$ que satisface las condiciones iniciales dadas por $V_0$.

Por tanto, existe un único campo paralelo $V$ a lo largo de $\alpha$.
\end{proof}


\subsection{El transporte paralelo}

\begin{definicion}{Transporte paralelo}
Dados $t_0, t_1 \in I$ con $\alpha(t_0)=p$ y $\alpha(t_1)=q$. Para cada $V_0 \in T_pS$, definimos el transporte paralelo de $V_0$ a lo largo de $\alpha$ hasta $q$ como la imagen por la aplicación $P_{t_0}^{t_1}(\alpha): T_pS \longrightarrow T_qS$ 
\[
P_{t_0}^{t_1}(\alpha)(V_0) = V(t_1) \in T_qS,
\]
donde $V$ es el único campo paralelo a lo largo de $\alpha$ con $V(t_0)=V_0$.
\end{definicion}

\begin{proposicion}{Isometría}
La aplicación $P_{t_0}^{t_1}(\alpha): T_pS \longrightarrow T_qS$ es una \textbf{isometría lineal}.
\end{proposicion}
\begin{proof}
Sean $v_0, w_0 \in T_{\alpha(t_0)}S$. Queremos probar dos propiedades: linealidad y conservación del producto escalar.

\textbf{1. Linealidad:}
Queremos demostrar que $P_{t_0}^{t_1}(\alpha)(v_0 + w_0) = P_{t_0}^{t_1}(\alpha)(v_0) + P_{t_0}^{t_1}(\alpha)(w_0)$.

Por el Teorema de Existencia y Unicidad de campos paralelos:
\begin{itemize}
    \item Existe un único campo $V \in \mathfrak{X}(\alpha)$ paralelo tal que $V(t_0) = v_0$.
    \item Existe un único campo $W \in \mathfrak{X}(\alpha)$ paralelo tal que $W(t_0) = w_0$.
    \item Existe un único campo $U \in \mathfrak{X}(\alpha)$ paralelo tal que $U(t_0) = v_0 + w_0$.
\end{itemize}
Por definición de transporte paralelo, tenemos que:
\[ P_{t_0}^{t_1}(\alpha)(v_0) = V(t_1), \quad P_{t_0}^{t_1}(\alpha)(w_0) = W(t_1), \quad P_{t_0}^{t_1}(\alpha)(v_0+w_0) = U(t_1). \]

Consideremos ahora el campo suma $V+W$. Sabemos por las propiedades de los campos paralelos que la suma de dos campos paralelos es también un campo paralelo.
Evaluamos este campo en $t_0$:
\[ (V+W)(t_0) = V(t_0) + W(t_0) = v_0 + w_0. \]
Observamos que el campo $V+W$ es paralelo y cumple la misma condición inicial que $U$. Por la \textbf{unicidad} del teorema de campos paralelos, deducimos que $U = V+W$.

Evaluando en $t_1$:
\[ U(t_1) = V(t_1) + W(t_1) \implies P(v_0 + w_0) = P(v_0) + P(w_0). \]

\textbf{2. Isometría:}
Queremos ver que se conserva el producto escalar:
\[ \inner{P_{t_0}^{t_1}(\alpha)(v_0)}{P_{t_0}^{t_1}(\alpha)(w_0)} \stackrel{?}{=} \inner{v_0}{w_0}. \]
Sustituyendo por los campos definidos anteriormente:
\[ \inner{V(t_1)}{W(t_1)} \stackrel{?}{=} \inner{V(t_0)}{W(t_0)}. \]
Sabemos, por una propiedad anterior, que el producto escalar de dos campos paralelos es constante a lo largo de la curva. Por tanto:
\[ \inner{V(t_1)}{W(t_1)} = \inner{V(t_0)}{W(t_0)}. \]
Esto demuestra que la aplicación conserva el producto escalar y, por tanto, conserva normas y ángulos (es una isometría).
\end{proof}

\begin{proposicion}{Otras propiedades}
    \begin{itemize}
        \item El transporte paralelo a lo largo de una curva $\alpha:I\to S$ no depende de la parametrización de la curva.
        \item Si $S_1,S_2$ son dos superficies regulares que son \textbf{tangentes} a lo largo de una curva, entonces el transporte paralelo a lo largo de esa curva es independiente de la superficie en la que se calcule.
    \end{itemize}
\end{proposicion}


% -------------------------------------------------------------------------
\section{Las Geodésicas}
Cuando trabajamos con superficies buscamos trabajar con curvas que sean "buenas" en el sentido de que se aproximen a las propiedades de las rectas lo mejor posible. Estas son las geodésicas. 

\begin{definicion}{Geodésica}
Una curva $\gamma: I \longrightarrow S$ es una geodésica de $S$ si su campo velocidad $\gamma'$ es paralelo a lo largo de la curva, es decir:
\[
\frac{D\gamma'}{dt} = 0.
\]
\end{definicion}
Con la siguiente proposición vamos a buscar que se aproximen a las propiedades de la recta como hemos comentado. 
\begin{proposicion}{Propiedades de las geodésicas}
Sea $\gamma$ una geodésica no constante:
\begin{enumerate}
    \item $|\gamma'(t)|$ es constante. Luego $\gamma$ es una curva regular. 
    \item Si $\gamma$ no está parametrizada por el arco, al menos está parametrizada proporcionalmente al arco.
    \item Las geodésicas son invariantes por isometrías locales.
    \item Una geodésica puede admitir intersecciones.
    \item Si $h:J\to I$ es un cambio de parámetro, $\beta=\gamma \circ h$ es geodésica si y solo si el cambio es afín: $h(s) = as+b$.
    \item Si $\gamma$ está p.p.a. y $\gamma''\neq 0$, es geodésica si y solo si su vector normal principal $n_\gamma$ es paralelo a la normal de la superficie $N$.
\end{enumerate}
\end{proposicion}

\begin{proof}[Demostración de las propiedades]
\mbox{} 
\begin{enumerate}
    \item \textbf{La rapidez es constante:} \\
    Como $\gamma$ es una geodésica, su campo velocidad $\gamma'$ es paralelo, es decir, $\frac{D\gamma'}{dt} = 0$.
    Sabemos que el transporte paralelo conserva el producto escalar y, por tanto, la norma. Así, $|\gamma'(t)| = |\gamma'(t_0)| = c$ (constante).
    Veamos que $c\neq0$: Si existiera $t_0$ tal que $\gamma'(t_0) = 0$, por la unicidad del campo paralelo (con condición inicial 0), tendríamos $\gamma'(t) \equiv 0$, lo que implicaría que la curva es constante (caso excluido).

    \item \textbf{Parametrización proporcional al arco:} \\
    Supongamos que $\gamma$ no está p.p.a. La longitud de arco se define como $s(t) = \int_{t_0}^t |\gamma'(u)| du$.
    Por el apartado anterior, sabemos que $|\gamma'(u)| = c$. Entonces:
    \[
    s(t) = \int_{0}^t c \, du = ct.
    \]
    Esto muestra que el parámetro $t$ es proporcional a la longitud de arco $s$ (relación lineal).

    \item \textbf{Cambio de parámetro afín:} \\
    Sea $\beta(s) = \gamma(h(s))$. Calculamos su aceleración derivando dos veces:
    \[
    \beta'(s) = \gamma'(h(s)) h'(s) \implies \beta''(s) = \gamma''(h(s)) (h'(s))^2 + \gamma'(h(s)) h''(s).
    \]
    Tomamos la derivada covariante (proyección tangencial $[\cdot]^\top$):
    \[
    \frac{D\beta'}{ds} = (\beta''(s))^\top = \underbrace{\left[ \gamma''(h(s)) \right]^\top}_{=0} (h'(s))^2 + \underbrace{\left[ \gamma'(h(s)) \right]^\top}_{\gamma'} h''(s).
    \]
    (El primer término es nulo porque $\gamma$ es geodésica). Nos queda $\frac{D\beta'}{ds} = \gamma'(h(s)) h''(s)$.
    Para que $\beta$ sea geodésica, necesitamos $\frac{D\beta'}{ds} = 0$. Como $\gamma' \neq 0$, esto equivale a $h''(s) = 0$, lo que implica $h(s) = as + b$.

    \item \textbf{Relación con la normal (Frenet):} \\
    Supongamos $\gamma$ p.p.a. y $\gamma'' \neq 0$. Por las fórmulas de Frenet, $\gamma''(s) = k n_\gamma$.
    Descomponemos la aceleración en componentes intrínsecas:
    \[
    \gamma''(s) = \frac{D\gamma'}{ds} + (\gamma''(s))^\perp.
    \]
    Al ser geodésica, $\frac{D\gamma'}{ds} = 0$, luego $\gamma''(s)$ es puramente normal a la superficie (paralelo a $N$).
    Como $\gamma''$ es paralelo a $n_\gamma$ y a $N$ simultáneamente, concluimos que $n_\gamma(s) = \pm N(s)$.
\end{enumerate}
\end{proof}

\begin{ejemplo}{Geodésicas en superficies elementales}
    \begin{itemize}
        \item Las geodésicas del plano son las rectas. 
        \item Las geodésicas de la esfera son las circunferencias máximas. 
        \item Las geodésicas del cilindro son las rectas, circunferencias y hélices.
    \end{itemize}
\end{ejemplo}

% -------------------------------------------------------------------------
% EJEMPLOS DE CÁLCULO DE GEODÉSICAS
% -------------------------------------------------------------------------
De la definición de geodésica, podemos deducir que $S$ es una geodésica sii $\gamma''+<\gamma',N'>N=0$. A partir de esta expresión, vamos a calcular las geodésicas en los ejemplos. 
\begin{proof} [Cálculo de geodésicas en el Plano]
Sea el plano $S = \{ p \in \mathbb{R}^3 : \inner{p}{\vec{a}} = c \}$ con vector normal constante $N(p) = \vec{a}$ (unitario).
Sea $\gamma: I \to S$ una geodésica. La condición de geodésica es que su aceleración $\gamma''$ sea normal a la superficie:
\[
\gamma'' = \frac{D\gamma'}{dt} + (\gamma'')^\perp = 0 + \inner{\gamma''}{N}N.
\]
\textbf{Truco:} Sabemos que $\inner{\gamma'}{N} = 0$ (la velocidad es tangente). Derivamos esta expresión:
\[
\inner{\gamma''}{N} + \inner{\gamma'}{N'} = 0.
\]
Como el plano tiene normal constante, $N' = 0$. Por tanto, $\inner{\gamma''}{N} = 0$.
Sustituyendo arriba:
\[
\gamma'' = 0 \cdot N = 0 \implies \gamma(t) = p + t\vec{v}.
\]
\textbf{Conclusión:} Las geodésicas del plano son las \textbf{rectas}.
\end{proof}

\begin{proof}[Cálculo de geodésicas en la Esfera $\mathbb{S}^2(r)$]
Sea $p \in \mathbb{S}^2(r)$ y $\vec{v} \in T_p\mathbb{S}^2(r)$ con $|\vec{v}|=c$.
Sabemos que $|\gamma'(t)| = c$ constante por estar considerando una geodésica (propiedades de las geodésicas). El vector normal es $N(t) = \frac{1}{r}\gamma(t)$.
La condición de geodésica es:
\[
\frac{D\gamma'}{dt} = \gamma'' - \inner{\gamma''}{N}N = 0 \implies \gamma'' = \inner{\gamma''}{N}N.
\]
Para calcular el término $\inner{\gamma''}{N}$, derivamos la condición de tangencia $\inner{\gamma'}{N}=0$:
\[
\inner{\gamma''}{N} + \inner{\gamma'}{N'} = 0 \implies \inner{\gamma''}{N} = - \inner{\gamma'}{N'}.
\]
Como $N(t) = \frac{1}{r}\gamma(t)$ (el vector posición), entonces $N'(t) = \frac{1}{r}\gamma'(t)$. Así:
\[
\inner{\gamma''}{N} = - \inner{\gamma'}{\frac{1}{r}\gamma'} = -\frac{1}{r}|\gamma'|^2 = -\frac{c^2}{r}.
\]
La ecuación diferencial queda:
\[
\gamma'' = -\frac{c^2}{r} N = -\frac{c^2}{r} \left( \frac{1}{r}\gamma \right) \implies \gamma'' + \frac{c^2}{r^2}\gamma = 0.
\]
Esta es la ecuación de un oscilador armónico ($\gamma'' + k^2\gamma = 0$). \\
\textit{Ver Anexo de ecuaciones diferenciales}
\\
Su solución general con condiciones iniciales $\gamma(0)=p, \gamma'(0)=\vec{v}$ es:
\[
\gamma(t) = \cos\left( \frac{c}{r}t \right)p + \frac{r}{c}\sin\left( \frac{c}{r}t \right)\vec{v}.
\]
\textbf{Conclusión:} Como la curva está contenida en el plano generado por $p$ y $\vec{v}$ que pasa por el origen, las geodésicas son \textbf{circunferencias máximas} (círculos grandes).
\end{proof}

\begin{proof}[Cálculo de geodésicas en el Cilindro]
Sea el cilindro $C$ de radio $r$ con eje $z$. Un punto es $\gamma(t) = (\gamma_1, \gamma_2, \gamma_3)$.
El normal es $N(\gamma(t)) = (\frac{\gamma_1}{r}, \frac{\gamma_2}{r}, 0)$, por lo que $N' = (\frac{\gamma_1'}{r}, \frac{\gamma_2'}{r}, 0)$.
La condición de geodésica $\gamma'' \perp S$ implica $\gamma'' = \lambda N$, es decir:
\[
(\gamma_1'', \gamma_2'', \gamma_3'') = \lambda \left( \frac{\gamma_1}{r}, \frac{\gamma_2}{r}, 0 \right).
\]
De la tercera componente obtenemos inmediatamente:
\[
\gamma_3'' = 0 \implies \gamma_3(t) = p_3 + t v_3 \quad \text{(Movimiento rectilíneo uniforme en el eje z)}.
\]
Para las otras componentes, usamos que la rapidez $|\gamma'|=c$ es constante:
\[
c^2 = |\gamma'|^2 = (\gamma_1')^2 + (\gamma_2')^2 + (\gamma_3')^2.
\]
Como $\gamma_3' = v_3$ (constante), tenemos que la velocidad horizontal al cuadrado es $(\gamma_1')^2 + (\gamma_2')^2 = c^2 - v_3^2$.
Esto reduce el problema a encontrar geodésicas en un círculo de radio $r$ con velocidad constante $\sqrt{c^2-v_3^2}$. La solución es un movimiento circular uniforme:
\[
\gamma_{1,2}(t) = \left( r \cos(\omega t), r \sin(\omega t) \right), \quad \text{con } \omega = \frac{\sqrt{c^2-v_3^2}}{r}.
\]
Combinando todo, la solución general es:
\[
\gamma(t) = \left( r \cos(At+B), r \sin(At+B), Ct+D \right).
\]
\textbf{Conclusión:} Las geodésicas del cilindro son \textbf{hélices} (si $v_3 \neq 0$ y rotación $\neq 0$), \textbf{circunferencias} (si $v_3=0$) o \textbf{rectas verticales} (si no hay rotación).
\end{proof}

\subsection{Ecuaciones diferenciales}
\noindent En su momento, cuando vimos si un campo era paralelo, $\forall V \in \mathfrak{X}(\alpha)$, con $\alpha(t) = X(u(t), v(t))$,
\[
V(t) = a(t)X_u(u(t), v(t)) + b(t)X_v(u(t), v(t)) \text{ es paralelo } \Leftrightarrow
\]
\[
\begin{cases}
a' + a(u'\Gamma_{11}^1 + v'\Gamma_{12}^1) + b(u'\Gamma_{12}^1 + v'\Gamma_{22}^1) = 0, \\[0.5em]
b' + a(u'\Gamma_{11}^2 + v'\Gamma_{12}^2) + b(u'\Gamma_{12}^2 + v'\Gamma_{22}^2) = 0.
\end{cases}
\]

\noindent Entonces, $\gamma: I \to S$ es geodésica $\Leftrightarrow \gamma'$ es paralelo.
\[
\gamma(t) = X(u(t), v(t)), \quad \gamma'(t) \in \mathfrak{X}(\gamma) \rightarrow \text{coordenadas } (\underbrace{u'(t)}_{a}, \underbrace{v'(t)}_{b})
\]
 Sustituyo en la ecuación anterior y me queda lo siguiente:

Sea $(U, \mathbf{X})$ una parametrización de $S$ y $\gamma(t) = \mathbf{X}(u(t), v(t))$. La condición de geodésica se traduce en el siguiente sistema de ecuaciones diferenciales de segundo orden (usando los símbolos de Christoffel $\Gamma_{ij}^k$):

\[
\begin{cases}
u'' + (u')^2\Gamma_{11}^1 + 2u'v'\Gamma_{12}^1 + (v')^2\Gamma_{22}^1 = 0, \\
v'' + (u')^2\Gamma_{11}^2 + 2u'v'\Gamma_{12}^2 + (v')^2\Gamma_{22}^2 = 0.
\end{cases}
\]

\begin{teorema}{Existencia y unicidad de geodésicas maximales}
Sea $S$ una superficie regular, $p \in S$ y $v \in T_pS$, existe una única geodésica  $\gamma_v: I_v \longrightarrow S$ (con $I_v$ abierto conteniendo al 0) tal que:
\begin{itemize}
    \item $\gamma_v(0) = p \quad \text{y} \quad \gamma_v'(0) = v.$
    \item Si $\alpha:J\to S$ es otra geodésica con $\alpha(0)= p $ y $\alpha'(0)= v$, entonces $J \subset I_v$ y $\alpha = \gamma_v|_J$
\end{itemize}

\end{teorema}
\begin{proof}[Idea de la demostración]
Sea $p \in S, \vec{v} \in T_pS$. Definimos el conjunto:
\[
\mathcal{J}_{p,\vec{v}} = \{ \gamma : I \to S \mid 0 \in I, \gamma(0)=p, \gamma'(0)=\vec{v} \}.
\]
Veamos que $\mathcal{J}_{p,\vec{v}} \neq \emptyset$ (vamos a usar las EDOs de las geodésicas).
Cogemos una parametrización de la superficie que contenga a $p$.
Sea $(U, X)$ parametrización con $p \in X(U) \implies p = X(u_0, v_0)$.
Expresamos el vector en la base coordenada:
\[
\vec{v} = a X_u(u_0, v_0) + b X_v(u_0, v_0).
\]
Por el \textbf{Teorema de Existencia y Unicidad de sistemas de EDOs}, existe una \textbf{única} solución $(u(t), v(t))$ al sistema de ecuaciones de las geodésicas tal que:
\[
\begin{cases}
u(0) = u_0, & v(0) = v_0 \\
u'(0) = a, & v'(0) = b
\end{cases}
\]
Defino $\gamma(t) = X(u(t), v(t))$. Esta curva cumple las condiciones iniciales y el sistema, por tanto es geodésica (y es única).
\[
\implies \mathcal{J}_{p,\vec{v}} \neq \emptyset.
\]

\vspace{0.5cm}

\begin{tcolorbox}[colback=red!5!white, colframe=red!75!black, title=¡OJO!]
Esto no acaba la demostración. Esta geodésica la he encontrado en $V = X(U)$ (en la parametrización).
¿Qué ocurre en la superficie? Esto es un resultado \textbf{local}, no global como lo está diciendo en el enunciado.
\end{tcolorbox}

\textbf{Técnica que usaremos mucho (Unicidad):}

Supongamos $\gamma_1 : I_1 \to S$ y $\gamma_2 : I_2 \to S$ con $\gamma_1(0) = \gamma_2(0) = p$ y $\gamma_1'(0) = \gamma_2'(0) = \vec{v}$.
Sea el conjunto:
\[
A = \{ t \in \underbrace{I_1 \cap I_2}_{\text{conexo}} \mid \gamma_1(t) = \gamma_2(t) \text{ y } \gamma_1'(t) = \gamma_2'(t) \}.
\]
Sabemos que $A \neq \emptyset$ porque $0 \in A$.

Vamos a ver que $A = I_1 \cap I_2$.
(No lo vamos a hacer en detalle). Demostramos que $A$ es \textbf{cerrado} y \textbf{abierto} en el conexo $I_1 \cap I_2$.
\[
\implies A = I_1 \cap I_2.
\]
Así, en el trozo común tengo las mismas geodésicas.

\textit{Para extenderlo a maximales creo que se va cogiendo intervalos y se extiende hasta donde se quiera.}
\end{proof}

\begin{definicion}{Geodésicamente completa}
Una superficie $S$ es \textbf{geodésicamente completa en un punto} $p\in S$ si $I_v = \mathbb{R}$ para todo $v\in T_pS$. Se dice además que $S$ es \textbf{geodésicamente completa} cuando lo es en todos sus puntos.
\end{definicion}
Recordemos que en una superficie, para una curva $\alpha$ también podíamos definir una base (Triedro de Darboux): 
$\alpha:I\to S$ p.p.a, $$\{\alpha', J\alpha', N\}$$ donde $N$ es el normal a la superficie y $J(\gamma)$ la rotación de $90º$. 

\begin{definicion}{Curvatura geodésica}
Sea $S$ una superficie regular. Para una curva \textbf{p.p.a}. $\alpha: I\to S$, la \textbf{curvatura geodésica} se define como:
\[
k_g(s) = \inner{\alpha''(s)}{J\alpha'(s)} = \inner{\alpha''(s)}{N(s) \wedge \alpha'(s)}.
\]
Una \textbf{pregeodésica} es una curva  que se puede reparametrizar para ser una geodésica.

\end{definicion}

En ejercicios demostraremos de dónde sale la siguiente definición: 
\begin{definicion}{Curvatura geodésica para $\alpha$ \textbf{no p.p.a}}
Sea $S$ una superficie regular.  $\alpha: I\to S$, la \textbf{curvatura geodésica} se define como:
\[
k_g(s)  = \frac{\inner{\alpha''(s)}{N(s) \wedge \alpha'(s)}}{|\alpha'|^3}.
\]
\end{definicion}

\begin{proposicion}{Caracterización Geodésica}
    Sea $\alpha: I \to S$ una curva \textbf{p.p.a.} en una superficie regular $S$. Entonces, $\alpha$ es ua geodésica si, y solo si, $k_g(s)=0 $ para todo $s\in I.$
\end{proposicion}
\begin{proof}[Demostración]
    Sabemos que la curvatura geodésica es $k_g = \langle \alpha'', N \wedge \alpha' \rangle$.
    Entonces:
    \[
    k_g = 0 \iff \langle \alpha'', N \wedge \alpha' \rangle = 0 \iff \alpha'' \perp (N \wedge \alpha') = \alpha'' \perp J\alpha'.
    \]
    Además, como $\alpha$ está parametrizada por el arco (p.p.a.), sabemos que $\langle \alpha', \alpha' \rangle = 1$. Derivando esta expresión:
    \[
    \langle \alpha'', \alpha' \rangle = 0 \iff \alpha'' \perp \alpha'.
    \]
    
    Si el vector aceleración $\alpha''$ es perpendicular a $\alpha'$ y también a $J\alpha'$ (que forman una base del plano tangente), entonces $\alpha''$ es perpendicular a todo el plano tangente:
    \[
    \implies \alpha'' \perp T_{\alpha(t)}S \implies \alpha'' \parallel N(\alpha(t)).
    \]
    Esto es equivalente a decir que $\alpha$ es una \textbf{geodésica} (por la propiedad de que su aceleración es normal a la superficie).
\end{proof}
% -------------------------------------------------------------------------
\section{La Aplicación Exponencial}

La aplicación exponencial nos permite mapear el espacio tangente sobre la superficie siguiendo geodésicas.

\begin{definicion}{Aplicación Exponencial}
Sea $p \in S$ con $S$ una superficie regular y $D_p = \{ v \in T_pS : 1 \in I_v \}$. Se define la aplicación exponencial $\exp_p: D_p \subset T_pS \longrightarrow S$ como:
\[
\exp_p(v) = \gamma_v(1), \quad \text{con } \exp_p(0) = p.
\]
\end{definicion}
\begin{observacion}{Dominio de la exponencial en superficies completas}
Si una superficie regular $S$ es geodésicamente completa en $p \in S$, entonces el dominio de la aplicación exponencial $D_p \equiv T_pS$.
\end{observacion}
\begin{observacion}{Interpretación métrica de la exponencial}
Otra forma de entender la aplicación exponencial es a través de la longitud de las curvas. Sea $\gamma_{\vec{v}}$ la geodésica radial definida por $\vec{v}$, es decir, con $\gamma_{\vec{v}}(0)=p$ y $\gamma_{\vec{v}}'(0)=\vec{v}$.

Sabemos que el punto imagen es $\exp_p(\vec{v}) = \gamma_{\vec{v}}(1)$. Calculemos la longitud de este segmento de geodésica desde $t=0$ hasta $t=1$:
\[
L_p^{\exp_p(\vec{v})}(\gamma_{\vec{v}}) = L_0^1(\gamma_{\vec{v}}) = \int_0^1 |\gamma_{\vec{v}}'(t)| \, dt.
\]
Recordemos una propiedad fundamental de las geodésicas: tienen **rapidez constante**. Por tanto, $|\gamma_{\vec{v}}'(t)| = |\gamma_{\vec{v}}'(0)| = |\vec{v}|$ para todo $t$. La integral se simplifica inmediatamente:
\[
\int_0^1 |\gamma_{\vec{v}}'(t)| \, dt = \int_0^1 |\vec{v}| \, dt = |\vec{v}| \cdot (1-0) = |\vec{v}|.
\]
\textbf{Conclusión:} La norma del vector $|\vec{v}|$ en el espacio tangente $T_pS$ representa exactamente la \textbf{distancia geodésica} sobre la superficie desde el punto $p$ hasta el punto $\exp_p(\vec{v})$. Es decir, la exponencial enrolla el plano tangente sobre la superficie preservando las longitudes radiales.
\end{observacion}
\begin{lema}{de homogeneidad de las geodésicas}
Sean $S$ una superficie regular, $p \in S$ y $\mathbf{v} \in T_pS$. Sea $\gamma_{\mathbf{v}} : I_{\mathbf{v}} \longrightarrow S$ la geodésica maximal con $\gamma_{\mathbf{v}}(0) = p$ y $\gamma_{\mathbf{v}}'(0) = \mathbf{v}$, y sea $\lambda > 0$. Si $(-\varepsilon_1, \varepsilon_2) \subset I_{\mathbf{v}}$, entonces $(-\varepsilon_1/\lambda, \varepsilon_2/\lambda) \subset I_{\lambda\mathbf{v}}$, y además $\gamma_{\lambda\mathbf{v}}(t) = \gamma_{\mathbf{v}}(\lambda t)$ para todo $t \in (-\varepsilon_1/\lambda, \varepsilon_2/\lambda)$, donde $\gamma_{\lambda\mathbf{v}} : I_{\lambda\mathbf{v}} \longrightarrow S$ es la única geodésica maximal con condiciones iniciales $\gamma_{\lambda\mathbf{v}}(0) = p$ y $\gamma_{\lambda\mathbf{v}}'(0) = \lambda\mathbf{v}$.
\end{lema}
Nosotros realmente usaremos: 
\begin{tcolorbox}[colback=yellow!10!white, colframe=yellow!50!black, title=Lema de Homogeneidad]
Si $\gamma_v$ es la geodésica con velocidad inicial $v$, entonces la geodésica con velocidad inicial $\lambda v$ es simplemente una reparametrización de la anterior:
\[
\gamma_{\lambda v}(t) = \gamma_v(\lambda t).
\]
Esto implica que $\exp_p(tv) = \gamma_v(t)$.
\end{tcolorbox}
\begin{proof}[Demostración del Lema de Homogeneidad]
Sea $\gamma_v : I_v \longrightarrow S$ la geodésica maximal tal que $\gamma_v(0) = p$ y $\gamma_v'(0) = \mathbf{v}$.

Definimos la curva auxiliar $\alpha : I_\alpha \longrightarrow S$, donde el intervalo es $I_\alpha = \{ t \in \mathbb{R} \mid \lambda t \in I_v \}$, dada por:
\[
\alpha(t) = \gamma_v(\lambda t).
\]
Comprobamos sus condiciones iniciales:
\begin{itemize}
    \item $\alpha(0) = \gamma_v(0) = p$.
    \item $\alpha'(t) = \lambda \gamma_v'(\lambda t) \implies \alpha'(0) = \lambda \gamma_v'(0) = \lambda \mathbf{v}$.
\end{itemize}

Calculamos la derivada covariante de la velocidad para ver si es geodésica:
\[
\frac{D\alpha'}{dt} = \lambda^2 \frac{D\gamma_v'(\lambda t)}{dt} = 0 \quad \text{(pues } \gamma_v \text{ es geodésica)}.
\]

Por tanto, $\alpha$ es una geodésica que parte de $p$ con velocidad $\lambda \mathbf{v}$.
Por el \textbf{Teorema de Existencia y Unicidad de geodésicas maximales}, $\alpha$ debe coincidir con $\gamma_{\lambda \mathbf{v}}$ en su dominio común.
Como $\gamma_{\lambda \mathbf{v}}$ es maximal, tenemos que $I_\alpha \subset I_{\lambda \mathbf{v}}$ y:
\[
\gamma_{\lambda \mathbf{v}}(t) = \alpha(t) \text{ en } I_\alpha \implies \gamma_{\lambda \mathbf{v}}(t) = \gamma_v(\lambda t).
\]

\textbf{Veamos lo de los intervalos:}
¿Se cumple que $\left( -\frac{\varepsilon_1}{\lambda}, \frac{\varepsilon_2}{\lambda} \right) \subset I_{\lambda \mathbf{v}}$?

Vamos a ver que $\left( -\frac{\varepsilon_1}{\lambda}, \frac{\varepsilon_2}{\lambda} \right) \subset I_\alpha$. Como $I_\alpha \subset I_{\lambda \mathbf{v}}$, ya lo tendríamos.

Sea $t \in \left( -\frac{\varepsilon_1}{\lambda}, \frac{\varepsilon_2}{\lambda} \right)$. Como $\lambda > 0$, multiplicando la desigualdad:
\[
-\varepsilon_1 < \lambda t < \varepsilon_2 \implies \lambda t \in (-\varepsilon_1, \varepsilon_2).
\]
Por hipótesis, $(-\varepsilon_1, \varepsilon_2) \subset I_v$, luego $\lambda t \in I_v$.
Por cómo hemos definido $I_\alpha = \{ t \mid \lambda t \in I_v \}$, concluimos que $t \in I_\alpha$.

Por tanto, el intervalo escalado está en el dominio y se cumple la fórmula de homogeneidad.
\end{proof}
\begin{observacion}{Observación del Teorema Anterior}
    El mismo teorema vale para un $\lambda<0$ si el intervalo $(-\epsilon_1,\epsilon_2)$ es simétrico (o sea, $\epsilon_1 =\epsilon_2$). 
\end{observacion}
\begin{proof}
    Basta aplicar la demostración anterior. La primera parte es idéntica y la parte de los intervalos es igual. Simplemente, al multiplicar por el $\lambda$, se intercambian las desigualdades, pero al ser el intervalo simétrico, también nos vale. 
\end{proof}

\begin{teorema}{Propiedades de la aplicación exponencial}
Sean $S$ una superficie regular y $p \in S$. Entonces:
\begin{enumerate}[label=\roman*)]
    \item Para cualesquiera $\mathbf{v} \in T_pS$ y $t \in I_{\mathbf{v}}$ con $t > 0$, se tiene que $t\mathbf{v} \in D_p$. Además, $\exp_p(t\mathbf{v}) = \gamma_{\mathbf{v}}(t)$.
    \begin{enumerate}[label=\roman{enumi}.\alph*)]
        \item $D_p$ es estrellado respecto a $\mathbf{0}$.
        \item Para todo $\mathbf{v} \in T_pS$, existe $\lambda > 0$ tal que $\lambda\mathbf{v} \in D_p$, es decir, \textit{todas las direcciones están en $D_p$}.
    \end{enumerate}
    \item El dominio de la exponencial $D_p$ es un abierto, y $\exp_p : D_p \longrightarrow S$ es una aplicación diferenciable.
    \item La aplicación $\exp_p$ es un difeomorfismo local en $\mathbf{0}$.
\end{enumerate}
\end{teorema}
\begin{proof}[Demostración]
\textbf{(i)} Sea $t \in I_v$. Usaremos el lema de homogeneidad con $\lambda = t \implies \frac{t}{t} = 1 \in I_{tv} \implies t\vec{v} \in D_p$ por la propia definición del conjunto. 
La otra parte se tiene también por el Lema de Homogeneidad de las geodésicas:
\[
\boxed{\exp_p(t\vec{v}) = \gamma_{tv}(1) \underset{\text{lema}}{=} \gamma_v(t)}
\]

\textbf{(i.a)}
Se tiene como consecuencia de \textbf{i}:
Sea $\vec{v}\in D_p$. ¿Se cumple que $[\vec{0},\vec{v}]\subset D_p$?. 
Sí porque $[\vec{0},\vec{v}]=\{\lambda\vec{v} : \lambda\in[0,1]\}.$ Como $I_v$ es un intervalo y $1\in I_v$, entonces $[0,1]\subset I_v$.\\
$$\lambda\in [0,1]\implies \lambda\in I_v \implies \gamma_v(\lambda) = exp_p(\lambda\vec{v}) \implies \lambda\vec{v}\in D_p$$. 

\textbf{(i.b)}
Como $0\in I_v$, al ser abierto $I_v$, $\exists\lambda>0$ tal que $\lambda \in I_v \implies \lambda\vec{v}\in D_p$. 

\textbf{(ii)} María Ángeles no lo demuestra. 

\textbf{(iii)} Consideramos la diferencial de la exponencial en el origen:
\[
\exp_p : D_p \subset T_pS \longrightarrow S \implies d(\exp_p)_{\vec{0}} : \underbrace{T_{\vec{0}}D_p}_{=T_pS} \longrightarrow \underbrace{T_{\exp_p(\vec{0})}S}_{=T_pS \leftarrow \exp_p(\vec{0})=p}
\]
\textit{Nota: En la diferencial ponemos el propio plano porque $D_p$ es un trozo de plano de $T_pS$, si le calculamos su plano tangente en un punto, obviamente saldrá todo $T_pS$}.

Es decir, $d(\exp_p)_{\vec{0}} : T_pS \longrightarrow T_pS$ viene dada por:
\[
d(\exp_p)_{\vec{0}}(\vec{v}) = \frac{d}{dt}\Big|_{t=0} (\exp_p \circ \alpha)(t)
\]
siendo $\alpha : I \longrightarrow D_p$ una curva con $\alpha(0)=\vec{0}, \alpha'(0)=\vec{v}$.
Como la elección de $\alpha$ puede ser cualquier curva con tal de que verifique las condiciones iniciales y, muy importante, esté en la superficie (en este caso $D_p$), pues cogeremos la más sencilla posible. Como $D_p$ es un plano, podremos entonces tomar la recta $\alpha(t) = \vec{v} \cdot t$.

Tendremos entonces:
\[
d(\exp_p)_{\vec{0}}(\vec{v}) = \frac{d}{dt}\Big|_{t=0} \exp_p(t\vec{v}) = \frac{d}{dt}\Big|_{t=0} \gamma_v(t) = \gamma_v'(0) = \vec{v}
\]
Es decir: $d(\exp_p)_{\vec{0}} \equiv \mathbb{I}_d$ (la identidad) $\implies \exp_p$ es un \textbf{difeomorfismo en el origen} (por el Teorema de la Función Inversa).
\end{proof}

%Ejemplo Aplicación Exponencial en la Esfera
\begin{ejemplo}{La aplicación exponencial en la esfera}
Veamos la aplicación exponencial en la esfera.

    % --- SOLUCIÓN: Usar center en lugar de figure ---
    \begin{center}
        \includegraphics[width=0.4\linewidth]{Imagenes/Exponencial Esfera.png}
    \end{center}
    % ------------------------------------------------

    \begin{center}
    \vspace{0.5cm} 
    \textit{[Esfera $\mathbb{S}^2(r)$ con plano tangente $T_p\mathbb{S}^2(r)$ en el polo norte $p$. Mostrar vectores $\vec{v}$ y su imagen $\exp_p(\vec{v})$ sobre la esfera.]}
    \end{center}

Sabemos (ya la calculamos) que la fórmula es:
\[
\exp_p(\vec{v}) = \gamma_v(1) = \cos\left(\frac{|\vec{v}|}{r}\right)p + \frac{r}{|\vec{v}|}\sin\left(\frac{|\vec{v}|}{r}\right)\vec{v}
\]
Sea p el polo norte de la esfera. Entonces, el \textbf{ecuador} es: $\text{Ecuador} = \{ q \in \mathbb{S}^2(r) : \langle p, q \rangle = 0 \}$.

\textbf{¿Cuándo se tiene $\exp_p(\vec{v}) \in \text{Ecuador}$?}
\[
\langle \exp_p(\vec{v}), p \rangle = 0 \iff \cos\left(\frac{|\vec{v}|}{r}\right)\langle p, p \rangle = 0
\]
(El término del seno se va porque $\langle \vec{v}, p \rangle = 0$, ya que $\vec{v} \in T_pS$).
Como $p \neq 0$, entonces:
\[
\cos\left(\frac{|\vec{v}|}{r}\right) = 0 \iff \frac{|\vec{v}|}{r} = \frac{\pi}{2} \implies |\vec{v}| = \frac{\pi}{2}r.
\]

Es decir, si el vector $\vec{v}$ tiene módulo $|\vec{v}| = \frac{\pi}{2}r$, la exponencial llegará al ecuador. Pero si lo queremos en un \textbf{punto concreto} $p_0$ del ecuador, hay que marcar la dirección de $\vec{v}$, sustituyendo en $\exp_p(\vec{v})$ sabiendo que $|\vec{v}| = \frac{\pi}{2}r$ (lo que implica $\cos(\frac{\pi}{2})=0$):

\[
\exp_p(\vec{v}) = p_0 \iff p_0 = \frac{r}{|\vec{v}|}\sin\left(\frac{\pi}{2}\right)\vec{v} \implies \boxed{\vec{v} = \frac{|\vec{v}|}{r}p_0 = \frac{\pi}{2}p_0}
\]

Entonces la exponencial llegará a un punto $p_0$ del ecuador partiendo de $p$ cogiendo $\vec{v} = \frac{\pi}{2}p_0$.

\textbf{¿Y si queremos llegar a la antípoda?} 

    \begin{center}
        \includegraphics[width=0.4\linewidth]{Imagenes/Antípoda Esfera.png }
    \end{center}
    % ------------------------------------------------
Es decir, qué $\vec{v} \in T_pS$ tomamos para que $\exp_p(\vec{v}) = -p$.
\[
\underbrace{\cos\left(\frac{|\vec{v}|}{r}\right)}_{\text{deberá ser } -1} p + \underbrace{\frac{r}{|\vec{v}|}\sin\left(\frac{|\vec{v}|}{r}\right)}_{\text{deberá ser } 0} \vec{v} = -p \implies \frac{|\vec{v}|}{r} = \pi \implies |\vec{v}| = \pi r
\]

Pero entonces aquí tenemos \textbf{infinitos vectores} (toda una circunferencia $\mathbb{S}^1(\pi r)$ en el plano tangente) que llevan $\exp_p$ a $-p$.

Entonces la exponencial ahí ya no puede ser inyectiva, pero en todo punto de $\mathbb{S}^2(r)$ quitando ese (la antípoda), habrá un entorno para el que la exponencial será un difeomorfismo.

Adelantando la siguiente definición, el conjunto $U=\{\vec v\in T_p\mathbb{S}^2(r): |  v | < \pi r\}$ será un \textbf{entorno normal}.

\end{ejemplo}


% -------------------------------------------------------------------------
\section{Entornos Normales}

Gracias a que la exponencial es un difeomorfismo local en 0, podemos definir coordenadas locales basadas en geodésicas.

\begin{definicion}{Entorno Normal}
Un entorno $V$ \textbf{de $p \in S$} se llama \textbf{entorno normal} de $p$ si $V$ es la imagen por $\exp_p$ de un entorno $U$ de $0 \in T_pS$ tal que:
\begin{enumerate}
    \item $U$ es estrellado respecto a 0.
    \item $\exp_p|_U: U \longrightarrow V$ es un difeomorfismo.
\end{enumerate}
\end{definicion}

\textbf{Nota:} Se dice que un abierto es un \textit{entorno uniformemente normal} si es entorno normal de todos sus puntos. Siempre existen estos entornos.

\begin{teorema}{Existencia y unicidad de geodésicas radiales}
Sea $V$ un entorno normal de $p_0$. Entonces, para todo punto $p \in V$, existe un \textbf{único} segmento de geodésica radial $\gamma_p: [0,1] \longrightarrow V$ que une $p_0$ con $p$ y está totalmente contenido en $V$.
\end{teorema}
\begin{proof}
Sea $V$ un entorno normal de $p_0$ y sea $p \in V$.
Por ser $V$ un entorno normal, $\exists \mathcal{U} \subset D_{p_0}$ abierto, \textbf{estrellado} respecto del origen $\vec{0} \in \mathcal{U}$, tal que $\exp_{p_0}: \mathcal{U} \longrightarrow V$ es un \textbf{difeomorfismo}.

Como $p \in V$ y $\exp_{p_0}: \mathcal{U} \longrightarrow V$ es un difeomorfismo \textcolor{blue}{\footnotesize (biyeccion)}, entonces:
\[
\exists! \vec{v} \in \mathcal{U} / \exp_{p_0}(\vec{v}) = p
\]

\noindent \textbf{\underline{-Existencia-}:} Construyamos la geodésica. \textcolor{blue}{\footnotesize (quiero ver $p = \exp_{p_0}(\vec{v}) = \gamma_v(1)$)}

Sabemos que con las condiciones iniciales será:
\[
\gamma_v : I_v \longrightarrow S \ / \ \gamma_v(0) = p_0, \ \gamma_v'(0) = \vec{v}. \quad \text{\textcolor{blue}{\footnotesize (geodésica maximal)}}
\]

Definimos entonces: $\gamma_p := \gamma_v\big|_{[0,1]}$. Veamos que cumple el teorema:

\begin{itemize}
    \item $\gamma_p(0) = \gamma_v(0) = p_0$.
    \item $\gamma_p(1) = \gamma_v(1) = \exp_{p_0}(\vec{v}) = p$
\end{itemize}

Faltaría ver que $\gamma_p$ ``no se sale'' de $V$, i.e., ¿$\gamma_p(t) \in V \ \forall t \in [0,1]$?

Sea $t \in (0,1)$. Como $\vec{v} \in \mathcal{U}$ y $\mathcal{U}$ es estrellado, $t \cdot \vec{v} \in \mathcal{U} \implies \exp_{p_0}(t\vec{v}) \in V$.

Pero:
\[
\exp_{p_0}(t\vec{v}) \underset{\text{\textcolor{blue}{\tiny Lema Homogeneidad Geodésicas}}}{=} \gamma_{tv}(1) = \gamma_v(t) \in V \quad \checkmark
\]





La idea es ver que existe una única geodésica que parte de $p_0$, llega a $p$ y \textbf{no se sale} de $V$.

Como $V$ es un entorno normal de $p_0$, existe un abierto $U \subset D_{p_0} \subset T_{p_0}S$ que es \textbf{estrellado} respecto del origen, tal que la restricción de la exponencial $\exp_{p_0}: U \longrightarrow V$ es un difeomorfismo.
Dado que $p \in V$ y la exponencial es biyectiva en $U$, existe un único vector $\vec{v} \in U$ tal que $\exp_{p_0}(\vec{v}) = p$.

\noindent \textbf{\underline{Unicidad}:} Supongamos que existe otra $\alpha: [0,1] \longrightarrow V$ geodésica con $\alpha(0) = p_0, \alpha(1) = p$.

Consideremos $\vec{w} = \alpha'(0)$, por la unicidad de geodésicas maximales se tiene que: $\alpha = \gamma_w|_{[0,1]}$.

Se tiene $p = \alpha(1) = \gamma_w(1) = \exp_{p_0}(\vec{w})$. Entonces, como $\exp_{p_0}: \mathcal{U} \longrightarrow V$ es un difeomorfismo y $\exp_{p_0}(\vec{v}) = p$ (con $\vec{v} \in \mathcal{U}$), si $\vec{w} \in \mathcal{U}$, por la inyectividad se tendrá $\vec{v} = \vec{w}$ y habremos terminado.

\vspace{0.3cm}

\noindent Faltará entonces ver que $\vec{w} \in \mathcal{U}$.

Se tiene que $\alpha([0,1])$ es compacto contenido en $V$ abierto \textcolor{blue}{\footnotesize (pq $\alpha$ es continua y $[0,1]$ compacto)}.
Pero la frontera de $V$ es problemática, luego, gracias a que $\alpha([0,1])$ es compacto, $\exists V_0$ abierto tal que:
\[
\alpha([0,1]) \subset V_0 \subset \overline{V_0} \subset V
\]

Por el difeomorfismo $\exp_{p_0}$ se pasa todo a $T_{p_0}S$, sean:
\[
\begin{aligned}
\tilde{\alpha} &= \exp_{p_0}^{-1}(\alpha) \\
\mathcal{U}_0 &= \exp_{p_0}^{-1}(V_0)
\end{aligned}
\quad \Bigg\} \quad \tilde{\alpha}([0,1]) \subset \mathcal{U}_0 \subset \overline{\mathcal{U}_0} \subset \mathcal{U}
\]

% --- DIBUJO ESQUEMÁTICO ---
\begin{center}
    \includegraphics[width=0.3\textwidth]{Imagenes/Unicidad Exponencial Geodésica.png} 
    \par \textit{\small Esquema del paso del entorno $V$ en $S$ al entorno $\mathcal{U}$ en $T_{p_0}S$ mediante $\exp_{p_0}^{-1}$.}
\end{center}

\noindent \textcolor{blue}{(veamos que $\vec{w} \in \mathcal{U}_0$ por red. abs.)}

Si el vector $\vec{w} \notin \mathcal{U}_0$, lo queremos "contraer" para que esté en $\mathcal{U}$, pero siga sin estar en $\mathcal{U}_0$, i.e., que vaya a la frontera de $\mathcal{U}_0$, en $\overline{\mathcal{U}_0}$.

Sea $t_0 \in (0,1)$ tal que $t_0 \cdot \vec{w} \in \mathcal{U} \setminus \mathcal{U}_0$, se tiene:
\[
\underbrace{\exp_{p_0}(t_0 \vec{w})}_{\in \mathcal{U}} = \gamma_{t_0\vec{w}}(1) = \gamma_w(t_0) = \alpha(t_0) = \underbrace{\exp_{p_0}(\tilde{\alpha}(t_0))}_{\in \mathcal{U}}
\]

Como $\exp_{p_0}(\mathcal{U})$ en $\mathcal{U}$ es un difeomorfismo, se tiene: $t_0 \cdot \vec{w} = \tilde{\alpha}(t_0)$ \quad \textbf{\Large} \quad porque $t_0 \cdot \vec{w} \notin \mathcal{U}_0$, pero $\tilde{\alpha}(t_0) \in \mathcal{U}_0$.

\[
(\exp_{p_0}(\vec{v}) = \exp_{p_0}(\vec{w}) \underset{\vec{v}, \vec{w} \in \mathcal{U}}{\implies} \vec{v} = \vec{w} \implies \gamma_v = \gamma_w)
\]


\end{proof}
% -------------------------------------------------------------------------
\section{El Lema de Gauss y Minimización}

Este es uno de los resultados clave que relaciona la geometría métrica (distancias) con las geodésicas.

\begin{observacion}{Identificación del espacio tangente}
\textbf{Nota:} Sabemos que $\exp_p: D_p \longrightarrow S$ y, dado $\vec{v} \in D_p$, se tiene que:
\[
d(\exp_p)_{\vec{v}} : \underbrace{T_{\vec{v}}D_p}_{\equiv T_pS} \longrightarrow T_{\exp_p(\vec{v})}S,
\]
pero $T_{\vec{v}}D_p$ no es realmente igual a $T_pS$, pero se identifican, porque realmente cambia el origen, y si cambia el origen, cambia el punto de referencia desde el que se toma $\vec{v}$.

\begin{center}
\includegraphics[width=0.7\textwidth]{Imagenes/Observación Lema de Gauss.png}
\end{center}
\end{observacion}

\begin{teorema}{Lema de Gauss}
Sea $p \in S$, $v \in D_p \setminus \{0\}$ y $w \in T_v(T_pS) \cong T_pS$. (Excluimos el caso en el que $v=\vec{0}$ porque en este caso, $d(\exp_p)_{\vec{0}}=1_{T_pS}$).
\begin{enumerate}
    \item \textbf{Radial:} Si $w$ y $v$ son colineales, entonces $|d(\exp_p)_v(w)| = |w|$ (la exponencial preserva longitudes en dirección radial).
    \item \textbf{Ortogonal:} Si $w \perp v$, entonces $d(\exp_p)_v(v) \perp d(\exp_p)_v(w)$ (la exponencial preserva la ortogonalidad con respecto a la dirección radial).
\end{enumerate}
\end{teorema}

\begin{proof}[Demostración (del Lema de Gauss)]

\textbf{(i)} Supongamos que $\vec{w}$ y $\vec{v}$ son colineales, i.e., $\vec{w} = \lambda\vec{v}$, entonces se tiene:

\[
d(\exp_p)_{\vec{v}}(\vec{w}) = \frac{d}{dt}\bigg|_{t=0} \exp_p(\alpha(t))
\]

Definimos la curva en el dominio $D_p$:
\[
\begin{cases}
\alpha: I \longrightarrow D_p \subset T_pS \\
\alpha(0)=\vec{v}, \ \alpha'(0)=\vec{w}
\end{cases}
\implies \alpha(t) = \vec{v} + t\vec{w} = \vec{v} + t\lambda\vec{v} = (1+t\lambda)\vec{v}
\]
\textit{\textcolor{blue}{\footnotesize (nota: porque $\exp_p: D_p \to S$)}}

Sustituyendo en la diferencial:
\[
\frac{d}{dt}\bigg|_{t=0} \exp_p((1+t\lambda)\vec{v}) = \frac{d}{dt}\bigg|_{t=0} \gamma_v(1+t\lambda) = \gamma_v'(1+t\lambda)\bigg|_{t=0} \cdot \lambda = \gamma_v'(1) \cdot \lambda
\]

Luego, tomando módulos y usando que la norma de la velocidad es constante por ser geodésica ($|\gamma_v'(1)| = |\gamma_v'(0)|$):
\[
|d(\exp_p)_{\vec{v}}(\vec{w})| = |\lambda| \cdot |\gamma_v'(1)| = |\lambda| \cdot |\gamma_v'(0)| = |\lambda| \cdot |\vec{v}| = |\lambda\vec{v}| = |\vec{w}| \quad \square
\]

\vspace{0.5cm}

\textbf{(ii)} Tenemos $\vec{w} \perp \vec{v}$. Llamamos $\alpha(t) = \vec{v} + t\vec{w}$.

Definimos $\varphi(s,t) = \exp_p(s \cdot \alpha(t))$. Habría que determinar su dominio...

\begin{center}
    % Aquí puedes poner tu captura o dejar el espacio
    \includegraphics[width=0.6\textwidth]{Imagenes/Demostración Lema de Gauss.png}
    
    \vspace{0.2cm}
    \textit{\small Esquema del dominio $D_p$ y la construcción vectorial.}
\end{center}

\noindent \textbf{Notas sobre el diagrama:}
\begin{itemize}
    \item Sobre $\vec{w}$: \textit{\textcolor{blue}{Nada nos asegura que $\vec{w} \in D_p$ ($\vec{w} \in T_{\vec{v}}D_p \equiv T_pS$).}}
    \item Sobre el vector resultante: \textit{\textcolor{blue}{Pero $s \cdot (\vec{v} + t\vec{w})$ sí lo queremos en $D_p$.}}
    \item Sobre el punto de origen $\vec{v}$: \textit{En este punto estará el plano tangente a $D_p$, entonces los vectores se apoyan ahí, por eso está ahí $\vec{w}$.}
\end{itemize}

Entonces el dominio serán intervalos para $t$ y $s$ tales que $s(\vec{v} + t\vec{w}) \in D_p$ ($\varphi$ tendrá un dominio de la forma $\varphi:(-\epsilon',1+\epsilon')\times(-\epsilon,\epsilon) \to S$), pero como la demostración es muy tediosa (está en su libro), supondremos que tenemos esos dos abiertos que nos definen nuestro dominio de $\varphi$. Hecho eso...



Calculamos las derivadas parciales de $\varphi(s,t)$ (llamamos $\alpha(t):=(\vec{v}+t\vec{w})$):
\[
\frac{\partial \varphi}{\partial t} = d(\exp_p)_{s\cdot \alpha(t)} (s\alpha'(t)) \implies \frac{\partial \varphi}{\partial t}(1,0) = d(\exp_p)_{\vec{v}}(\vec{w})
\]
\[
\frac{\partial \varphi}{\partial s} = d(\exp_p)_{s\cdot \alpha(t)} (\alpha(t)) \implies \frac{\partial \varphi}{\partial s}(1,0) = d(\exp_p)_{\vec{v}}(\vec{v})
\]

Tenemos entonces que probar que $\frac{\partial \varphi}{\partial t}(1,0) \perp \frac{\partial \varphi}{\partial s}(1,0)$, pero más aún...
\[
\text{¿ } \left\langle \frac{\partial \varphi}{\partial t}(s,0), \frac{\partial \varphi}{\partial s}(s,0) \right\rangle = 0 \text{ ?}
\]

Sea $f(s) = \left\langle \frac{\partial \varphi}{\partial t}(s,0), \frac{\partial \varphi}{\partial s}(s,0) \right\rangle$. Si vemos que en un punto $f$ vale 0 y que $f'=0$, ya estará.

$\to$ \textbf{Veamos lo que vale $f(0)$}. Se tiene que $\frac{\partial \varphi}{\partial t}(0,0) = d(\exp_p)_{\vec{0}}(\vec{0}) = d(\text{Id})_{\vec{0}}(\vec{0}) = \vec{0}$.
\[
\text{Luego } f(0) = \left\langle \frac{\partial \varphi}{\partial t}(0,0), \frac{\partial \varphi}{\partial s}(0,0) \right\rangle = \langle \vec{0}, \vec{v} \rangle = 0
\]

$\to$ \textbf{¿$f'(s)=0$?} Derivando el producto escalar:
\[
f'(s) = \underbrace{\left\langle \frac{\partial^2 \varphi}{\partial s \partial t}(s,0), \frac{\partial \varphi}{\partial s} \right\rangle}_{\text{\textcircled{1}}} + \underbrace{\left\langle \frac{\partial \varphi}{\partial t}, \frac{\partial^2 \varphi}{\partial s^2} \right\rangle}_{\text{\textcircled{2}}}
\]
Veamos cada uno por separado.

\vspace{0.3cm}

\textbf{Sobre \textcircled{2}:}
\[
\frac{\partial^2 \varphi}{\partial s^2}(s,0) = \frac{d^2}{ds^2}(\exp_p(s\cdot \vec{v})) = \gamma_v''(s)
\]
\textit{\textcolor{blue}{(es la $\varphi(s,t)$ en el $(s,0)$, y como $t$ no influye en la derivada, se puede evaluar antes de derivar, como si tuviéramos $\tilde{\varphi}(s) = \varphi(s,0)$)}}.
Como $\gamma_v$ es geodésica, su aceleración $\gamma_v''$ va en la dirección del normal.

Para $\frac{\partial \varphi}{\partial t}(s,0)$, si fijamos en $\varphi(s,t)$ la $s$, por ejemplo $s_0$, se nos queda que $\varphi(s_0,t) =:  \beta_{s_0}(t)$ es una curva en la superficie. Entonces $\frac{\partial \varphi}{\partial t}(s,0) = \beta_s'(0)$ es el vector velocidad de una curva en $S$, entonces es \textbf{tangente}.
Como $\frac{\partial^2 \varphi}{\partial s^2}$ va en la dirección del normal, entonces son ortogonales:
\[
\implies \text{\textcircled{2}} = \left\langle \frac{\partial \varphi}{\partial t}, \frac{\partial^2 \varphi}{\partial s^2} \right\rangle = 0
\]
\textit{\textbf{Observemos que hasta ahora no hemos usado en ningún momento la ortogonalidad de $\vec{v}$ y $\vec{w}$. O sea, que estos argumentos y resultados que estamos sacando de aquí, realmente, nos valen para cualquier par de vectores.  }}
\vspace{0.3cm}

\textbf{Sobre \textcircled{1}:}
\[
\left\langle \frac{\partial^2 \varphi}{\partial s \partial t}(s,0), \frac{\partial \varphi}{\partial s}(s,0) \right\rangle = \left\langle \frac{d}{dt}\Big|_{t=0} \frac{\partial \varphi}{\partial s}(s,t), \frac{\partial \varphi}{\partial s}(s,t) \Big|_{t=0} \right\rangle = 
\]
\[
= \frac{1}{2} \frac{d}{dt}\Big|_{t=0} \left\langle \frac{\partial \varphi}{\partial s}(s,t), \frac{\partial \varphi}{\partial s}(s,t) \right\rangle = \frac{1}{2} \frac{d}{dt}\Big|_{t=0} \left| \frac{\partial \varphi}{\partial s} \right|^2 = \circledast
\]

Pero se tiene que: $\frac{\partial \varphi}{\partial s} = \frac{d}{ds}(\exp_p(s \cdot \alpha(t))) = \frac{d}{ds}(\gamma_{\alpha(t)}(s)) = \gamma_{\alpha(t)}'(s)$.
\textit{\textcolor{blue}{(la derivada no involucra a $t$, entonces $\alpha(t)$ es un vector ``constante'')}}

Luego:
\[
\left| \frac{\partial \varphi}{\partial s} \right|^2 = |\gamma_{\alpha(t)}'(s)|^2 \underset{\uparrow}{=} |\gamma_{\alpha(t)}'(0)|^2 = |\alpha(t)|^2 = |\vec{v} + t\vec{w}|^2 = |\vec{v}|^2 + t^2|\vec{w}|^2 + 2t\langle \vec{v}, \vec{w} \rangle
\]
\textit{\textcolor{blue}{(es constante por ser geodésica)}}

\vspace{0.3cm}

\textbf{Entonces:}
\[
\circledast = \frac{1}{2} \cdot \frac{\partial}{\partial t}\Big|_{t=0} (|\vec{v}|^2 + t^2|\vec{w}|^2 + 2t\langle \vec{v}, \vec{w} \rangle) = \frac{1}{2} \cdot 2\langle \vec{v}, \vec{w} \rangle = \langle \vec{v}, \vec{w} \rangle = 0
\]
\textit{\textcolor{blue}{(por hip. $\vec{v} \perp \vec{w}$)}}

Concluimos entonces que $f'(s) = \text{\textcircled{1}} + \text{\textcircled{2}} = 0$, como queríamos probar. 
\end{proof}

\begin{definicion}{Discos y circunferencias geodésicas}
Sea $r > 0$ tal que $D(\vec{0}, r) \subset D_p$ ó $\mathbb{S}(\vec{0}, r) \subset D_p$. Definimos:
\[
\mathcal{D}(p, r) := \exp_p(D(\vec{0}, r)) = \text{\textbf{disco geodésico}}
\]
\textit{\textcolor{blue}{\footnotesize (como dejar caer el disco de $D_p$ sobre la superficie)}}

\[
\mathcal{S}(p, r) := \exp_p(\mathbb{S}(\vec{0}, r)) = \text{\textbf{circunferencia geodésica}}
\]

El \textbf{radio geodésico} que parte de $p$ es la imagen por $\exp_p$ de una semirrecta en $T_pS$ que parte de $\vec{0}$.
\end{definicion}

\begin{lema}{de Gauss (Versión geométrica)}
En una superficie regular, las circunferencias geodésicas y los radios geodésicos se cortan \textbf{ortogonalmente}.
\end{lema}

\begin{proof}

La primera versión (algebraica) nos decía que como $\vec{v} \perp \vec{w}$, entonces:
\[
d(\exp_p)_{\vec{v}}(\vec{v}) \perp d(\exp_p)_{\vec{v}}(\vec{w})
\]

Pero analicemos qué significa cada término geométricamente:
\begin{center}
    \includegraphics[width=0.4\textwidth]{Imagenes/Versión Geométrica Lema Gauss.png}
    
\end{center}

\textbf{1. El vector radial:}
Se tiene que:
\[
d(\exp_p)_{\vec{v}}(\vec{v}) = \frac{d}{dt}\Big|_{t=0} \exp_p(\alpha(t)) = (\exp_p \circ \alpha)'(0)
\]
donde $\alpha(t) = \vec{v} + t\vec{v}$ (recta radial en el tangente) con $\alpha(0)=\vec{v}, \alpha'(0)=\vec{v}$.
\textit{Esto es el vector velocidad de la curva $\exp_p \circ \alpha$, que es un \textbf{Radio Geodésico}.}

\textbf{2. El vector tangencial:}
\[
d(\exp_p)_{\vec{v}}(\vec{w}) = \frac{d}{dt}\Big|_{t=0} \exp_p(\beta(t)) = (\exp_p \circ \beta)'(0)
\]
donde $\beta(t)$ es la curva en el plano tangente (la circunferencia) tal que $\beta(0)=\vec{v}$ y $\beta'(0)=\vec{w}$ (tangente a la circunferencia en $T_pS$).
\textit{Esto es el vector tangente a la imagen de la circunferencia, es decir, a la \textbf{circunferencia geodésica}.}

\textbf{Conclusión:}
Hemos probado entonces que los radios geodésicos cortan ortogonalmente a las circunferencias geodésicas.
\end{proof}
%% PROPIEDAD MINIMIZANTE DE LAS GEODÉSICAS %%

\subsection{Propiedad Minimizante}
\begin{teorema}{Propiedad minimizante de las geodésicas}
Sea $S$ una superficie regular, $V$ un \underline{entorno normal} centrado en $p_0 \in S$ y $p \in V$.

\begin{enumerate}[label=(\roman*)]
    \item El segmento de la geodésica radial $\gamma_p: [0,1] \longrightarrow V$ que une $p_0 = \gamma_p(0)$ y $p = \gamma_p(1)$ es la única curva \textbf{contenida en $V$} de menor longitud uniendo $p_0$ y $p$. 
   
    Es decir, si $\alpha: [a,b] \longrightarrow V$ es otra curva diferenciable en $V$ con $\alpha(a)=p_0$ y $\alpha(b)=p$, entonces:
    \[
    L_0^1(\gamma_p) \le L_a^b(\alpha),
    \]
    dándose la igualdad si, y solo si, $\alpha$ es una reparametrización de $\gamma_p$.

     \textbf{Importante:} Si no parto de este punto $p_0$, NO PUEDO ASEGURAR NADA. 
    
    \item Además, si $r>0$ es tal que $\mathcal{D}(p_0, r) \quad (\text{disco geodésico})\subset V(p_0)$, dado $p \in \mathcal{D}(p_0, r)$ se tiene que:
    \[
    L_0^1(\gamma_p) \le L_a^b(\alpha)
    \]
    \textbf{para toda curva $\alpha: [a,b] \longrightarrow S$ }verificando $\alpha(a)=p_0$ y $\alpha(b)=p$.
\end{enumerate}
\end{teorema}

\begin{observacion}{Notas e interpretación geométrica}
\begin{enumerate}[label=(\arabic*)]
    \item Es muy importante que $V$ sea entorno normal.

    \item La geodésica radial es la curva más corta uniendo esos puntos \textbf{DENTRO} del entorno $V$. Fuera de él no podemos asegurar nada.
    Sin embargo, para el disco geodésico $\mathcal{D}(p_0, r) \subset V$ sí podemos asegurar que la geodésica radial que une el centro con cualquier otro punto del disco es la curva más corta que lo hace en \underline{\textbf{TODA}} la superficie.

    \begin{center}
        % Espacio para el dibujo de la superficie con el entorno V y la curva saliéndose
        \includegraphics[width=0.4\textwidth]{Imagenes/Minimizante Geodésicas.png} 
    \end{center}

    \item \textbf{En el cilindro por ejemplo:}
    
    \begin{center}
        % Espacio para el dibujo del cilindro y la franja
        \includegraphics[width=0.4\textwidth]{Imagenes/Entorno Normal Cilindro.png}
    \end{center}

    La franja sombreada, $V$, es un entorno normal.
    La curva $\alpha$ (la que da la vuelta por detrás)  valdría para el Teorema porque no se sale de $V$ ($\alpha \subset V$), pero no podemos asegurarlo para $\gamma_p \not\subset V$ (el segmento recto).
\end{enumerate}
\end{observacion}
\begin{proof}[Demostración del Teorema (Apartado i)]

\textbf{(i)} $V$ es un entorno normal de $p_0$, entonces $\exp_{p_0}: \mathcal{U} \longrightarrow V$ es un \textbf{difeomorfismo}, por lo que si tenemos $p \in V \implies \exists! \vec{v} \in \mathcal{U} / \exp_{p_0}(\vec{v}) = p$, y con ese vector $\vec{v}$ construimos la radial:
\[
\gamma_p := \gamma_v\big|_{[0,1]} \ / \ \gamma_p(0)=p_0, \ \gamma_p(1)=p, \ \gamma_p'(0)=\vec{v}
\]

Supongamos que tenemos otra curva $\alpha: [a,b] \longrightarrow V$ tal que $\alpha(a)=p_0$ y $\alpha(b)=p$.
\[
\text{¿ } L_a^b(\alpha) \ge L_0^1(\gamma_p) \text{ ?}
\]
Tenemos que:
\[
L_0^1(\gamma_p) = \int_0^1 |\gamma_p'(t)| \, dt = \int_0^1 |\gamma_p'(0)| \, dt = \int_0^1 |\vec{v}| \, dt = |\vec{v}|
\]
Tendremos entonces que ver... ¿$L_a^b(\alpha) \ge |\vec{v}|$?

\begin{enumerate}
    \item Si $p = p_0$: $\gamma_p \equiv p_0 \implies \vec{v} = \vec{0} \implies L_a^b(\alpha) \ge 0$ obviamente. \checkmark

    \item Si $p \neq p_0$: Si la curva vuelve a pasar muchas veces por $p_0$, todo ese trozo nos da igual, nos quedamos con el trozo que no vuelve a pasar.

    
    Sea $t_0 \in (a,b)$ tal que $\forall t > t_0, \ \alpha(t) \neq p_0$ (y $\alpha(t_0)=p_0$).
    Bastaría entonces probar que: $L_{t_0}^b(\alpha) \ge |\vec{v}|$.
    
    Reparametrizaremos $\alpha$ para que en vez de en $[t_0, b]$ esté en $[0,1]$ para ahorrarnos términos.
    Es decir, $\alpha$ será tal que $\alpha: [0,1] \longrightarrow \textbf{V}$ y $\alpha(t) \neq p_0 \ \forall t> 0$. ¿$L_0^1(\alpha) \ge |\vec{v}|$?
\end{enumerate}

\noindent \textbf{Paso a polares en el tangente:}
Observemos que $\tilde{\alpha}(t) \neq \vec{0} \ \forall t > 0$, porque (por reducción al absurdo) si $\exists t > 0 / \tilde{\alpha}(t) = \vec{0} \implies \alpha(t) = \exp_{p_0}(\tilde{\alpha}(t)) = p_0$ (hemos descartado este caso). 

Definimos entonces:
\[
r(t) := \begin{cases} |\tilde{\alpha}(t)| & \text{si } t > 0 \\ 0 & \text{si } t = 0 \end{cases} \quad (\text{\textcolor{blue}{escalar}})
\]
\[
V(t) := \frac{\tilde{\alpha}(t)}{|\tilde{\alpha}(t)|} = \frac{\tilde{\alpha}(t)}{r(t)}, \quad t > 0 \quad (\text{\textcolor{blue}{vectorial}, } V(0) \text{ ni se define})
\]

Intentemos expresar $\alpha$ tal que $\alpha'$ tenga una expresión algo más manejable para poder calcular la integral de $L_0^1(\alpha)$ y acotarla.
Tenemos que $\alpha(t) = \exp_{p_0}(\tilde{\alpha}(t)) = \exp_{p_0}(r(t) \cdot V(t))$.

\textit{Como aquí el vector depende de $t$, no nos ayudaría expresarlo como $\gamma_{V(t)}(r(t))$ porque eso tampoco sabemos derivarlo, así que hagámoslo a pelo:}

Aplicamos la regla de la cadena diferencial:
\[
\alpha'(t) = d(\exp_{p_0})_{r \cdot V} (r'V + rV') = r' \cdot d(\exp_{p_0})_{r \cdot V}(V) + r \cdot d(\exp_{p_0})_{r \cdot V}(V')
\]

Elevamos al cuadrado la norma para usar el Lema de Gauss:
\[
|\alpha'(t)|^2 = (r')^2 \left| d(\exp_{p_0})_{rV}(V) \right|^2 + r^2 \left| d(\exp_{p_0})_{rV}(V') \right|^2 + 2rr' \underbrace{\left\langle d(\exp_{p_0})_{rV}(V), d(\exp_{p_0})_{rV}(V') \right\rangle}_{0}
\]

\textbf{Justificación del 0:}
Como $\langle V, V \rangle = 1 \implies 2\langle V', V \rangle = 0 \implies V \perp V'$.
Por el Lema de Gauss (versión geométrica/algebraica):
\[
d(\exp_{p_0})_{rV}(V) \perp d(\exp_{p_0})_{rV}(V')
\]
\textit{\textcolor{blue}{\footnotesize (no molesta que sea $rV$ en vez de $v$, pq $V \perp V' \implies rV \perp rV'$)}}

\textbf{Simplificación de la norma:}
Se tiene también: $\left| d(\exp_{p_0})_{rV}(V) \right|^2 = |V|^2 = 1$ (porque la exponencial preserva normas radiales / Lema de Gauss).

Luego:
\[
|\alpha'(t)|^2 = (r')^2 + \underbrace{r^2 \left| d(\exp_{p_0})_{rV}(V') \right|^2}_{\ge 0 \quad \textcircled{1}} \ge (r')^2
\]
Tomando raíz cuadrada:
\[
|\alpha'(t)| \ge |r'| \ge r'\quad \textcircled{2}
\]

\textbf{Conclusión de la desigualdad:}
Se tiene:
\[
L_0^1(\alpha) = \int_0^1 |\alpha'(t)| \, dt \ge \lim_{\varepsilon \to 0} \int_{\varepsilon}^1 r' \, dt =\lim_{\varepsilon \to 0} r(1) - r (\varepsilon) = r(1) - r(0) = |\tilde{\alpha}(1)| = |\exp_{p_0}^{-1}(\alpha(1))| =  |\exp_{p_0}^{-1}(p)|=|\vec{v}| \quad \checkmark
\]

\vspace{0.5cm}
\hrule
\vspace{0.5cm}

\noindent \textbf{Falta ver la igualdad:}
Supongamos que $L_0^1(\alpha) = |\vec{v}|$. Esto implica igualdad en \textcircled{1} y en \textcircled{2}.

\begin{enumerate}[label=\textcircled{\arabic*}]
    \setcounter{enumi}{1}
    \item $|r'| = r' \iff r' > 0 \iff r$ es estrictamente creciente.
    \item $|d(\exp_{p_0})_{rV}(V')|^2 = 0 \iff d(\exp_{p_0})_{rV}(V') = 0 \iff V' = 0 \iff V = \text{cte}\implies V(t)=V(1)=\frac{\vec{v}}{|\vec{v}|}$.
\end{enumerate}
En esta última cadena de igualdades estamos usando que $\alpha\subset V\implies \tilde{\alpha} \subset U\implies d(\exp_{p_0})_{rV}$ es un isomorfismo lineal.\\
Si $V(t)$ es constante, digamos $V_0$, entonces:
\[
\alpha(t) = \exp_{p_0}(r(t)V(t)) = \exp_{p_0}\left( r(t) \frac{\vec{v}}{|\vec{v}|} \right) = \gamma_v\left( \frac{r(t)}{|\vec{v}|} \right) = \gamma_p\left( \frac{r(t)}{|\vec{v}|} \right)
\]
Y esto es una \textbf{reparametrización de $\gamma_p$}. \qed
\end{proof}


%%%% PARTE 2 %%%%%%
\begin{proof}[Demostración (Apartado ii)]
Sea $r > 0$ tal que $\mathcal{D}(p_0, r) \subset V$.
Dado $p \in \mathcal{D}(p_0, r) \subset V \implies \exists! \vec{v} \in \mathcal{U} / \exp_{p_0}(\vec{v}) = p$.

Sea $\gamma_p : [0,1] \longrightarrow V$ la geodésica radial con $\gamma_p(0) = p_0$ y $\gamma_p(1) = p$.
Tomamos $\alpha$ otra curva que los une y la reparametrizamos $\alpha : [0,1] \longrightarrow S$. O sea, $\alpha(0)=p_0, \alpha(1)=p$.


\noindent \textbf{\underline{Caso 1.-}} Si $\alpha|_{[0,1]} \subset \mathcal{D}(p_0, r) \subset V \implies$
Por (i) se tiene $L_0^1(\gamma_p) \le L_0^1(\alpha)$, dándose la igualdad si $\alpha$ es una reparametrización de $\gamma_p$ (en general se cumple si $\alpha$ está en $V$, pero nos interesa que esté o no en el disco).

\vspace{0.3cm}


\begin{center}
    % Espacio para el dibujo del Caso 2 (curva saliéndose del disco)
    \includegraphics[width=0.5\textwidth]{Imagenes/Diagrama_Disco.png}
    \par \textit{\small Esquema: La curva $\alpha$ se sale del disco $\mathcal{D}(p_0, r)$ y corta a la frontera del disco más pequeño en $p^*$.}
\end{center}



\noindent \textbf{\underline{Caso 2.-}} Supongamos que $\alpha \not\subset \mathcal{D}(p_0, r)$.

Se tiene $p = \exp_{p_0}(\vec{v}) \in \mathcal{D}(p_0, r) = \exp_{p_0}(D(\vec{0}, r))$ y como $\exp_{p_0}$ es un difeomorfismo:
\[
\implies \vec{v} \in D(\vec{0}, r) \implies |\vec{v}| < r \quad \text{\small (podemos entonces encontrar un número en medio)}
\]

Sea $r^* > 0$ tal que $|\vec{v}| < r^* < r$, i.e., $\vec{v} \in D(\vec{0}, r^*) \subsetneq D(\vec{0}, r)$.
\[
\overset{\exp_{p_0}}{\implies} p \in \mathcal{D}(p_0, r^*) \subset \mathcal{D}(p_0, r)
\]

Sea $t_0 = \inf \{ t \in [0,1] : \alpha(t) \notin \mathcal{D}(p_0, r^*) \}$ (el primer valor para el que $\alpha$ se "sale" del disco pequeño).
Luego $\alpha([0, t_0)) \subset \mathcal{D}(p_0, r^*)$ y trabajando ahí tenemos:

Llamando $p^* = \alpha(t_0) \in \mathcal{S}(p_0, r^*) \subset \mathcal{D}(p_0, r) \subset V$ (está en la frontera). \textit{(Recordemos que denotamos a $\mathcal{S}$ como la frontera)}.\\
Sea $\gamma_{p^*}$ la geodésica radial que une $\gamma_{p^*}(0)=p_0$ con $\gamma_{p^*}(1)=p^*$, entonces por (i) se tiene:

\[
L_0^1(\alpha) \ge \underbrace{L_0^{t_0}(\alpha)}_{\text{es un trozo de } \alpha} \ge L_0^1(\gamma_{p^*}) = \int_0^1 |\gamma_{p^*}'(t)| \, dt = \int_0^1 |\gamma_{p^*}'(0)| \, dt = |\gamma_{p^*}'(0)| = \circledast
\]

Definimos $\vec{w} := \gamma_{p^*}'(0)$ (esto no lo conocemos, pero $\gamma_{p^*}' (0) = \gamma_w'(0) = \vec{w}$).
Sabemos que $p^* = \exp_{p_0}(\vec{w}) \in \exp_{p_0}(\mathcal{S}(\vec{0}, r^*)) \implies |\vec{w}| = r^*$.

Entonces retomando $\circledast$:
\[
\circledast = r^* > |\vec{v}| = L_0^1(\gamma_p) \implies L_0^1(\alpha) \ge L_0^1(\gamma_p)
\]
\end{proof}

% -------------------------------------------------------------------------

\section{Coordenadas normales y geodésicas polares}
\subsection{Sistema de coordenadas Normales}
Sean $p_0 \in S$ un punto de una superficie regular $S$ y $V$ un entorno normal de $p_0$.
Sea $\mathcal{U}$ el abierto estrellado para el cual $\exp_{p_0}: \mathcal{U} \longrightarrow V$ es un difeomorfismo.

\noindent \textbf{¿Cómo podemos dar una \underline{parametrización} para $V$?}

\begin{itemize}
    \item Sea $\{\vec{e}_1, \vec{e}_2\}$ una base ortonormal de $T_{p_0}S$.
    \item Sea $\phi: \mathbb{R}^2 \longrightarrow T_{p_0}S$ dada por $\phi(u,v) = u\cdot\vec{e}_1 + v\cdot\vec{e}_2$, para la cual $\phi(0,0) = \vec{0} \in \mathcal{U}$. \textit{(Nota: esto identifica $\mathbb{R}^2$ con $T_{p_0}S$)}.
    \item Existe un entorno $\mathcal{U}_0 \subset \mathbb{R}^2$ del $(0,0)$ tal que $\phi: \mathcal{U}_0 \longrightarrow \mathcal{U}$ es un difeomorfismo (podemos identificar $\mathcal{U}_0 \equiv \mathcal{U}$).
\end{itemize}

\begin{definicion}{Sistema de coordenadas normales}
La aplicación $X: \mathcal{U} \equiv \mathcal{U}_0 \subset \mathbb{R}^2 \longrightarrow V \subset S$, dada por:
\[
X(u,v) = \exp_{p_0}(\phi(u,v)) = \exp_{p_0}(u\cdot\vec{e}_1 + v\cdot\vec{e}_2)
\]
es una parametrización de $V$, llamada \textbf{\underline{sistema de coordenadas normales en $p_0$}}.
\end{definicion}

\noindent Si $p \in V$, sea $\vec{v} \in T_{p_0}S$ el único vector tal que $\exp_{p_0}(\vec{v}) = p$. Entonces $\vec{v} = v_1\cdot\vec{e}_1 + v_2\cdot\vec{e}_2$.
El par $(v_1, v_2)$ recibe el nombre de \textbf{\underline{coordenadas normales del punto $p$}}.


\textbf{Propiedades de la métrica en coordenadas normales:}
En el punto $p_0$ (el origen de coordenadas $u=v=0$):
\begin{itemize}
    \item $E(0,0)=1, \quad F(0,0)=0, \quad G(0,0)=1$ (la métrica es la identidad).
    \item Las derivadas primeras de los coeficientes métricos se anulan: $E_u = E_v = F_u = F_v = G_u = G_v = 0$.
\end{itemize}

\subsection{Coordenadas geodésicas polares}

Sean $p_0 \in S$ un punto de una superficie regular $S$ y $V$ un entorno normal de $p_0$.
Sea $\mathcal{U}$ el abierto estrellado para el cual $\exp_{p_0}: \mathcal{U} \longrightarrow V$ es un difeomorfismo.

\noindent \textbf{¿Cómo podemos dar una parametrización para $V$?} (Versión polar)

\begin{itemize}
    \item Sea $\{\vec{e}_1, \vec{e}_2\}$ una base ortonormal de $T_{p_0}S$.
    \item Dado $\vec{v} \in T_{p_0}S$, existe un par $(r, \theta)$ tal que $\vec{v} = r\cos\theta\vec{e}_1 + r\sen\theta\vec{e}_2$ (coordenadas polares).
    \item Sea $L = \{ \lambda \cdot \vec{e}_1 : \lambda \ge 0 \}$ (semirrecta) y sea $\phi: (0, +\infty) \times (0, 2\pi) \longrightarrow T_{p_0}S \setminus L$ dada por:
    \[
    \phi(r, \theta) = r\cos\theta\vec{e}_1 + r\sen\theta\vec{e}_2, \quad \text{que es un difeomorfismo.}
    \]
    \item Esto es una parametrización del entorno $V$ pero no me cubre el punto $p_0$
\end{itemize}

\begin{definicion}{Sistema de coordenadas geodésicas polares}
La aplicación $X: \mathcal{U}_0 := \phi^{-1}(\mathcal{U} \setminus L) \longrightarrow V_0 := \exp_{p_0}(\mathcal{U} \setminus L) \subset V$, dada por:
\[
X(r, \theta) = \exp_{p_0}(\phi(r, \theta)) = \exp_{p_0}(r\cos\theta\vec{e}_1 + r\sen\theta\vec{e}_2)
\]
es una parametrización de $V_0$, que se denomina \textbf{\underline{sistema de coordenadas geodésicas polares}} centradas en $p_0$.

Si $p \in V$, sea $\vec{v} \in T_{p_0}S$ el único vector tal que $\exp_{p_0}(\vec{v}) = p$. El par $(r, \theta)$ tal que $\vec{v} = r\cos\theta\vec{e}_1 + r\sen\theta\vec{e}_2$ recibe el nombre de \textbf{\underline{coordenadas geodésicas polares del punto $p$}} centradas en $p_0$.
\end{definicion}

\begin{observacion}{Nota sobre el dominio}
El problema de las coordenadas polares es que se dejan sin cubrir una recta del plano por ser el dominio de $\phi$ un abierto, que sería un radio geodésico en $S$. Además, la parametrización es en un entorno de un punto el cual no cubre a ese punto (el origen), pero esto lo resuelven las propiedades.
\end{observacion}

\begin{tcolorbox}[colback=yellow!10!white, colframe=yellow!50!black, title=Resumen escueto]
\begin{itemize}
    \item \textbf{Coords. normales:} $X(u,v) = \exp_{p_0}(u\cdot\vec{e}_1 + v\cdot\vec{e}_2)$
    \item \textbf{Coords. polares:} $X(r,\theta) = \exp_{p_0}(r\cos\theta\cdot\vec{e}_1 + r\sen\theta\cdot\vec{e}_2)$
\end{itemize}
\begin{flushright}
    \small $\left( \text{donde } \{\vec{e}_1, \vec{e}_2\} \text{ es una base ortonormal de } T_{p_0}S \right)$
\end{flushright}
\end{tcolorbox}

\begin{proposicion}{Propiedades de las coordenadas normales}
\begin{enumerate}[label=(\arabic*)]
    \item Se pueden definir para ``superficies'' de dimensión $n \ge 2$.
    \item $X(0,0) = \exp_{p_0}(\vec{0}) = p_0$.
    \item $X_u(0,0) = \vec{e}_1$ y $X_v(0,0) = \vec{e}_2$.
    \item $E(u,0) = 1$, $F(0,0) = 0$, $G(0,v) = 1$.
    \item En el $(0,0)$: $E_u = E_v = F_u = F_v = G_u = G_v = 0$.
\end{enumerate}
\end{proposicion}

\begin{proof}[Demostración de (3), (4) y (5)]
\textit{(la (2) está ahí mismo)}.

\textbf{1. Para las propiedades (3) y (4):}
\[
X(u,0) = \exp_{p_0}(u \cdot \vec{e}_1) = \gamma_{e_1}(u) \implies X_u(u,0) = \gamma_{e_1}'(u)
\]
\[
X(0,v) = \gamma_{e_2}(v) \implies X_v(0,v) = \gamma_{e_2}'(v)
\]
Entonces, evaluando en el origen:
\[
X_u(0,0) = \gamma_{e_1}'(0) = \vec{e}_1 \quad \text{y} \quad X_v(0,0) = \gamma_{e_2}'(0) = \vec{e}_2
\]
Luego calculamos los coeficientes métricos:
\[
\begin{aligned}
E(u,0) &= \langle X_u(u,0), X_u(u,0) \rangle = |\gamma_{e_1}'(u)|^2 = |\gamma_{e_1}'(0)|^2 = |\vec{e}_1|^2 = 1 \\
G(0,v) &= \langle X_v, X_v \rangle (0,v) = |\vec{e}_2|^2 = 1 \\
F(0,0) &= \langle X_u, X_v \rangle (0,0) = \langle \vec{e}_1, \vec{e}_2 \rangle = 0 \quad \text{\textcolor{gray}{(ortonormal)}}
\end{aligned}
\]
\textit{\footnotesize (donde coinciden $X(u,0)$ y $X(0,v)$ es el $(0,0)$)}.

\vspace{0.3cm}

\textbf{2. Para la propiedad (5):}
Si $\vec{v} = v_1\vec{e}_1 + v_2\vec{e}_2 \implies X(v_1, v_2) = \exp_{p_0}(v_1\vec{e}_1 + v_2\vec{e}_2) = \gamma_v(1)$.

Consideramos la geodésica reescalada:
\[
\gamma_v(t) = \gamma_{tv}(1) = \exp_{p_0}(t\vec{v}) = \exp_{p_0}(t v_1 \vec{e}_1 + t v_2 \vec{e}_2) = X(tv_1, tv_2)
\]
En general, tenemos la curva en coordenadas: $\alpha(t) = X(\tilde{\alpha}(t)) = X(u(t), v(t)) \implies \begin{cases} u(t) = t \cdot v_1 \\ v(t) = t \cdot v_2 \end{cases}$.

Si escribimos el sistema de ecuaciones diferenciales que cumplen las geodésicas (G) y se simplifica (teniendo en cuenta que $u' = v_1, v' = v_2$ y $u'' = v'' = 0$):

\[
\left\{
\begin{aligned}
u'' + (u')^2 \Gamma_{11}^1 + 2u'v' \Gamma_{12}^1 + (v')^2 \Gamma_{22}^1 &= 0 \\
v'' + (u')^2 \Gamma_{11}^2 + 2u'v' \Gamma_{12}^2 + (v')^2 \Gamma_{22}^2 &= 0
\end{aligned}
\right\}
\implies
\left\{
\begin{aligned}
v_1^2 \Gamma_{11}^1 + 2v_1v_2 \Gamma_{12}^1 + v_2^2 \Gamma_{22}^1 &= 0 \\
v_1^2 \Gamma_{11}^2 + 2v_1v_2 \Gamma_{12}^2 + v_2^2 \Gamma_{22}^2 &= 0
\end{aligned}
\right\}
\]

Como el sistema vale para \textbf{cualquier} $\vec{v}=(v_1, v_2)$, probamos con vectores específicos:
\begin{itemize}
    \item Si $\vec{v}=(1,0) \implies 1^2 \cdot \Gamma_{11}^k + 0 + 0 = 0 \implies \Gamma_{11}^1(0,0) = 0, \ \Gamma_{11}^2(0,0) = 0$.
    \item Si $\vec{v}=(0,1) \implies 0 + 0 + 1^2 \cdot \Gamma_{22}^k = 0 \implies \Gamma_{22}^1(0,0) = 0, \ \Gamma_{22}^2(0,0) = 0$.
    \item Si $\vec{v}=(1,1) \implies \Gamma_{11}^k + 2\Gamma_{12}^k + \Gamma_{22}^k = 0$. Como los extremos son 0, queda $2\Gamma_{12}^k = 0 \implies \Gamma_{12}^1 = 0, \Gamma_{12}^2 = 0$.
\end{itemize}

Como todos los $\Gamma_{ij}^k(0,0) = 0$, esto implica que todas las derivadas primeras de los coeficientes métricos se anulan:
\[
E_u = E_v = F_u = F_v = G_u = G_v = 0 \quad (\text{en el } (0,0)) \quad \qed
\]
\end{proof}

\begin{proposicion}{Propiedades de las coordenadas polares}
\begin{enumerate}[label=(\arabic*)]
    \item Solo pueden definirse para superficies 2-dimensionales.
    \item No cubre el propio punto $p_0$, pero...
\end{enumerate}
\end{proposicion}

\begin{teorema}{Propiedades métricas de las coordenadas polares}
Sea $X(r, \theta)$ el sistema de coordenadas geodésicas polares centradas en $p_0$. Entonces se verifica:
\begin{enumerate}[label=\textbullet]
    \item $E(r, \theta) = 1$, $F(r, \theta) = 0$, $G(r, \theta) > 0$ (para $r>0$).
    \item $\lim_{r \to 0} G(r, \theta) = 0$, $\lim_{r \to 0} (\sqrt{G})_r(r, \theta) = 1$.
\end{enumerate}
\end{teorema}

\noindent \textbf{\underline{Notación}:} $\vec{v}_{r\theta} := r \cdot \vec{v}_{\theta} := r \cdot \cos\theta \cdot \vec{e}_1 + r \cdot \sen\theta \cdot \vec{e}_2$.

\begin{proof}[Demostración]
Se tiene por definición: $X(r, \theta) = \exp_{p_0}(r \cdot \vec{v}_{\theta})$. Luego, derivando:
\[
\begin{cases}
X_r(r, \theta) = d(\exp_{p_0})_{r \cdot \vec{v}_{\theta}}(\vec{v}_{\theta}) \\
X_\theta(r, \theta) = d(\exp_{p_0})_{r \cdot \vec{v}_{\theta}}(r \cdot \vec{v}_{\theta}')
\end{cases}
\]
Entonces se tiene:

\textbf{1. Cálculo de E:}
\[
E = |X_r|^2 = |d(\exp_{p_0})_{r\vec{v}_{\theta}}(\vec{v}_{\theta})|^2 \underset{\text{\textcolor{blue}{\tiny Lema de Gauss}}}{=} |\vec{v}_{\theta}|^2 = 1
\]
\textit{\textcolor{blue}{\footnotesize (vectores proporcionales / radiales se conservan)}}

\textbf{2. Cálculo de F:}
\[
F = \langle X_r, X_\theta \rangle = \langle d(\exp_{p_0})_{r\vec{v}_{\theta}}(\vec{v}_{\theta}), d(\exp_{p_0})_{r\vec{v}_{\theta}}(r \cdot \vec{v}_{\theta}') \rangle = 0
\]
\textit{\textcolor{blue}{\footnotesize (Por Lema de Gauss, ya que $\vec{v}_{\theta}' = -\sen\theta\vec{e}_1 + \cos\theta\vec{e}_2 \implies \langle \vec{v}_{\theta}, \vec{v}_{\theta}' \rangle = 0$)}}

\textbf{3. Cálculo de G y sus límites:}
\[
G = \langle X_\theta, X_\theta \rangle = |d(\exp_{p_0})_{r\vec{v}_{\theta}}(r \cdot \vec{v}_{\theta}')|^2 = r^2 |d(\exp_{p_0})_{r\vec{v}_{\theta}}(\vec{v}_{\theta}')|^2 > 0
\]
\[
\implies \lim_{r \to 0} G = \lim_{r \to 0} r^2 \cdot \underbrace{|\dots|^2}_{\text{acotado}} = 0
\]
\textit{\textcolor{blue}{\footnotesize (esto tiene que ser acotado, pero tiende a $|\vec{v}_\theta'|^2=1$)}}

\textit{Esta demostración Maria Ángeles no lo demuestra en clase pero está bien saberlo}\\
Para $\lim_{r \to 0} (\sqrt{G})_r$:
Sabemos que $\sqrt{G} = r |d(\exp_{p_0})_{r\vec{v}_{\theta}}(\vec{v}_{\theta}')|$. Derivando respecto a $r$:
\[
(\sqrt{G})_r = |d(\exp_{p_0})_{r\vec{v}_{\theta}}(\vec{v}_{\theta}')| + r \cdot \frac{\partial}{\partial r}\Big( |d(\exp_{p_0})_{r\vec{v}_{\theta}}(\vec{v}_{\theta}')| \Big)
\]
Tomando el límite cuando $r \to 0$:
\[
\lim_{r \to 0} (\sqrt{G})_r = \underbrace{|d(\exp_{p_0})_{\vec{0}}(\vec{v}_{\theta}')|}_{|\vec{v}_{\theta}'|=1} + \underbrace{0 \cdot (\dots)}_{\to 0} = 1 \quad \square
\]
\textit{\textcolor{blue}{\footnotesize (el término de la derivada debe ser acotado, y lo es)}}
\end{proof}


\begin{observacion}{Interpretación geométrica de las variables polares}
¿Qué ocurre si fijamos alguna de las variables en las coordenadas polares?

\[
X(r, \theta) = \exp_{p_0}(r\cos\theta\vec{e}_1 + r\sen\theta\vec{e}_2) = \exp_{p_0}(r(\underbrace{\cos\theta\vec{e}_1 + \sen\theta\vec{e}_2}_{\vec{V}_\theta}))
\]

\textbf{Fijo $\theta_0$}:
\[
X(r, \theta_0) = \exp_{p_0}(r \cdot \vec{V}_{\theta_0}) = \text{\textbf{radio geodésico}}
\]

\textbf{Fijo $r_0$}:
\[
X(r_0, \theta) = \exp_{p_0}(\underbrace{r_0\cos\theta\vec{e}_1 + r_0\sen\theta\vec{e}_2}_{\partial D(\vec{0}, r_0)}) = \text{\textbf{circunferencia geodésica}}
\]
\end{observacion}




\begin{observacion}{Curvatura de Gauss en coordenadas polares}
Las coordenadas geodésicas polares tienen $E=1, F=0$, i.e., son parametrizaciones ortogonales. Entonces se puede calcular la curvatura de Gauss como:
\[
K = \frac{-1}{2\sqrt{EG}} \left[ \left(\frac{E_\theta}{\sqrt{EG}}\right)_\theta + \left(\frac{G_r}{\sqrt{EG}}\right)_r \right]
\]
Como $E=1 \implies E_\theta=0$ y $\sqrt{EG} = \sqrt{G}$:
\[
K = \frac{-1}{2\sqrt{G}} \left( \frac{G_r}{\sqrt{G}} \right)_r = \frac{-1}{\sqrt{G}} \left( \frac{G_r}{2\sqrt{G}} \right)_r = \frac{-1}{\sqrt{G}} (\sqrt{G})_{rr}
\]
\textit{\textcolor{blue}{\footnotesize (usando que $(\sqrt{G})_r = \frac{G_r}{2\sqrt{G}}$)}}

Es decir, en una parametrización por coordenadas geodésicas polares, la curvatura de Gauss cumple la relación con $G$:
\[
\boxed{\sqrt{G} \cdot K + (\sqrt{G})_{rr} = 0}
\]
\end{observacion}


\noindent Esto se suele utilizar para calcular $G$ cuando esta es difícil y $K$ es ``sencillo''.
De hecho, si $K \equiv \text{cte}$, se resuelve la EDP y sale:
\[
G = \begin{cases} 
r^2 & \text{si } K=0 \\
\frac{1}{K}\sen^2(\sqrt{K}\cdot r) & \text{si } K>0 \\
\frac{-1}{K}\sinh^2(\sqrt{-K}\cdot r) & \text{si } K<0 
\end{cases}
\]
\begin{proof}[Resolución de la ecuación de Jacobi para $K$ cte.]
Consideramos la relación fundamental que liga la curvatura de Gauss con el coeficiente métrico $G$ en coordenadas geodésicas polares:
\[
\sqrt{G} \cdot K + (\sqrt{G})_{rr} = 0
\]
Definimos la función $f(r) = \sqrt{G}(r, \theta)$. La ecuación se reescribe como una EDO lineal de segundo orden con coeficientes constantes:
\[
f''(r) + K f(r) = 0
\]
Para resolver este problema de valores iniciales, utilizamos las propiedades métricas en el origen ($r \to 0$):
\begin{itemize}
    \item $f(0) = \lim_{r \to 0} \sqrt{G} = 0$.
    \item $f'(0) = \lim_{r \to 0} (\sqrt{G})_r = 1$.
\end{itemize}

Analizamos la solución según el signo de la curvatura $K$:

\textbf{Caso 1: $K = 0$ (Plano)}
La ecuación es $f''(r) = 0$. Integrando dos veces obtenemos $f(r) = Ar + B$. 
Imponiendo las condiciones iniciales:
\[ f(0) = 0 \implies B = 0; \quad f'(0) = 1 \implies A = 1 \]
Por tanto, $f(r) = r \implies \boxed{G(r, \theta) = r^2}$.

\textbf{Caso 2: $K > 0$ (Esfera)}
La solución general de $f''(r) + K f(r) = 0$ es $f(r) = A \cos(\sqrt{K}r) + B \sen(\sqrt{K}r)$.
Imponiendo las condiciones iniciales:
\[ f(0) = 0 \implies A = 0; \quad f'(r) = B\sqrt{K}\cos(\sqrt{K}r) \implies f'(0) = B\sqrt{K} = 1 \implies B = \frac{1}{\sqrt{K}} \]
Por tanto, $f(r) = \frac{1}{\sqrt{K}}\sen(\sqrt{K}r) \implies \boxed{G(r, \theta) = \frac{1}{K}\sen^2(\sqrt{K}r)}$.

\textbf{Caso 3: $K < 0$ (Plano hiperbólico)}
La solución general de $f''(r) - |K| f(r) = 0$ es $f(r) = A \cosh(\sqrt{-K}r) + B \senh(\sqrt{-K}r)$.
Imponiendo las condiciones iniciales:
\[ f(0) = 0 \implies A = 0; \quad f'(0) = B\sqrt{-K} = 1 \implies B = \frac{1}{\sqrt{-K}} \]
Por tanto, $f(r) = \frac{1}{\sqrt{-K}}\senh(\sqrt{-K}r) \implies \boxed{G(r, \theta) = -\frac{1}{K}\senh^2(\sqrt{-K}r)}$.
\end{proof}


\section{Teorema de Minding}
\begin{lema}{Isometrías locales y coeficientes métricos}


Sea $\phi: S_1 \to S_2$ una isometría local entre superficies regulares. Entonces, para todo $p \in S_1$, existen parametrizaciones $X: U \to S_1$ (cubriendo $p$) y $\overline{X}: U \to S_2$ (cubriendo $\phi(p)$) definidas sobre el \textbf{mismo abierto} $U \subset \mathbb{R}^2$ tales que los coeficientes de sus primeras formas fundamentales coinciden:
\[
    E = \overline{E}, \quad F = \overline{F}, \quad G = \overline{G}.
\]
\end{lema}

\begin{lema}{Recíproco del lema anterior}
Sean $X: U \to S_1$ y $\overline{X}: U \to S_2$ parametrizaciones tales que $E = \overline{E}$, $F = \overline{F}$ y $G = \overline{G}$ en todo $U$. Entonces, la aplicación composición:
\[
    \phi = \overline{X} \circ X^{-1} : X(U) \to \overline{X}(U)
\]
es una isometría (global) entre los abiertos $X(U)$ y $\overline{X}(U)$. 
\end{lema}




\begin{teorema}{Teorema de Minding}
(Recíproco del Teorema Egregium de Gauss para cuando $K \equiv \text{cte}$).

Sean $S$ y $\bar{S}$ dos superficies regulares con igual curvatura de Gauss \textbf{\underline{CONSTANTE}}. Entonces, $S$ y $\bar{S}$ son localmente isométricas.
\end{teorema}

\begin{proof}[Demostración]



Tenemos $S$ y $\bar{S}$ con $K = \bar{K}$ constante. Queremos ver que son isométricas.
Sean $p \in S, \bar{p} \in \bar{S}$, ¿$\exists V(p), \bar{V}(\bar{p})$ y $\varphi: V \longrightarrow \bar{V}$ isometría?
Para ello, fijamos bases ortonormales $\{\vec{e}_1, \vec{e}_2\}$ de $T_pS$ y $\{\vec{f}_1, \vec{f}_2\}$ de $T_{\bar{p}}\bar{S}$, y definimos la isometría lineal entre los planos tangentes:
\[
\tilde{\varphi}: T_pS \longrightarrow T_{\bar{p}}\bar{S} \quad / \quad \tilde{\varphi}(u\vec{e}_1 + v\vec{e}_2) = u\vec{f}_1 + v\vec{f}_2
\]
Definimos entonces la aplicación:
\[
\varphi = \exp_{\bar{p}} \circ \, \tilde{\varphi} \circ \exp_p^{-1}
\]
Para facilitarlo, hagamos todo con $T_pS$, $T_{\bar{p}}\bar{S}$ y luego ``pasemos'' todo usando la exponencial.
Construyamos la siguiente isometría lineal:
\[
\tilde{\varphi}: T_pS \longrightarrow T_{\bar{p}}\bar{S}, \quad \{\vec{e}_1, \vec{e}_2\} \text{ base ortonormal de } T_pS \text{ y } \{\vec{f}_1, \vec{f}_2\} \text{ base ortonormal de } T_{\bar{p}}\bar{S}.
\]
\[
u\cdot\vec{e}_1 + v\cdot\vec{e}_2 \longmapsto u\cdot\vec{f}_1 + v\cdot\vec{f}_2
\]

Tenemos el siguiente esquema:

\begin{center}
    % ESPACIO PARA EL ESQUEMA GRANDE DE LOS PLANOS Y SUPERFICIES
    \includegraphics[width=0.8\textwidth]{Imagenes/Esquema-Minding.png}
    
    \vspace{0.3cm}
    \textit{\small Esquema de la construcción: Paso de los entornos en los planos tangentes (donde $\tilde{\varphi}$ es isometría) a las superficies mediante la exponencial.}
\end{center}

Definimos la aplicación $\varphi = \exp_{\bar{p}} \circ \, \tilde{\varphi} \circ \exp_p^{-1}$.
En los entornos contorneados se puede mover libremente (todo es biyectivo), por lo que se ve clara la biyectividad.

\textbf{1. Relación con las coordenadas geodésicas polares:}
Consideremos las parametrizaciones $X(r, \theta) = \exp_p(\phi(r, \theta))$ y $\bar{X}(r, \theta) = \exp_{\bar{p}}(\bar{\phi}(r, \theta))$.
Verifiquemos que el diagrama conmuta, es decir, $(\varphi \circ X)(r, \theta) = \bar{X}(r, \theta)$:
\[
\begin{aligned}
(\varphi \circ X)(r, \theta) &= (\varphi \circ \exp_p \circ \phi)(r, \theta) \\
&= (\exp_{\bar{p}} \circ \tilde{\varphi} \circ \underbrace{\exp_p^{-1} \circ \exp_p}_{Id} \circ \phi)(r, \theta) \\
&= (\exp_{\bar{p}} \circ \tilde{\varphi})(r\cos\theta\vec{e}_1 + r\sen\theta\vec{e}_2) \\
&= \exp_{\bar{p}}(r\cos\theta\vec{f}_1 + r\sen\theta\vec{f}_2) \\
&= \exp_{\bar{p}}(\bar{\phi}(r, \theta)) = \bar{X}(r, \theta) \quad \checkmark
\end{aligned}
\]

\textbf{2. Verificación de la isometría vía coeficientes métricos:}
Como el diagrama conmuta ($\bar{X} = \varphi \circ X$), basta ver que $E=\bar{E}$, $F=\bar{F}$ y $G=\bar{G}$:
\begin{itemize}
    \item Por ser coordenadas geodésicas polares: $E = \bar{E} = 1$ y $F = \bar{F} = 0$.
    \item Como $K = \bar{K}$ es constante, la resolución de la EDP $\sqrt{G}_{rr} + K\sqrt{G} = 0$ con las condiciones iniciales en el origen nos da la misma función $G(r, \theta) = \bar{G}(r, \theta)$ para ambos casos (según el signo de $K$).
\end{itemize}
Al coincidir los coeficientes métricos, $\varphi$ es una isometría.


\textbf{3. Cálculo de la diferencial en el origen:}
Probemos que $d\varphi_p = \tilde{\varphi}$. Usando la regla de la cadena:
\[
d\varphi_p = d(\exp_{\bar{p}} \circ \, \tilde{\varphi} \circ \exp_p^{-1})_p = d(\exp_{\bar{p}})_{\tilde{\varphi}(\exp_p^{-1}(p))} \circ d(\tilde{\varphi})_{\exp_p^{-1}(p)} \circ d(\exp_p^{-1})_p
\]
Como $\exp_p^{-1}(p) = \vec{0}$ y $\tilde{\varphi}(\vec{0}) = \vec{0}$, y sabiendo que $d(\exp)_{\vec{0}} = Id$ y $d(\tilde{\varphi}) = \tilde{\varphi}$ por ser lineal:
\[
d\varphi_p = d(\exp_{\bar{p}})_{\vec{0}} \circ \tilde{\varphi} \circ (d\exp_p)_{\vec{0}}^{-1} = Id \circ \tilde{\varphi} \circ Id = \tilde{\varphi} \quad \checkmark
\]

\textbf{4. Conclusión:}
Para nuestro caso, cogemos $X$ y $\bar{X}$ como las \textbf{coordenadas geodésicas polares} de cada superficie.

\begin{itemize}
    \item Se ve que conmutan (por construcción de $\varphi$ a través de las exponenciales y la isometría lineal).
    \item Se sabe que $E = \bar{E} = 1$ y $F = \bar{F} = 0$.
    \item Para ver que $G = \bar{G}$, se distinguen los 3 casos de la observación anterior para $K \equiv \text{cte}$. Como $K = \bar{K}$, las fórmulas darán el mismo resultado para $G$ y $\bar{G}$.
\end{itemize}

Sale todo y se tiene que $\varphi$ es isometría.

\end{proof}







       

    \chapter{Curvas Regulares a trozos y Variaciones}
    %% Archivo: capitulos/tema2.tex

% -------------------------------------------------------------------------
\section{Curvas regulares a trozos y Variaciones}

Para estudiar las propiedades minimizantes de las geodésicas (que son las curvas más cortas localmente), necesitamos 
ampliar el espacio de curvas admisibles para permitir esquinas.

\begin{definicion}{Curva regular a trozos}
Una curva regular a trozos es una aplicación continua 
$\alpha: [a,b] \longrightarrow \mathbb{R}^n$ para la cual existe una 
partición $a=t_0 < t_1 < \dots < t_k = b$ tal que, para cada $i=1, \dots, k$, 
la restricción $\alpha_i := \alpha|_{[t_{i-1}, t_i]}$ es una curva regular diferenciable.
\end{definicion}

\textbf{Notación y Vértices:}
Denotaremos los límites laterales de la derivada en los puntos de la partición como:
\[
\alpha'_{-}(t_i) = \lim_{t \to t_i^{-}} \alpha'(t)=\alpha'_i(t_i), \quad \alpha'_{+}(t_i) = \lim_{t \to t_i^{+}} \alpha'(t)=\alpha'_{i+1}(t_i).
\]
Diremos que $\alpha(t_i)$ es un \textbf{vértice} de $\alpha$ si $\alpha'_{-}(t_i) \neq \alpha'_{+}(t_i)$.
\\
\textit{Observación:} Toda curva regular a trozos puede reparametrizarse por el arco (p.p.a.). Y en $t=a$ y $t=b$ solo existe una derivada, excepto si la curva es cerrada (y podría haber un vértice o cerrarse regularmente).
\begin{figure}
    \centering
    \includegraphics[scale=0.65]{imagenes/CurvaRegularTrozos.png}    
\end{figure}


\subsection{El concepto de Variación}
\begin{definicion}{Variación de una curva}
Sean $\alpha : [a, b] \longrightarrow S$ una curva regular a trozos en una superficie regular $S$ y $a = s_0 < s_1 < \dots < s_k = b$ una partición de $[a, b]$ tal que, para cada $i = 1, \dots, k$, $\alpha_i := \alpha|_{[s_{i-1}, s_i]}$ es una curva regular.

Una \textbf{variación} de $\alpha$ es una aplicación continua $\phi : [a, b] \times (-\varepsilon, \varepsilon) \longrightarrow S$, que es diferenciable en cada uno de los rectángulos $[s_{i-1}, s_i] \times (-\varepsilon, \varepsilon)$, y tal que:
\begin{enumerate}[label=\roman*)]
    \item $\phi(s, 0) = \phi_0(s) = \alpha(s)$ para todo $s \in [a, b]$;
    \item para todo $t \in (-\varepsilon, \varepsilon)$, $\alpha_t = \phi(\cdot, t) : [a, b] \longrightarrow S$ es una curva parametrizada regular a trozos, llamada \textbf{curva de la variación}.
\end{enumerate}

Además, se dice que $\phi$ es una \textbf{variación propia} (o con \textbf{extremos fijos}) cuando $\alpha_t(a) = \phi(a, t) = \alpha(a)$ y $\alpha_t(b) = \phi(b, t) = \alpha(b)$ para todo $t \in (-\varepsilon, \varepsilon)$.
\end{definicion}



\begin{figure}
    \centering
    \includegraphics[scale=0.4]{imagenes/Variación de Una Cruva.png}
\end{figure}

\begin{definicion}{Curvas Transversales y de Variación}
Dada una variación $\phi$ de $\alpha$, se llama \textbf{curvas transversales} a las curvas $\beta_s : (-\varepsilon, \varepsilon) \longrightarrow S$ definidas por $\beta_s(t) = \phi(s, t)$, para cada $s \in [a, b]$.
Se llama \textbf{curvas de la variación} a las curvas $\alpha_t : [a, b] \longrightarrow S$ definidas por $\alpha_t(s) = \phi(s, t)$, para cada $t \in (-\varepsilon, \varepsilon)$.    
\end{definicion}



\begin{definicion}{Campo Variacional}
Sea $\phi$ una variación de una curva regular a trozos $\alpha$. Se define el \textbf{campo variacional} de $\phi$ como:
\[
\beta'(s)=Z(s) = \frac{\partial \phi}{\partial t}(s, 0) \in \mathfrak{X}(\alpha).
\]
\end{definicion}    


\textbf{Tipos de variaciones:}
\begin{itemize}
    \item \textbf{Variación propia (extremos fijos):} Si $\alpha_t(a) = \alpha(a)$ y $\alpha_t(b) = \alpha(b)$ para todo $t$. Esto implica que $Z(a)=0$ y $Z(b)=0$.
    \item \textbf{Variación normal:} Si el campo variacional es ortogonal a la curva, $\inner{Z(s)}{\alpha'(s)} = 0$.
\end{itemize}


% -------------------------------------------------------------------------
\section{Fórmulas de Variación de la Longitud}
Una variación $\phi$ de $\alpha$ genera una familia uniparamétrica $\{\alpha_t\}_{t\in (-\epsilon,\epsilon)}$ de curvas en la superficie, cuya longitud es el llamado \textbf{funcional longitud:}
\[
L(t) := L_{a}^{b}(\alpha_{t}) = \sum_{i=1}^{k} L_{s_{i-1}}^{s_{i}}(\alpha_{t}) = \sum_{i=1}^{k} \int_{s_{i-1}}^{s_{i}} |\alpha_{t}^{\prime}(s)| \, ds.
\]
Ponemos los sumatorios porque tenemos una curva regular a trozos, y cada tramo es regular, así que podemos calcular la longitud de cada tramo y sumarlas.



\begin{teorema}{Primera fórmula de variación}
Sea $\alpha: [0,l] \longrightarrow S$ una curva regular a trozos p.p.a. y sea $Z$ el campo variacional de una variación $\phi$. Entonces:
\[
L'(0) = \left[ \inner{Z(s)}{\alpha'(s)} \right]_0^l - \sum_{i=1}^{k-1} \inner{Z(s_i)}{\Delta_i \alpha'} - \int_0^l \inner{Z(s)}{\frac{D\alpha'}{ds}(s)} ds,
\]
donde $\Delta_i \alpha' = \alpha'_{+}(s_i) - \alpha'_{-}(s_i)$ es el "salto" de velocidad en los vértices.
\end{teorema}

Si la variación es \textbf{propia} (extremos fijos), los términos de frontera desaparecen:
\[
L'(0) = - \sum_{i=1}^{k-1} \inner{Z(s_i)}{\Delta_i \alpha'} - \int_0^l \inner{Z(s)}{\frac{D\alpha'}{ds}(s)} ds.
\]
\begin{observacion}{Fórmula de variación para las geodésicas}
    Observemos que si $\alpha$ es una geodésica, entonces $\frac{D\alpha'}{ds} \equiv 0$. Por tanto, para cualquier 
    variación propia de $\alpha$, se cumple que que $L'(0) = - \sum_{i=1}^{k-1} \inner{Z(s_i)}{\Delta_i \alpha'}$. En particular, si $\alpha$ no tiene vértices, 
    entonces $L'(0) = 0$ para toda variación propia de $\alpha$. (Por ser geodésica, no tendrá vértices porque es regular)
    Será cierto el recíproco? Es decir, si $L'(0) = 0$ para toda variación propia de $\alpha$, entonces $\alpha$ es una geodésica?
\end{observacion}

\begin{proof}[Demostración de la primera variación de la longitud de arco]
Sea $\alpha: [0, l] \to S$ una curva p.p.a. y regular a trozos, con vértices en $0 = s_0 < s_1 < \dots < s_k = l$. Sea $\phi(s,t)$ una variación de $\alpha$, tal que $\alpha_t(s) = \phi(s,t)$ y el campo variacional es $Z(s) = \frac{\partial \phi}{\partial t}(s,0)$.

Tomamos un intervalo donde la curva es regular, $[s_{i-1}, s_i]$. Calculamos la longitud de arco de la curva variada en ese tramo, que denotaremos como $L_i(t)$:
\[
L_i(t) = \int_{s_{i-1}}^{s_i} |\alpha'_t(s)| \, ds = \int_{s_{i-1}}^{s_i} \left| \frac{\partial \phi}{\partial s}(s,t) \right| ds = \int_{s_{i-1}}^{s_i} \left\langle \frac{\partial \phi}{\partial s}, \frac{\partial \phi}{\partial s} \right\rangle^{\frac{1}{2}} ds
\]

Derivemos $L_i(t)$ respecto del parámetro de variación $t$:
\[
L'_i(t) = \int_{s_{i-1}}^{s_i} \frac{2 \left\langle \frac{\partial^2 \phi}{\partial s \partial t}, \frac{\partial \phi}{\partial s} \right\rangle}{2 \underbrace{\left\langle \frac{\partial \phi}{\partial s}, \frac{\partial \phi}{\partial s} \right\rangle^{\frac{1}{2}}}_{|\alpha'_t(s)|}} \, ds
\]

Sustituyendo en $t = 0$, y recordando que $\alpha_0(s) = \alpha(s)$ está p.p.a. (por lo que $|\alpha'_0(s)| = 1$):
\[
L'_i(0) = \int_{s_{i-1}}^{s_i} \frac{\left\langle \frac{\partial^2 \phi}{\partial s \partial t}(s,0), \frac{\partial \phi}{\partial s} \right\rangle}{\underbrace{|\alpha'_0(s)|}_{= 1}} \, ds = \int_{s_{i-1}}^{s_i} \left\langle \frac{\partial^2 \phi}{\partial s \partial t}(s,0), \frac{\partial \phi}{\partial s}(s,0) \right\rangle ds
\]

Por la regla del producto para la derivada de un producto escalar, se tiene:
\[
\frac{d}{ds} \left\langle \frac{\partial \phi}{\partial t}, \frac{\partial \phi}{\partial s} \right\rangle = \left\langle \frac{\partial^2 \phi}{\partial t \partial s}, \frac{\partial \phi}{\partial s} \right\rangle + \left\langle \frac{\partial \phi}{\partial t}, \frac{\partial^2 \phi}{\partial s^2} \right\rangle
\]
Despejando el primer sumando del lado derecho (y usando el Teorema de Schwarz $\frac{\partial^2 \phi}{\partial t \partial s} = \frac{\partial^2 \phi}{\partial s \partial t}$) e integrando:

\begin{align*}
L'_i(0) &= \int_{s_{i-1}}^{s_i} \left[ \frac{d}{ds} \left\langle \underbrace{\frac{\partial \phi}{\partial t}(s,0)}_{Z(s)}, \underbrace{\frac{\partial \phi}{\partial s}(s,0)}_{\alpha'(s)} \right\rangle - \left\langle \underbrace{\frac{\partial \phi}{\partial t}(s,0)}_{Z(s)}, \underbrace{\frac{\partial^2 \phi}{\partial s^2}(s,0)}_{\alpha''(s)} \right\rangle \right] ds \\
&= \left[ \langle Z(s), \alpha'(s) \rangle \right]_{s_{i-1}}^{s_i} - \int_{s_{i-1}}^{s_i} \langle Z(s), \alpha''(s) \rangle \, ds
\end{align*}

Sabemos que $\alpha''(s) = \frac{D\alpha'}{ds} + (\alpha'')^\perp$. Al hacer el producto escalar $\langle Z(s), \alpha''(s) \rangle$, como el campo variacional $Z(s)$ es un campo tangente a la superficie, su producto con la componente normal $(\alpha'')^\perp$ se anula. Por tanto, nos quedamos únicamente con la componente tangente:
\[
L'_i(0) = \left[ \langle Z(s), \alpha'(s) \rangle \right]_{s_{i-1}}^{s_i} - \int_{s_{i-1}}^{s_i} \left\langle Z(s), \frac{D\alpha'}{ds}(s) \right\rangle ds
\]

Entonces, sumando para todos los tramos $i = 1, \dots, k$, obtenemos la variación total de la longitud $L'(0) = \sum_{i=1}^k L'_i(0)$:
\begin{align*}
L'(0) &= \sum_{i=1}^k \left( \langle Z(s_i), \alpha'_-(s_i) \rangle - \langle Z(s_{i-1}), \alpha'_+(s_{i-1}) \rangle \right) - \underbrace{\sum_{i=1}^k \int_{s_{i-1}}^{s_i} \left\langle Z, \frac{D\alpha'}{ds} \right\rangle ds}_{\int_0^l \langle Z, \frac{D\alpha'}{ds} \rangle \, ds}
\end{align*}
Donde $\alpha'_-(s_i)$ y $\alpha'_+(s_{i-1})$ denotan las velocidades por la izquierda y por la derecha en los respectivos vértices, ya que la curva es solo regular a trozos.

Falta desarrollar la primera parte (la suma telescópica con saltos). Extrayendo los términos de los extremos $0$ y $l$, y reordenando los índices:
\begin{align*}
\sum_{i=1}^k \big( \langle Z(s_i), &\alpha'_-(s_i) \rangle - \langle Z(s_{i-1}), \alpha'_+(s_{i-1}) \rangle \big) \\
&= \langle Z(l), \alpha'(l) \rangle + \sum_{i=1}^{k-1} \langle Z(s_i), \alpha'_-(s_i) \rangle - \langle Z(0), \alpha'(0) \rangle - \sum_{i=2}^k \langle Z(s_{i-1}), \alpha'_+(s_{i-1}) \rangle \\
\intertext{Haciendo un cambio de índice en el último sumatorio para que coincida con el anterior:}
&= \langle Z(l), \alpha'(l) \rangle - \langle Z(0), \alpha'(0) \rangle + \sum_{i=1}^{k-1} \langle Z(s_i), \alpha'_-(s_i) \rangle - \sum_{i=1}^{k-1} \langle Z(s_i), \alpha'_+(s_i) \rangle \\
&= \left[ \langle Z, \alpha' \rangle \right]_0^l - \sum_{i=1}^{k-1} \left\langle Z(s_i), \underbrace{\alpha'_+(s_i) - \alpha'_-(s_i)}_{\Delta_i \alpha'} \right\rangle
\end{align*}

Sustituyendo esto en la expresión general de $L'(0)$, concluimos la demostración:
\[
L'(0) = - \int_0^l \left\langle Z, \frac{D\alpha'}{ds} \right\rangle ds + \left[ \langle Z, \alpha' \rangle \right]_0^l - \sum_{i=1}^{k-1} \langle Z(s_i), \Delta_i \alpha' \rangle
\]
\end{proof}



\begin{teorema}{Caracterización variacional de las geodésicas}
Una curva regular a trozos p.p.a. $\alpha$ es un segmento de geodésica si, y solo si, $L'(0) = 0$ 
para toda variación propia de $\alpha$.
\end{teorema}

Esto implica que las geodésicas son los \textbf{puntos críticos} del funcional longitud.

\begin{proposicion}{Existencia de variaciones para un campo dado}
Sean $\alpha : [0, \ell] \longrightarrow S$ una curva regular a trozos p.p.a. en una superficie regular $S$ y $0 = s_0 < s_1 < \dots < s_k = \ell$ una partición de $[0, \ell]$ 
tal que $\alpha|_{[s_{i-1}, s_i]}$ es regular para cada $i = 1, \dots, k$. 
Sea $Z$ un campo de vectores tangente cualquiera a lo largo de $\alpha$, continuo en $[0, \ell]$ y diferenciable en cada subintervalo $[s_{i-1}, s_i]$.
\begin{enumerate}[label=\roman*)]
    \item Entonces existe una variación $\phi$ de $\alpha$ cuyo campo variacional es $Z$.
    \item Si $Z(0) = \mathbf{0}$ y $Z(\ell) = \mathbf{0}$, se puede elegir $\phi$ de forma que sea variación propia.
\end{enumerate}
\end{proposicion}
\begin{proof}[Demostración de la proposición]
    Lo vamos a demostrar en la hoja de problemas. 
\end{proof}


\begin{proof}[Demostración del teorema de caracterización variacional de las geodésicas]
La demostración consta de dos implicaciones.

\vspace{0.3cm}
\noindent $\Longrightarrow]$ \textbf{Supongamos que $\alpha$ es geodésica.} 

Por definición de geodésica, la curva $\alpha$ es regular en todo su dominio (no tiene vértices) y su aceleración geodésica es nula, es decir, $\frac{D\alpha'}{ds} = 0$. 
Además, como estamos considerando una variación propia $\phi$, los extremos de la curva permanecen fijos durante la deformación, lo que implica que el campo variacional se anula en los extremos: $Z(0) = Z(l) = 0$. Al no existir vértices, los términos de salto $\Delta_i \alpha'$ también son nulos.
Sustituyendo estas condiciones en la fórmula de la primera variación de la longitud (obtenida en la proposición anterior), resulta de manera directa que:
\[
L'(0) = - \int_0^l \left\langle Z, \underbrace{\frac{D\alpha'}{ds}}_{=0} \right\rangle ds + \underbrace{\langle Z(l), \alpha'(l) \rangle}_{=0} - \underbrace{\langle Z(0), \alpha'(0) \rangle}_{=0} - \sum_{i=1}^{k-1} \langle Z(s_i), \underbrace{\Delta_i \alpha'}_{=0} \rangle = 0.
\]

\vspace{0.3cm}
\noindent $\Longleftarrow]$ \textbf{Supongamos que $L'(0) = 0$ para toda variación propia.}

Como la variación es propia, los términos de frontera desaparecen ($Z(0)=Z(l)=0$), y nuestra hipótesis de partida se reduce a la ecuación:
\begin{equation} \label{eq:hipotesis_variacion}
0 = L'(0) = - \sum_{i=1}^{k-1} \langle Z(s_i), \Delta_i \alpha' \rangle - \int_0^l \left\langle Z, \frac{D\alpha'}{ds} \right\rangle ds
\end{equation}
para cualquier campo variacional $Z$ a lo largo de $\alpha$ que se anule en los extremos. Y esto se tiene que cumplir para toda variación propia. 

La demostración de que $\alpha$ es geodésica la dividiremos en dos pasos:

\textbf{Paso 1: Demostrar que $\alpha$ es geodésica a trozos.} \\
Fijemos un intervalo de regularidad cualquiera $(s_{i-1}, s_i)$. Sea $f: [0, l] \to \mathbb{R}$ una función diferenciable tal que $f(s) > 0$ 
para todo $s \in (s_{i-1}, s_i)$ y $f(s) = 0$ en el resto del dominio $[0, l]$.

Construimos el siguiente campo vectorial a lo largo de $\alpha$:
\[
Z(s) = f(s) \frac{D\alpha'}{ds}(s) \in \mathfrak{X}(\alpha)
\]
Por propiedades conocidas de las variaciones, existe una variación propia $\phi$ de $\alpha$ que tiene a este $Z$ como campo variacional asociado. Evaluamos nuestra hipótesis \eqref{eq:hipotesis_variacion} para este campo $Z$ en particular:
\[
0 = - \sum_{j=1}^{k-1} \langle \underbrace{Z(s_j)}_{=0}, \Delta_j \alpha' \rangle - \int_0^l \left\langle f(s) \frac{D\alpha'}{ds}, \frac{D\alpha'}{ds} \right\rangle ds
\]
La sumatoria se anula porque la función $f$ vale cero en todos los vértices $s_j$. La integral, al anularse el integrando fuera de $(s_{i-1}, s_i)$, se reduce a:
\[
0 = - \int_{s_{i-1}}^{s_i} f(s) \left| \frac{D\alpha'}{ds} \right|^2 ds
\]
Dado que $f(s) > 0$ en el intervalo de integración y la norma al cuadrado es siempre no negativa ($\left| \frac{D\alpha'}{ds} \right|^2 \ge 0$), la única forma de que la integral de una función no negativa sea cero es que el integrando sea idénticamente nulo casi por todas partes. Al ser las funciones continuas, deducimos que:
\[
\frac{D\alpha'}{ds} = 0 \quad \text{en } (s_{i-1}, s_i).
\]
Como este razonamiento es válido para cualquier subintervalo, concluimos que $\alpha$ es una geodésica a trozos.

\vspace{0.3cm}
\textbf{Paso 2: Demostrar que no existen vértices.} \\
Ahora debemos probar que $\alpha$ es globalmente regular, es decir, que no hay cambios bruscos de dirección en los vértices: $\alpha'_+(s_i) = \alpha'_-(s_i)$, o equivalentemente, $\Delta_i \alpha' = 0$ para todo $i$.

Fijemos un vértice arbitrario $s_i$. Mediante el uso de particiones de la unidad y transporte paralelo, es posible construir un campo vectorial diferenciable $Z \in \mathfrak{X}(\alpha)$ tal que en dicho vértice tome exactamente el valor del salto, $Z(s_i) = \Delta_i \alpha'$, y que se anule en todos los demás vértices, $Z(s_j) = 0$ para todo $j \neq i$, así como en los extremos de la curva.
Tomando la variación propia asociada a este nuevo campo $Z$ y aplicando de nuevo la hipótesis \eqref{eq:hipotesis_variacion}:
\[
0 = - \sum_{j=1}^{k-1} \langle Z(s_j), \Delta_j \alpha' \rangle - \int_0^l \left\langle Z, \underbrace{\frac{D\alpha'}{ds}}_{=0} \right\rangle ds
\]
La integral se anula completamente porque, según demostramos en el Paso 1, $\frac{D\alpha'}{ds} = 0$ en el interior de cada subintervalo. La sumatoria colapsa a un único término, correspondiente al vértice $s_i$ que hemos fijado:
\[
0 = - \langle Z(s_i), \Delta_i \alpha' \rangle = - \langle \Delta_i \alpha', \Delta_i \alpha' \rangle = - |\Delta_i \alpha'|^2
\]
Esto implica que $|\Delta_i \alpha'|^2 = 0$, de donde se deduce ineludiblemente que $\Delta_i \alpha' = 0$.

Al ser nulo el salto en cada vértice, la derivada $\alpha'$ es continua en todo el dominio $[0, l]$. Por lo tanto, la curva está formada por un único "trozo" regular. Queda así demostrado que $\alpha$ es, en efecto, una geodésica global.
\end{proof}

\begin{definicion}{Variación normal}
Sea $\alpha : [0, \ell] \longrightarrow S$ una curva regular a trozos en una superficie regular $S$. Una variación de $\alpha$ se dice \textbf{normal} si su campo variacional $Z$ es normal a $\alpha$, esto es, $\langle Z(s), \alpha'(s) \rangle = 0$.
\end{definicion}

\begin{corolario}{}
Sea $\alpha: [a,b] \longrightarrow S$ una curva regular p.p.a. en una superficie regular $S$. Entonces $\alpha$ es un segmento de geodésica de $S$ si, y solo si, $L'(0) = 0$ para toda variación \textbf{\textcolor{red}{normal}} $\phi$ de la curva $\alpha$.
\end{corolario}

\begin{observacion}{Justificación del uso de variaciones normales}
En el teorema general de variaciones se habla de cualquier variación, pero aquí se puede exigir únicamente que sea una variación \textbf{normal}. La justificación matemática (como indican las notas manuscritas) es la siguiente:

Si para la demostración construimos el campo variacional ortogonal clásico:
\[
Z(s) = f(s) \frac{D\alpha'}{ds}
\]
podemos asegurar que este campo \textbf{es un campo normal} a la curva $\alpha$.

\textbf{Demostración de este hecho:}
Como la curva $\alpha$ está parametrizada por longitud de arco (p.p.a.), su vector velocidad tiene norma constante igual a 1. Es decir:
\[
\langle \alpha', \alpha' \rangle = 1
\]
Si derivamos esta expresión a lo largo de la curva (usando la derivada covariante):
\[
\frac{d}{ds} \langle \alpha', \alpha' \rangle = 2 \left\langle \alpha', \frac{D\alpha'}{ds} \right\rangle = 0 \implies \left\langle \alpha', \frac{D\alpha'}{ds} \right\rangle = 0
\]
Multiplicando por la función escalar $f(s)$:
\[
\left\langle \alpha', f(s) \frac{D\alpha'}{ds} \right\rangle = 0 \implies \langle \alpha', Z(s) \rangle = 0
\]

\textbf{Conclusión:}
Como el producto escalar del campo $Z$ con el vector tangente $\alpha'$ es cero, significa que \textit{``la parte tangencial se va''}. El campo $Z$ no tiene componente en la dirección de la curva, por lo que es un campo estrictamente normal, y la variación asociada $\phi$ será una variación normal.
\end{observacion}

\begin{teorema}{Segunda fórmula de variación}
Sea $\gamma: [0,l] \longrightarrow S$ una geodésica p.p.a. (por tanto, $L'(0)=0$). Para una variación propia y normal con campo $Z$, la segunda derivada de la longitud es:
\[
L''(0) = \int_0^l \left[ \left| \frac{DZ}{ds} \right|^2 - K(\gamma(s)) |Z(s)|^2 \right] ds.
\]
\end{teorema}
\textit{Nota:} Aquí aparece explícitamente la curvatura de Gauss $K$. Si $K < 0$, entonces $L''(0) > 0$, lo que sugiere que en superficies de curvatura negativa las geodésicas minimizan la longitud (son mínimos locales estables).

% -------------------------------------------------------------------------
% --- SECCIÓN: INTEGRACIÓN EN SUPERFICIES ---

\section{Integración en superficies}

\begin{definicion}{Elemento de área}
Sea $S \subset \mathbb{R}^{3}$ una superficie regular y orientada, con aplicación de Gauss $N$. Se denomina \textbf{\textcolor{mainred}{elemento de área}} de $S$ en un punto $p \in S$ a la aplicación $dA(p) : T_{p}S \times T_{p}S \longrightarrow \mathbb{R}$ dada por $dA(p)(\mathbf{v}, \mathbf{w}) = \det(\mathbf{v}, \mathbf{w}, N(p))$.
\end{definicion}

\begin{observacion}{MUCHO OJO}
    No confundir con una diferencial o con el operador forma (Operador Weingarten).\\
    $dA(p)$ es una forma bilineal antisimétrica.\\
    $dA(p)(X_{u}(q), X_{v}(q)) = \sqrt{EG - F^{2}}(q)$.
\end{observacion}

\begin{proof}[Demostración de la tercera observación]
Sea $S$ una superficie regular orientada y sea $X: U \subset \mathbb{R}^2 \longrightarrow S$ una parametrización de $S$ en un entorno del punto $p = X(q)$, donde $q = (u,v) \in U$.

Por definición topológica y algebraica, el elemento de área $dA$ (la 2-forma de área asociada a la métrica y a la orientación de $S$) evaluado en un par de vectores tangentes $\vec{w}_1, \vec{w}_2 \in T_pS$ devuelve el volumen con signo del paralelepípedo formado por $\vec{w}_1, \vec{w}_2$ y el vector normal unitario $N(p)$. Es decir, se define mediante el producto mixto:
$$dA(p)(\vec{w}_1, \vec{w}_2) = \langle \vec{w}_1 \times \vec{w}_2, N(p) \rangle$$

Para demostrar la observación, evaluamos esta 2-forma en los vectores de la base coordenada del plano tangente, $\{X_u(q), X_v(q)\}$:
$$dA(p)(X_u(q), X_v(q)) = \langle X_u(q) \times X_v(q), N(p) \rangle$$

Sabemos que el campo normal unitario $N(p)$ que orienta la superficie de forma compatible con la parametrización está dado precisamente por el producto vectorial normalizado de la base coordenada:
$$N(p) = \frac{X_u(q) \times X_v(q)}{|X_u(q) \times X_v(q)|}$$

Sustituyendo la expresión de $N(p)$ en nuestra evaluación de la 2-forma:
$$dA(p)(X_u(q), X_v(q)) = \left\langle X_u(q) \times X_v(q), \frac{X_u(q) \times X_v(q)}{|X_u(q) \times X_v(q)|} \right\rangle$$

Por la bilinealidad (o linealidad en este caso) del producto escalar, podemos extraer el escalar del denominador:
$$dA(p)(X_u(q), X_v(q)) = \frac{1}{|X_u(q) \times X_v(q)|} \langle X_u(q) \times X_v(q), X_u(q) \times X_v(q) \rangle$$

Recordando que $\langle \vec{v}, \vec{v} \rangle = |\vec{v}|^2$ para cualquier vector en $\mathbb{R}^3$, el numerador se convierte en el cuadrado de la norma:
$$dA(p)(X_u(q), X_v(q)) = \frac{|X_u(q) \times X_v(q)|^2}{|X_u(q) \times X_v(q)|} = |X_u(q) \times X_v(q)|$$

Físicamente y geométricamente, esto corrobora que el valor de la 2-forma sobre la base es el área del paralelogramo tangente que expanden $X_u$ y $X_v$. Ahora, para conectarlo con la geometría intrínseca de la superficie, utilizamos la \textbf{Identidad de Lagrange}:
$$|\vec{a} \times \vec{b}|^2 = |\vec{a}|^2 |\vec{b}|^2 - \langle \vec{a}, \vec{b} \rangle^2$$

Aplicando esta identidad de álgebra vectorial a nuestros vectores tangentes:
$$|X_u(q) \times X_v(q)|^2 = |X_u(q)|^2 |X_v(q)|^2 - \langle X_u(q), X_v(q) \rangle^2$$

A continuación, recordamos la definición de los coeficientes de la Primera Forma Fundamental:
\begin{itemize}
    \item $E(q) = \langle X_u(q), X_u(q) \rangle = |X_u(q)|^2$
    \item $F(q) = \langle X_u(q), X_v(q) \rangle$
    \item $G(q) = \langle X_v(q), X_v(q) \rangle = |X_v(q)|^2$
\end{itemize}

Sustituyendo directamente estos coeficientes métricos en la identidad de Lagrange desarrollada:
$$|X_u(q) \times X_v(q)|^2 = E(q)G(q) - (F(q))^2$$

Tomando la raíz cuadrada positiva (puesto que la norma representa un área geométrica real, siempre no negativa):
$$|X_u(q) \times X_v(q)| = \sqrt{EG - F^2}(q)$$

Por lo tanto, enlazando el principio con el final del desarrollo, queda formalmente demostrado que:
$$dA(p)(X_u(q), X_v(q)) = \sqrt{EG - F^2}(q)$$
\end{proof}

\begin{observacion}{Relación de áreas mediante la diferencial}
Con la diferencial "pasamos" del plano a la superficie:



Si tomamos un pequeño rectángulo $R$ de lados $\Delta u, \Delta v$ en el dominio de los parámetros a partir de un punto $q$, la diferencial linealiza esta transformación. Se tiene para la dirección $u$:
\[
dX_q((\Delta u, 0)) = \Delta u \underbrace{dX_q(1,0)}_{X_u}
\]

El área del paralelogramo $R'$ formado en el plano tangente $T_pS$ por la imagen de los lados del rectángulo se calcula mediante la norma del producto vectorial:
\[
A(R') = |dX_q(\Delta u, 0) \times dX_q(0, \Delta v)| = \Delta u \Delta v |X_u \times X_v| = \underbrace{\Delta u \Delta v}_{A(R)} \sqrt{EG-F^2}
\]

Al pasar al límite (haciendo que el rectángulo sea infinitamente pequeño), el área del paralelogramo tangente $A(R')$ y el área de la región curva real sobre la superficie $A(\tilde{R})$ se igualan, obteniéndose el siguiente límite:
\[
\lim_{\Delta u, \Delta v \to 0} \frac{A(\tilde{R})}{A(R)} = \sqrt{EG-F^2}
\]

\textbf{Conclusión:}
Las áreas entre el plano de parámetros y la superficie están relacionadas (distorsionadas) localmente mediante el factor $\sqrt{EG-F^2}$. Esto es lo que justifica que, al integrar, el elemento diferencial de área sea $dA = \sqrt{EG-F^2} \, du \, dv$.
\end{observacion}

\begin{center}
    \includegraphics[width=0.7\textwidth]{Imagenes/AreaSuperficies.png}
\end{center}

%%%% AREA DE UNA REGION %%%

\begin{definicion}{Área de una región}
El área de una región $R \subset S$ viene dada por la expresión
\[
A(R) = \iint_{X^{-1}(R)} \sqrt{EG - F^{2}} \, du \, dv,
\]
donde $(U, X)$ es cualquier parametrización tal que $X(U)$ contiene a $R$.
\end{definicion}

\begin{definicion}{Soporte de una función}
Sea $f : S \longrightarrow \mathbb{R}$ una función real definida sobre una superficie regular orientada $S$. El \textbf{\textcolor{mainred}{soporte}} (compacto) de $f$ es el conjunto
\[
\text{sop}(f) = \text{cl}\{p \in S : f(p) \neq 0\}.
\]
\end{definicion}




\begin{observacion}{El problema de integrar funciones sobre superficies}
Si se tiene una función escalar $f: S \longrightarrow \mathbb{R}$ definida sobre una superficie, nos planteamos cómo calcular su integral:
\[
\int_S f \, dA \quad \text{?}
\]

``Nos vamos'' al plano utilizando una parametrización local $X: \mathcal{U} \subset \mathbb{R}^2 \longrightarrow S$. Podríamos intentar definir la integral simplemente arrastrando la función:
\[
\int_S f \, dA \overset{?}{=} \iint_{\mathcal{U}} (f \circ X)(u,v) \, du \, dv
\]
Se puede hacer esto, pero... \textbf{¿es una buena definición?} Es decir, ¿depende de la parametrización?

\textbf{¡Pues es una castaña de definición!} ¿Por qué?

Si tenemos dos cartas distintas:
Sea $\phi = \bar{X}^{-1} \circ X : \mathcal{U} \longrightarrow \bar{\mathcal{U}}$ el cambio de coordenadas, tal que $(\bar{u}, \bar{v}) = \phi(u,v)$. 

Aplicando el Teorema del Cambio de Variable para integrales dobles, se tiene:
\[
\iint_{\bar{\mathcal{U}}} (f \circ \bar{X})(\bar{u}, \bar{v}) \, d\bar{u} \, d\bar{v} = \iint_{\phi^{-1}(\bar{\mathcal{U}})} (f \circ \bar{X})(\phi(u,v)) \, |J\phi| \, du \, dv
\]

Sabiendo que $\phi^{-1}(\bar{\mathcal{U}}) = \mathcal{U}$ y que la composición es $(f \circ \bar{X}) \circ \phi = f \circ (\bar{X} \circ \bar{X}^{-1} \circ X) = f \circ X$, sustituimos y nos queda:
\[
\iint_{\bar{\mathcal{U}}} (f \circ \bar{X})(\bar{u}, \bar{v}) \, d\bar{u} \, d\bar{v} = \iint_{\mathcal{U}} \underbrace{(f \circ \bar{X})(\phi(u,v))}_{f \circ X} \underbrace{|J\phi|}_{\substack{\text{pero esto se} \\ \text{queda ahí}}} \, du \, dv \neq \iint_{\mathcal{U}} (f \circ X)(u,v) \, du \, dv
\]

Entonces \textbf{no es una buena definición}, debe aparecer el elemento de área por algún lado, luego...
\end{observacion}

\begin{definicion}{Integral de $f$ en $S$ (Caso Local)}
Sea $f : S \longrightarrow \mathbb{R}$ una función con soporte compacto, definida sobre una superficie regular orientada $S$, tal que existe una parametrización $(U, X)$ de $S$ de forma que $\text{sop}(f) \subset X(U)$. Se define la \textbf{\textcolor{mainred}{integral de $f$ en $S$}} como
\[
\int_{S} f \, dA := \iint_{U} (f \circ X)(u, v) \sqrt{EG - F^{2}}(u, v) \, du \, dv
\]
(siempre y cuando la integral sobre $U \subset \mathbb{R}^{2}$ esté definida).
\end{definicion}
Hay que ver que esta definición no depende de la parametrización. Para ello, basta con aplicar el Teorema del Cambio de Variable y la fórmula de transformación de los coeficientes métricos $E, F, G$ bajo cambios de coordenadas. Lo veremos en la hoja de problemas.
Ahora bien, \textbf{¿Y si el soporte de $f$ no está contenido en un entorno coordenado?}

\begin{definicion}{Partición de la unidad}
Una \textbf{\textcolor{mainred}{partición}} (diferenciable) \textbf{\textcolor{mainred}{de la unidad}} en una superficie regular $S$ es una colección de funciones (diferenciables) $f_{i} : S \longrightarrow \mathbb{R}$, $i = 1, \dots, n$,
 tales que $0 \le f_{i} \le 1$, $i = 1, \dots, n$ y $\sum_{i=1}^{n} f_{i} \equiv 1$. 
 La partición $\{f_{i}\}_{i=1}^{n}$ está \textbf{subordinada a un cubrimiento abierto} de $S$, $\{V_{1}, \dots, V_{n}\}$, si $\text{sop}(f_{i}) \subset V_{i}$, $i = 1, \dots, n$.\\
Recordemos que un \textbf{cubrimiento} de una superficie es coger un  montón de parametrizaciones de forma que al final los 
entornos coordenados (imagen de cada una de nuestras parametrizaciones) me lo cubran todo.
\end{definicion}

\begin{observacion}{Particiones de la unidad y soporte de funciones}
\begin{itemize}
    \item Dado un cubrimiento finito $\{V_1, \ldots, V_n\}$ por abiertos de $S$, siempre existe una partición (diferenciable) de la unidad subordinada a éste.
    \item Sea $\{g, g_1, \ldots, g_n\}$ una partición de la unidad subordinada al cubrimiento $\{S \setminus \text{sop}(f), X_1(U_1), \ldots, X_n(U_n)\}$, y sea $f_i := f g_i$. Entonces $f = \sum_{i=1}^n f_i$ y $\text{sop}(f_i) \subset X_i(U_i)$ para todo $i = 1, \ldots, n$.
\end{itemize}

\vspace{0.3cm}
\textbf{Justificación (Desarrollo lógico de los soportes):}

Por definición de partición de la unidad subordinada al cubrimiento dado, sabemos dónde se localiza el soporte de cada función de la partición:
\begin{itemize}
    \item Soporte de $g_1$ en $X_1(U_1)$.
    \item Soporte de $g_2$ en $X_2(U_2)$.
    \item $\vdots$
    \item Soporte de $g$ en $S \setminus \text{sop}(f)$.
\end{itemize}

Dado que las funciones de la partición de la unidad suman $1$ en todo punto, podemos reescribir la función $f$ en un punto $p$ de la siguiente manera:
\[
f(p) = f(p) \cdot 1 = f(p) \left( g(p) + \sum g_i(p) \right) = \underbrace{f(p)g(p)}_{0} + \sum \underbrace{f \cdot g_i}_{f_i}(p) = \sum f_i(p)
\]

\textbf{¿Por qué el término $f(p) \cdot g(p) = 0$?}
\begin{itemize}
    \item Si $f(p) \neq 0 \implies p \in \text{sop}(f)$. Pero por construcción, sabemos que $\text{sop}(g) \subset S \setminus \text{sop}(f)$, lo que implica que $p$ está fuera del soporte de $g$, y por tanto $g(p) = 0$.
    \item (Si $f(p) = 0$ ya está, el producto es trivialmente cero).
\end{itemize}

Finalmente, para ver que cada $f_i$ se queda dentro de su respectivo entorno coordenado, analizamos su soporte:
Y se tiene: 
\[
\text{sop}(f_i) = \text{sop}(f \cdot g_i) = (\text{sop}(f)) \cap (\text{sop}(g_i)) \subset X_i(U_i) \quad \square
\]
\end{observacion}

\begin{definicion}{Integral de $f$ en $S$ (Caso Global)}
Se define la \textbf{\textcolor{mainred}{integral de $f$ en $S$}} como
\[
\int_{S} f \, dA := \sum_{i=1}^{n} \int_{S} f_{i} \, dA.
\]
\end{definicion}

% --- OBSERVACIONES SOBRE EL CÁLCULO DE ÁREAS ---

\begin{tcolorbox}[colback=maingreen!5!white, colframe=maingreen!75!black, title=Observaciones sobre la extensión del área]
Es importante notar que, por definición, no podemos calcular directamente el área de superficies enteras que no estén acotadas (como el cilindro o el cono) utilizando una sola región $R$. 

\begin{itemize}
    \item \textbf{Superficies compactas:} En estos casos, las superficies son regiones $R$ en sí mismas, lo que facilita el cálculo global del área.
    \item \textbf{Casos especiales:} Existen superficies no acotadas, como la \textbf{pseudoesfera}, donde el cálculo del área sí es posible, aunque constituye un caso excepcional.
    \item \textbf{Estrategia de cálculo:} Lo ideal es emplear parametrizaciones que cubran la superficie salvo un conjunto de \textbf{medida nula}. De esta forma, evitamos la necesidad de recurrir explícitamente a particiones de la unidad muy complejas.
\end{itemize}
\end{tcolorbox}

% --- EJEMPLO: ÁREA DE LA ESFERA ---

\begin{ejemplo}{Cálculo del área de la esfera $\mathbb{S}^2$}
Queremos calcular el área total de la esfera de radio $r$. Para ello, integramos la función constante $f \equiv 1$ sobre toda la superficie:
\[
A(\mathbb{S}^2) = \int_{\mathbb{S}^2} 1 \, dA.
\]
Como $sop(1) = \mathbb{S}^2$, necesitaríamos una parametrización que cubra toda la esfera. Dado que esto es topológicamente imposible con una sola carta, utilizamos una parametrización que deje fuera únicamente un conjunto de medida nula (los meridianos/polos):

Sea la parametrización en coordenadas geográficas:
\[
X(\theta, \varphi) = (r \cos\theta \sin\varphi, \, r \sin\theta \sin\varphi, \, r \cos\varphi)
\]
con $(\theta, \varphi) \in (0, 2\pi) \times (0, \pi)$. 

Calculando los coeficientes de la primera forma fundamental, obtenemos el elemento de área:
\[
\sqrt{EG - F^2} = r^2 \sin\varphi.
\]
*(Nota: En tus apuntes aparece $\sin\theta$, dependiendo de la asignación de ángulos en la parametrización)*.

Finalmente, el área resulta:
\[
A(\mathbb{S}^2) = \int_{0}^{\pi} \int_{0}^{2\pi} r^2 \sin\varphi \, d\theta \, d\varphi = 4\pi r^2.
\]
\end{ejemplo}

\begin{teorema}{Teorema del cambio de variable}
Sean $S_{1}$ y $S_{2}$ dos superficies regulares, conexas y orientadas, cuyos elementos de área representamos por $dA_{1}$ y $dA_{2}$, respectivamente. Sea $f : S_{2} \longrightarrow \mathbb{R}$ una función con soporte compacto. Si $\phi : S_{1} \longrightarrow S_{2}$ es un difeomorfismo, entonces
\[
\int_{S_{2}} f \, dA_{2} = \int_{S_{1}} (f \circ \phi) \, |\det(d\phi)| \, dA_{1}.
\]
\end{teorema}
\begin{proof}
    No hay demostración
\end{proof}
\begin{proposicion}{Corolario}
Sea $S$ una superficie regular orientada, $N$ su aplicación de Gauss y $p \in S$ con $K(p) \neq 0$. Si $V \subset S$ es un entorno de $p$ donde $N|_{V} : V \longrightarrow N(V)$ es un difeomorfismo, entonces
\[
A(N(V)) = \int_{V} |K| \, dA,
\]
siempre y cuando esta integral tome un valor finito.
\end{proposicion}

\textit{Observación: Si $K(p) \neq 0$, siempre puedo encontrar el entorno de $p$ donde tengo el difeomorfismo.}


% --- SECCIÓN: VARIACIONES DEL ÁREA ---

\section{Variaciones del área. Las superficies minimales}

\begin{definicion}{Variación normal de una superficie}
Sean $U \subset \mathbb{R}^{2}$ un abierto y $X : U \longrightarrow \mathbb{R}^{3}$ una aplicación tales que $X(U)$ es una superficie regular. Si $D \subset \mathbb{R}^{2}$ es un disco abierto cuya clausura $\text{cl } D \subset U$, entonces $X(D)$ es una superficie regular para la cual $X|_{D}$ es una parametrización.

Sea $h : \text{cl } D \longrightarrow \mathbb{R}$ una función diferenciable que se anula en la frontera de $D$. Se llama \textbf{\textcolor{mainred}{variación normal}} de $X(D)$ determinada por $h$ a la aplicación $\phi : \text{cl } D \times (-\varepsilon, \varepsilon) \longrightarrow \mathbb{R}^{3}$ definida por
\[
\phi(u, v, t) = X(u, v) + t h(u, v) N(X(u, v)),
\]
donde $N$ es el vector normal unitario y $\varepsilon > 0$ es suficientemente pequeño para que la aplicación $X^{t} : D \longrightarrow \mathbb{R}^{3}$ dada por $X^{t}(u, v) = \phi(u, v, t)$ sea una parametrización de la superficie regular $X^{t}(D)$ para todo $t \in (-\varepsilon, \varepsilon)$.
\end{definicion}


Una variación $\phi$ de $X(D)$ genera una familia uniparamétrica $\{X^{t}(D)\}_{t \in (-\varepsilon, \varepsilon)}$ de superficies regulares. El área de estas superficies da lugar al llamado \textbf{\textcolor{mainred}{funcional área}}:
\[
A(t) = A(X^{t}(D)).
\]

\begin{observacion}{}
Todo esto realmente es sencillo. 
Cogemos un disco $D$ (por comodidad) con la clausura del disco contenida en el abierto $\mathcal{U}$ (es decir, $\bar{D} \subset \mathcal{U}$). 
Se tiene que $\mathbf{X}(D)$ es una superficie regular y la restricción $\mathbf{X}|_D$ es una parametrización. Le creamos una \textbf{variación normal} a esta superficie $\mathbf{X}(D)$.

\textbf{Analogía geométrica:} Es como una cama elástica, cada ``posición'' o deformación de la tela sería una variación normal de la superficie.

Esa pequeña variación nos la da el parámetro $t \in (-\varepsilon, \varepsilon)$ (se varía estrictamente en la dirección del vector normal). 

Y todas esas parametrizaciones deformadas nos dan una \textbf{familia} de superficies.
\end{observacion}

\begin{center}
    \includegraphics[width=0.7\textwidth]{Imagenes/VariacionNormalSup.png}
\end{center}

% --- DEFINICIONES DE SUPERFICIES MINIMALES ---

\begin{definicion}{Caracterización Variacional (I)}
Una superficie regular $S$ es \textbf{\textcolor{mainred}{minimal}} si, para cada punto de la misma, existe un entorno $\Omega$ tal que éste es la superficie de menor área entre todas aquellas que tienen como frontera la frontera de $\Omega$.
\end{definicion}

\begin{definicion}{Ecuación en Derivadas Parciales (II)}
Una superficie es \textbf{\textcolor{mainred}{minimal}} si la función diferenciable $f$ que determina su gráfica verifica la \textbf{ecuación de Euler-Lagrange}:
\[
f_{xx}(1+f_{y}^{2}) - 2f_{xy}f_{x}f_{y} + f_{yy}(1+f_{x}^{2}) = 0.
\]
\end{definicion}

\begin{definicion}{Curvatura Media (III)}
Una superficie es \textbf{\textcolor{mainred}{minimal}} si su \textbf{curvatura media} se anula en todo punto, esto es, si $H \equiv 0$.
\end{definicion}

\begin{definicion}{Aplicación de Gauss (IV)}
Se llama \textbf{\textcolor{mainred}{superficie minimal}} a aquella cuya aplicación de Gauss es \textbf{conforme}\footnote{conserva ángulos.} y no es una esfera.
\end{definicion}

\begin{definicion}{Parametrización Isoterma (V)}
La superficie determinada por una parametrización \textbf{\textcolor{mainred}{isoterma}} $X(u,v) = (x(u,v), y(u,v), z(u,v))$ (es decir, tal que $E=G$ y $F=0$) es \textbf{\textcolor{mainred}{minimal}} si las funciones coordenadas son \textbf{armónicas}, esto es, si:
\[
\Delta x = \Delta y = \Delta z = 0.
\]
\end{definicion}

\begin{definicion}{Membrana Jabonosa (VI)}
Se denomina \textbf{\textcolor{mainred}{superficie minimal}} aquella que se puede reproducir mediante una \textbf{membrana jabonosa}.
\end{definicion}

















\begin{ejemplo}{Ejemplos clásicos}
\begin{itemize}
    \item \textbf{El Plano:} La solución trivial ($K=H=0$).
    \item \textbf{El Catenoide:} Generada al rotar una catenaria. (Euler, 1740: única minimal de revolución).
    \item \textbf{El Helicoide:} Superficie reglada generada por una recta que rota y sube (escalera de caracol). (Meusnier, 1770).
    \item \textbf{Superficies de Scherk:} Definidas implícitamente, por ejemplo, $e^z \cos x - \cos y = 0$.
\end{itemize}
Es interesante notar que el helicoide puede deformarse isométricamente en un catenoide pasando por una familia continua de superficies minimales.
\end{ejemplo}
\subsubsection{Superficies minimales como puntos críticos del área}

\begin{proposicion}{Existencia de funciones meseta}
Sean $D$ un disco abierto de $\mathbb{R}^2$ y $q \in D$. Entonces, para todo $\varepsilon > 0$ tal que $D(q, \varepsilon) \subset D$, existe una función diferenciable $\tilde{h} : D \longrightarrow \mathbb{R}$, llamada \textbf{\textcolor{mainred}{función meseta}}, verificando que:
\begin{itemize}
    \item $\text{sop}(\tilde{h}) \subset D(q, \varepsilon)$,
    \item $0 \le \tilde{h} \le 1$,
    \item $\tilde{h} \equiv 1$ en el disco $D(q, \varepsilon/2)$.
\end{itemize}
\end{proposicion}

\begin{teorema}{Superficies minimales como puntos críticos del área}
Sean $U \subset \mathbb{R}^2$ un abierto y $X : U \longrightarrow \mathbb{R}^3$ una aplicación tal que $X(U)$ es una superficie regular. Sea $D$ un disco abierto en $\mathbb{R}^2$ cuya clausura $\text{cl } D \subset U$. Entonces, $X(D)$ es una superficie minimal si, y solo si, $A'(0) = 0$ para cualquier variación normal $\phi$ de $X(D)$.
\end{teorema}


\begin{proof}
Dada la variación normal $\phi(u, v, t) = X(u, v) + t h(u, v) N(u, v)$, el funcional área es:
\[
A(t) = \iint_{D} \sqrt{E_t G_t - F_t^2} \, du \, dv.
\]
Para calcular $A'(0)$, primero aproximamos los coeficientes de la primera forma fundamental de la variación $X^t$. Tras omitir términos de orden $t^2$ que desaparecerán al derivar y evaluar en $t=0$, obtenemos:
\begin{align*}
E_t &= E - 2th e + t^2 R_1, \\
F_t &= F - 2th f + t^2 R_2, \\
G_t &= G - 2th g + t^2 R_3.
\end{align*}
Sustituyendo en el discriminante y utilizando la definición de curvatura media $H = \frac{gE + eG - 2fF}{2(EG - F^2)}$, se llega a:
\[
\sqrt{E_t G_t - F_t^2} = \sqrt{EG - F^2} \sqrt{1 - 4thH + t^2 \tilde{R}}.
\]
Derivando bajo el signo integral y evaluando en $t=0$:
\[
A'(0) = -2 \iint_{D} h H \sqrt{EG - F^2} \, du \, dv = -2 \int_{X(D)} h H \, dA.
\]

\textbf{($\implies$)} Si $X(D)$ es minimal, $H \equiv 0$, por lo que $A'(0) = 0$ para toda variación.

\textbf{($\impliedby$)} Supongamos que $A'(0) = 0$ para toda $h$, pero existe un punto $p = X(q)$ tal que $H(p) \neq 0$ (p.ej. $H(p) > 0$). Por continuidad, $H > 0$ en un entorno $D(q, r)$. 
Tomamos como función de la variación una \textbf{función meseta} $\tilde{h}$ con soporte en dicho disco. Entonces:
\[
A'(0) = -2 \iint_{D(q,r)} \tilde{h} H \sqrt{EG - F^2} \, du \, dv.
\]
Separando la integral en $D(q, r) \setminus D(q, r')$ y $D(q, r')$, donde $\tilde{h} \equiv 1$:
\begin{itemize}
    \item En la zona exterior, la integral es $\le 0$ (ya que $\tilde{h}, H, \sqrt{\dots}$ son $\ge 0$).
    \item En la zona interior, la integral es estrictamente $< 0$ (pues $H > 0$ y $\tilde{h} = 1$).
\end{itemize}
Esto implica $A'(0) < 0$, lo cual contradice la hipótesis $A'(0) = 0$. Por tanto, $H$ debe ser cero en todo punto.
\end{proof}



    % Capítulo 3
    \chapter{Teorema de Gauss Bonett}
    %
\section{Versión local del teorema de Gauss-Bonnet}

% [ESPACIO PARA DIBUJO PÁGINA 2: Curva con vectores tangentes y ángulo externo]
\begin{center}
    \textit{[Aquí iría el dibujo de la curva $\alpha$ con los vértices, los vectores tangentes $\alpha'_-, \alpha'_+$ y el ángulo externo $\epsilon_i$]}
\end{center}

\begin{definicion}{Polígono curvado}
Sea $\alpha : [0, l] \longrightarrow \mathbb{R}^2$ una curva regular a trozos, p.p.a., cerrada (i.e. $\alpha(0) = \alpha(l)$) y simple. Consideremos una partición $0 = s_0 < s_1 < \dots < s_k = l$ del intervalo $[0, l]$ de forma que $\alpha(s_i)$, $i = 0, \dots, k$, son los \textbf{vértices} de $\alpha$.
La imagen de $\alpha$ recibe el nombre de \textbf{\textcolor{mainred}{polígono curvado}}.

Se dice que la parametrización de $\alpha$ está \textbf{positivamente orientada} si el vector normal $J\alpha'(s)$ apunta al interior de la región acotada por $\alpha$ para todo $s \in [0, l]$ (donde $\alpha$ es regular).


En cada vértice $\alpha(s_i)$, representamos por $\epsilon_i$ el \textbf{ángulo externo}, definido como el ángulo entre el vector tangente de llegada y el de salida:
\[
\epsilon_i = \text{áng}(\alpha'_-(s_i), \alpha'_+(s_i)) \in (-\pi, \pi].
\]
\end{definicion}
\textit{Nota: Cuando decimos que apunta al interior nos referimos a que, literalmente, al dibujarlo apunte hacia la superficie.}
\begin{teorema}{Rotación de las tangentes (Hopf)}
Sea $\alpha : [0, l] \longrightarrow \mathbb{R}^2$ una curva regular a trozos, p.p.a., cerrada y simple, con vértices $\alpha(s_i)$. Sean $\epsilon_i$ los correspondientes ángulos externos y $\theta_i(s)$ el ángulo que forma el vector tangente $\alpha'(s)$ con una dirección fija (eje X) en cada tramo regular. Entonces:
\[
\sum_{i=1}^k (\theta_i(s_i) - \theta_i(s_{i-1})) + \sum_{i=1}^k \epsilon_i = \pm 2\pi,
\]
donde el signo depende de la orientación de $\alpha$ ($+2\pi$ si es positiva).
\end{teorema}

\begin{tcolorbox}[colback=maingreen!5!white, colframe=maingreen!75!black, title=Observaciones: Extensión a superficies]
¿Se puede extender este resultado al caso de curvas cerradas y simples en una superficie regular $S$?
\begin{itemize}
    \item Sea $(U, X)$ una parametrización positiva de $S$ tal que $U$ es homeomorfo a un disco abierto del plano y sea $\alpha : I \longrightarrow S$ una curva regular a trozos, p.p.a., cerrada y simple, con $\alpha(I) \subset X(U)$.
    \item El teorema de rotación de las tangentes sigue siendo cierto, pero ahora:
    \begin{enumerate}
        \item $J$ es la estructura compleja de la superficie (rotación de 90º en el plano tangente).
        \item Los ángulos de rotación se miden respecto a una base ortonormal, por ejemplo $\{ \frac{X_u}{\sqrt{E}}, \frac{X_v}{\sqrt{G}} \}$ (si $F=0$).
    \end{enumerate}
\end{itemize}
\end{tcolorbox}

\begin{proposicion}{Fórmula de la curvatura geodésica para coordenadas ortogonales}
Sea $\alpha : I \longrightarrow S$ una curva regular, p.p.a., contenida en una superficie regular y orientada $S$, y sea $(U, X)$ una parametrización ortogonal ($F=0$) positivamente orientada tal que $\alpha(I) \subset X(U)$.
Entonces, la curvatura geodésica viene dada por:
\[
k_g(s) = \theta'(s) + \frac{1}{2\sqrt{EG(s)}} \left[ -u'(s)E_v(s) + v'(s)G_u(s) \right],
\]
donde $\theta(s)$ es el ángulo que forma $\alpha'(s)$ con $X_u$.
\end{proposicion}
\begin{proof}
    Lo demostramos en problemas. 
\end{proof}

\begin{teorema}{de Green}
Sea $\alpha : [0, \ell] \longrightarrow \mathbb{R}^2$, $\alpha(s) = (u(s), v(s))$, una parametrización positivamente orientada de un polígono curvado en $\mathbb{R}^2$, y sea $\Omega$ el subconjunto abierto acotado por este. Sean ahora $P, Q : \text{cl}\,\Omega \longrightarrow \mathbb{R}$ funciones diferenciables, $P = P(u, v)$, $Q = Q(u, v)$. Entonces,
\[
\iint_{\Omega} \left( \frac{\partial Q}{\partial u}(u, v) - \frac{\partial P}{\partial v}(u, v) \right) \, du \, dv = \sum_{i=1}^{k} \int_{s_{i-1}}^{s_i} \left[ P(s)u'(s) + Q(s)v'(s) \right] \, ds,
\]
donde $0 = s_0 < s_1 < \dots < s_k = \ell$ es una partición de $[0, \ell]$ tal que $\alpha|_{[s_{i-1}, s_i]}$ es diferenciable para todo $i = 0, \dots, k$, y $P(s) = P(u(s), v(s))$, $Q(s) = Q(u(s), v(s))$.
\end{teorema}

\vspace{0.5cm}

Si $\Gamma \subset S$ es un polígono curvado y $\alpha : [0, \ell] \longrightarrow \Gamma \subset S$ es una parametrización por la longitud de arco de $\Gamma$, positivamente orientada, representamos por
\[
\int_{\Gamma} k_g \, ds = \sum_{i=1}^{k} \int_{\alpha_i} k_g \, ds = \sum_{i=1}^{k} \int_{s_{i-1}}^{s_i} k_g(s) \, ds,
\]
donde $s$ es el parámetro arco y $0 = s_0 < s_1 < \dots < s_k = \ell$ es una partición de $[0, \ell]$ de forma que $\alpha_i = \alpha|_{[s_{i-1}, s_i]}$ es una curva regular.

\begin{tcolorbox}[colback=yellow!10!white, colframe=orange!70!white, title=Nota, fonttitle=\bfseries]
La expresión $\int_{\alpha_i} k_g \, ds$ es una \textbf{integral de línea}. Habría que multiplicar por $|\alpha'|$, pero si la curva está parametrizada por el arco (p.p.a.), entonces evidentemente $|\alpha'|=1$ y no aparece explícitamente.
\end{tcolorbox}

% Definición de Región
\begin{definicion}{Región de una superficie}
Una \textbf{región} $R$ de una superficie regular $S$ es un subconjunto relativamente compacto (su clausura es compacta) cuya frontera $\text{bd}(R)$ cumple que cada componente conexa es una curva regular a trozos, cerrada y simple (sin agujeros que se toquen, etc.). O sea, que cada componente conexa sea homeomorfa a $\mathbb{S}^1$
\end{definicion}

\begin{teorema}{Teorema de Gauss-Bonnet (Versión Local)}
Sea $R \subset S$ una \textbf{región simple} de una superficie regular y orientada $S$, de modo que $R \subset X(U)$, siendo $(U, X)$ una parametrización ortogonal de $S$. Entonces:
\[
\int_R K \, dA + \int_{\text{bd } R} k_g \, ds + \sum_{i=1}^k \epsilon_i = 2\pi,
\]
donde $\epsilon_i$ representan los ángulos externos en los vértices del polígono curvado $\text{bd } R$.
\end{teorema}

\begin{proof}
Partimos de la fórmula de la curvatura geodésica para coordenadas ortogonales:
\[
k_g(s) = \frac{1}{2\sqrt{EG}}[-u'E_v + v'G_u] + \theta'(s).
\]
Integramos $k_g$ a lo largo de la frontera $\alpha$ (suma de integrales en cada tramo regular $\alpha_i$):
\[
\int_{\alpha} k_g \, ds = \sum_{i=1}^k \int_{s_{i-1}}^{s_i} k_g(s) \, ds.
\]
Sustituyendo la expresión de $k_g$:
\[
= \sum_{i=1}^k \int_{s_{i-1}}^{s_i} \left( \frac{-E_v u' + G_u v'}{2\sqrt{EG}} \right) ds + \sum_{i=1}^k \int_{s_{i-1}}^{s_i} \theta_i'(s) \, ds.
\]
Analizamos el segundo sumando (el de los ángulos $\theta$). Por el Teorema fundamental del cálculo en cada tramo:
\[
\sum_{i=1}^k (\theta_i(s_i) - \theta_i(s_{i-1})).
\]
Por el \textbf{Teorema de rotación de las tangentes}, sabemos que esta suma más la suma de los ángulos externos es $2\pi$ (ya que la curva es simple y positiva):
\[
\sum (\Delta \theta_i) = 2\pi - \sum_{i=1}^k \epsilon_i.
\]
Ahora analizamos el primer sumando. Identificamos $P = \frac{G_u}{2\sqrt{EG}}$ no, perdón, identificamos los términos para aplicar Green:
Consideramos el integrando $\frac{-E_v}{2\sqrt{EG}} u' + \frac{G_u}{2\sqrt{EG}} v'$.
Tomamos $P = \frac{-E_v}{2\sqrt{EG}}$ y $Q = \frac{G_u}{2\sqrt{EG}}$.
Aplicamos el \textbf{Teorema de Green} en el dominio plano $X^{-1}(R)$:
\[
\sum \int (P u' + Q v') ds = \iint_{X^{-1}(R)} \left( \frac{\partial Q}{\partial u} - \frac{\partial P}{\partial v} \right) du \, dv.
\]
Calculamos las derivadas parciales (recordando la fórmula de la curvatura de Gauss para $F=0$):
\[
\frac{\partial Q}{\partial u} - \frac{\partial P}{\partial v} = \left( \frac{G_u}{2\sqrt{EG}} \right)_u - \left( \frac{-E_v}{2\sqrt{EG}} \right)_v = \left( \frac{G_u}{2\sqrt{EG}} \right)_u + \left( \frac{E_v}{2\sqrt{EG}} \right)_v.
\]
Sabemos que la curvatura de Gauss es $K = -\frac{1}{2\sqrt{EG}} \left[ \left( \frac{E_v}{\sqrt{EG}} \right)_v + \left( \frac{G_u}{\sqrt{EG}} \right)_u \right]$.
Por tanto, la integral doble es exactamente:
\[
\iint_{X^{-1}(R)} -K \sqrt{EG} \, du \, dv = -\int_R K \, dA.
\]
Reuniendo todo:
\[
\int_{\text{bd } R} k_g \, ds = -\int_R K \, dA + 2\pi - \sum \epsilon_i.
\]
Reordenando los términos obtenemos la fórmula buscada:
\[
\int_R K \, dA + \int_{\text{bd } R} k_g \, ds + \sum \epsilon_i = 2\pi.
\]
\end{proof}


% ----------------------------------------------------------------------
% SECCIÓN 2: VERSIÓN GLOBAL
% ----------------------------------------------------------------------
\section{Versión global del teorema de Gauss-Bonnet}
Dada una región $R$, cada polígono curvado $\Gamma_i$ que delimita su frontera se puede parametrizar mediante una curva $\alpha_i : [0, \ell_i] \longrightarrow \Gamma_i$ p.p.a., regular a trozos, y \textbf{positivamente orientada} ($J\alpha'_i(s)$ apunta al interior de $R$).
% --- TRIANGULACIONES Y CARACTERÍSTICA DE EULER ---

\begin{definicion}{Conceptos de triangulación}
\begin{enumerate}[label=\roman*)]
    \item Dado un polígono curvado parametrizado por una curva $\alpha$ regular a trozos (p.p.a.), con \textbf{\textcolor{mainred}{vértices}} $\alpha(s_0), \dots, \alpha(s_k)$, llamamos \textbf{\textcolor{mainred}{aristas}} del polígono a cada uno de los trozos $\alpha_i := \alpha|_{[s_{i-1}, s_i]}$ donde $\alpha$ es regular.
    \item Un \textbf{\textcolor{mainred}{triángulo}} en una superficie regular $S$ es una región simple $R \subset S$ tal que $\text{bd } R$ es un polígono curvado con tres vértices.
    \item Una \textbf{\textcolor{mainred}{triangulación}} $\mathfrak{T} = \{\tau_1, \dots, \tau_n\}$ de una región $R$ es una colección finita de triángulos $\tau_i$ tal que:
    \begin{itemize}
        \item[iii.a)] $\text{cl } R = \bigcup_{i=1}^n \text{cl } \tau_i;$
        \item[iii.b)] si $i \neq j$ entonces, o bien $\text{cl } \tau_i \cap \text{cl } \tau_j = \emptyset$, o bien $\text{cl } \tau_i \cap \text{cl } \tau_j$ es una arista común (completa), o bien es un vértice común.
    \end{itemize}
    \item Dada una triangulación $\mathfrak{T}$ de una región $R$, se denominan \textbf{\textcolor{mainred}{caras}} de $\mathfrak{T}$ a cada uno de los triángulos $\tau_i$ que la componen.
    \item Dada una triangulación $\mathfrak{T}$ de una región $R$, se define la \textbf{\textcolor{mainred}{característica de Euler-Poincaré}} como:
    \[ \chi(R) = C - A + V. \]
\end{enumerate}
\end{definicion}

\begin{tcolorbox}[colback=maingreen!5!white, colframe=maingreen!75!black, title=Propiedades de la Triangulación]
\begin{itemize}
    \item La característica de Euler-Poincaré de una región $R$ no depende de la triangulación elegida (es un \textbf{invariante topológico}).
    \item Toda región de una superficie $S$ admite una triangulación.
    \item Toda superficie regular $S$ puede cubrirse con parametrizaciones ortogonales $(U_i, X_i)$ compatibles con la orientación de $S$. Además, existe una triangulación $\mathfrak{T}$ de $R$ tal que cada $\tau \in \mathfrak{T}$ está contenido en algún entorno coordenado $X_i(U_i)$.
    \item Al orientar los triángulos de $\mathfrak{T}$ positivamente, cada dos triángulos adyacentes determinan orientaciones \textbf{opuestas} en la arista común.
\end{itemize}
\end{tcolorbox}

\begin{lema}{Relación entre caras y aristas}
Sea $R$ una región de una superficie regular sobre la que se efectúa una triangulación $\mathfrak{T}$. Entonces:
\[ 3C = 2A_{int} + A_{ext}, \]
donde $A_{ext}$ y $A_{int}$ son, respectivamente, el número de aristas exteriores (pertenecen a la frontera de $R$) e interiores de $\mathfrak{T}$.
\end{lema}
\begin{proof}
    Se demuestra en la hoja de problemas. 
\end{proof}

\begin{teorema}{Clasificación de superficies compactas}
Sea $S$ una superficie regular, conexa, orientable y compacta. Entonces:
\[ \chi(S) \in \{ 2, 0, -2, -4, \dots, -2n, \dots \}. \]
\textit{Nota manuscrita: En una región simple (homeomorfa al disco), $\chi(R) = 1$. En la esfera, $\chi(S) = 2$.}
\end{teorema}

\begin{teorema}{Teorema de Gauss-Bonnet (Versión Global)}
Sea $R \subset S$ una región de una superficie regular y orientada, y sean $\Gamma_1, \dots, \Gamma_n$ los polígonos curvados que determinan su frontera, positivamente orientados. Sea $\{\epsilon_1, \dots, \epsilon_p\}$ el conjunto total de ángulos externos. Entonces:
\[
\int_R K \, dA + \sum_{i=1}^n \int_{\Gamma_i} k_g \, ds + \sum_{j=1}^p \epsilon_j = 2\pi \chi(R).
\]
\end{teorema}

\begin{tcolorbox}[title=Consecuencias importantes]
\begin{itemize}
    \item Si $S$ es una superficie compacta ($\chi(S)$):
    \[
    \int_S K \, dA = 2\pi \chi(S).
    \]
    \item Si $S$ es compacta con $K > 0$, entonces $\chi(S) > 0 \implies \chi(S)=2 \implies S$ es homeomorfa a la esfera.(El recíproco no es cierto). 
    \item Si $S$ es compacta con $K > 0$, dos geodésicas cerradas simples cualesquiera se cortan.
\end{itemize}
\end{tcolorbox}

\begin{proof}
Tomamos una triangulación $\mathfrak{T} = \{ \tau_1, \dots, \tau_C \}$ de la región $R$ tal que cada triángulo esté contenido en un entorno coordenado.
Cada triángulo $\tau_i$ tiene 3 aristas y 3 vértices, con ángulos externos $\epsilon^i_1, \epsilon^i_2, \epsilon^i_3$.

Aplicamos el \textbf{Teorema de Gauss-Bonnet Local} a cada triángulo $\tau_i$:
\[
\int_{\tau_i} K \, dA + \int_{\partial \tau_i} k_g \, ds + \sum_{j=1}^3 \epsilon^i_j = 2\pi.
\]
Sumamos esta igualdad para los $C$ triángulos ($i=1, \dots, C$):
\[
\sum_{i=1}^C \int_{\tau_i} K \, dA + \sum_{i=1}^C \int_{\partial \tau_i} k_g \, ds + \sum_{i=1}^C \sum_{j=1}^3 \epsilon^i_j = 2\pi C.
\]
Analizamos cada término por separado:

\textbf{1. Integral de la Curvatura:}
\[
\sum \int_{\tau_i} K \, dA = \int_R K \, dA.
\]

\textbf{2. Integral de la Curvatura Geodésica:}
Separamos las aristas en interiores y exteriores.
\[
\sum \int_{\partial \tau_i} k_g \, ds = \sum_{\text{Aristas int}} \int k_g \, ds + \sum_{\text{Aristas ext}} \int k_g \, ds.
\]
Las aristas interiores se recorren dos veces (una por cada triángulo adyacente) en sentidos opuestos. Como la orientación cambia el signo de $k_g ds$, la suma sobre aristas interiores es \textbf{cero}.
Solo quedan las exteriores, que forman la frontera de $R$:
\[
= \int_{\partial R} k_g \, ds.
\]

\textbf{3. Suma de Ángulos:}
Usamos que el ángulo externo es $\epsilon = \pi - \phi$ (donde $\phi$ es el ángulo interno).
\[
\sum_{i,j} \epsilon^i_j = \sum_{i,j} (\pi - \phi^i_j) = 3\pi C - \sum \phi^i_j.
\]
Separamos la suma de ángulos internos según el tipo de vértice en la triangulación:
\begin{itemize}
    \item \textbf{Vértices interiores ($V_{int}$):} La suma de los ángulos alrededor de un vértice interior es $2\pi$.
    \item \textbf{Vértices frontera ($V_{ext}$):}
        \begin{itemize}
            \item Vértices originales del polígono ($\alpha(s_k)$): La suma es el ángulo interior del polígono $\pi - \epsilon_k$.
            \item Vértices falsos o añadidos por la triangulación en los bordes ($V_{ext, B}$): La suma es $\pi$.
        \end{itemize}
\end{itemize}
Entonces:
\[
\sum \phi = 2\pi V_{int} + \sum_{k=1}^p (\pi - \epsilon_k) + \pi V_{ext, B}.
\]
Sustituyendo esto en la expresión de los $\epsilon$:
\[
\sum \epsilon_{totales} = 3\pi C - \left[ 2\pi V_{int} + p\pi - \sum \epsilon_k + \pi V_{ext, B} \right].
\]
Usamos el Lema $3C = 2A_{int} + A_{ext}$ y relaciones combinatorias para simplificar.
Sabemos que $2\pi C - \sum \epsilon_{totales}$ debe ser igual a $2\pi \chi(R)$ tras operar con $C-A+V$.

\textit{(Simplificación final según notas):}
La ecuación original queda:
\[
\int_R K + \int_{\partial R} k_g + (2\pi A + \sum \epsilon_k - 2\pi V) = 2\pi C.
\]
Reagrupando $2\pi (C - A + V) = 2\pi \chi(R)$.
Finalmente:
\[
\int_R K \, dA + \int_{\partial R} k_g \, ds + \sum \epsilon_k = 2\pi \chi(R).
\]
\end{proof}

    % Capítulo 4
    \chapter{Completitud y Teorema de Hopf-Rinow}
    %
% ----------------------------------------------------------------------
% SECCIÓN 1: DISTANCIA INTRÍNSECA
% ----------------------------------------------------------------------
\section{Distancia intrínseca en una superficie}

\begin{definicion}{Distancia intrínseca}
Dados dos puntos $p, q \in S$ (supondremos siempre $S$ conexa), se define el conjunto de curvas que los unen:
\[
\Omega(p,q) = \{ \alpha : [a,b] \longrightarrow S : \alpha \text{ regular a trozos}, \alpha(a)=p, \alpha(b)=q \} \cup 
\]
\[
\{ \alpha :\{a\}\to S: \alpha(a)=p \quad \text{si}\quad  p=q \}.
\]
Se define la \textbf{\textcolor{mainred}{distancia intrínseca}} en la superficie $S$ como la aplicación $d : S \times S \longrightarrow \mathbb{R}$ dada por:
\[
d(p,q) := \inf \{ L_a^b(\alpha) : \alpha \in \Omega(p,q) \}.
\]
\end{definicion}

\begin{tcolorbox}[colback=maingreen!5!white, colframe=maingreen!75!black, title=Observaciones]
La función está bien definida y es, de hecho, una distancia:
\begin{enumerate}[label=\alph*)]
    \item $\Omega(p,q) \neq \emptyset$, esto es, cualesquiera dos puntos de $S$ pueden unirse mediante una curva regular a trozos.
    \item Existe el ínfimo de la definición de distancia (ya que las longitudes son no negativas).
    \item La distancia intrínseca cumple las propiedades de distancia (positiva, simétrica, desigualdad triangular).
\end{enumerate}
\end{tcolorbox}

\begin{proof}[Justificación de a) Conexión por caminos]
Fijamos $p \in S$. Sea $$A = \{ q \in S : \exists \alpha \text{ regular a trozos uniendo } p \text{ y } q \}$$
Queremos ver que $A = S$. Como $S$ es conexa, basta ver que $A \neq \emptyset$, es abierto y es cerrado.

1. \textbf{$A \neq \emptyset$:} $p \in A$ (trivialmente, curva de un punto).

2. \textbf{$A$ es abierto:}
   Sea $q \in A$. Por definición de superficie, existe un entorno de $q$ que es un disco geodésico $\mathcal{D}(q, \epsilon)$ (imagen por la exponencial de un disco en el plano tangente).
   Propiedad: Si tomo un punto $q' \in \mathcal{D}(q, \epsilon)$, existe una \textbf{geodésica radial} $\gamma$ que une $q$ y $q'$.
   Como $q \in A$, existe $\alpha$ que une $p$ con $q$.
   Definimos la curva unión $\alpha \cup \gamma$, que es regular a trozos y une $p$ con $q'$.
   Por tanto, $q' \in A$. Esto implica que $\mathcal{D}(q, \epsilon) \subset A$.

3. \textbf{$A$ es cerrado:}
   Veamos que $S \setminus A$ es abierto. Sea $q \in S \setminus A$ (es decir, $q$ no se puede unir con $p$).
   Tomamos un disco normal $\mathcal{D}(q, \epsilon)$. ¿$\mathcal{D}(q, \epsilon) \subset S \setminus A$?
   Supongamos que no. Entonces $\mathcal{D}(q, \epsilon) \cap A \neq \emptyset$. Sea $q' \in \mathcal{D}(q, \epsilon) \cap A$.
   \begin{itemize}
       \item Como $q' \in A$, existe $\alpha$ que une $p$ con $q'$.
       \item Como $q' \in \mathcal{D}(q, \epsilon)$, existe $\gamma$ geodésica radial que une $q'$ con $q$.
       \item La unión $\alpha \cup \gamma$ uniría $p$ con $q$, lo que implicaría $q \in A$. ¡Contradicción!
   \end{itemize}
   Por tanto, $\mathcal{D}(q, \epsilon) \cap A = \emptyset$, luego $A$ es cerrado.
\end{proof}
\begin{observacion}{Bola en la distancia intrínseca}
Ahora podemos considerar la \textbf{bola} (abierta) en la \textbf{distancia intrínseca} de centro $p$ y radio $r$:
\[
B_d(p, r) = \{ q \in S : d(p, q) < r \}.
\]

\begin{enumerate}[label=\roman*)]
    \item Dado $p_0 \in S$, siempre se verifica que $\mathcal{D}(p_0, r) \subset B_d(p_0, r)$.
    
    \item Dados $p_0 \in S$ y $R > 0$, existe $r > 0$ tal que $B_d(p_0, r) \subset \mathcal{D}(p_0, R)$.
    
    \item Sean $p_0 \in S$ y $R > 0$ tales que $\mathcal{D}(p_0, R)$ es entorno normal de $p_0$. Entonces, para todo $r > 0$ tal que $B_d(p_0, r) \subset \mathcal{D}(p_0, R)$, se tiene que 
    \[
    \mathcal{D}(p_0, r) = B_d(p_0, r).
    \]
    
    \item Además, sobre el conjunto $S$, las topologías $\tau_u$ (usual) y $\tau_d$ (métrica) coinciden.
\end{enumerate}
\end{observacion}


\begin{proof}[Justificación de i): $\mathcal{D}(p_0, r) \subset B_d(p_0, r)$]
    Sea $p \in \mathcal{D}(p_0, r)$. Por definición del disco geodésico, esto implica que $p = \exp_{p_0}(\vec{v})$ para algún vector $\vec{v} \in T_{p_0}S$ con norma $|\vec{v}| < r$.
    
    Consideramos la geodésica radial $\gamma_v(t) = \exp_{p_0}(t\vec{v})$ definida para $t \in [0,1]$. Esta curva une $p_0$ con $p$. Calculamos su longitud:
    \[
    L_0^1(\gamma_v) = \int_0^1 |\gamma_v'(t)| \, dt = \int_0^1 |\vec{v}| \, dt = |\vec{v}|.
    \]
    Por definición, la distancia intrínseca es el ínfimo de las longitudes de todas las curvas que unen los puntos, por lo que:
    \[
    d(p_0, p) \le L(\gamma_v) = |\vec{v}| < r.
    \]
    Como $d(p_0, p) < r$, concluimos que $p \in B_d(p_0, r)$.
    \textit{(Nota: La geodésica radial nos asegura que la distancia es menor que $r$, colocándolo dentro de la bola métrica).}
\end{proof}

\begin{proof}[Justificación de iii): Igualdad en entornos normales]
    Supongamos que $\mathcal{D}(p_0, R)$ es un entorno normal y tomamos $r$ tal que $B_d(p_0, r) \subset \mathcal{D}(p_0, R)$. Queremos probar la igualdad.
    
    Ya sabemos por (i) que $\mathcal{D}(p_0, r) \subset B_d(p_0, r)$. Veamos la otra inclusión: $\supseteq$.
    
    Sea $p \in B_d(p_0, r)$. Esto significa que $d(p_0, p) < r$.
    Como $p$ está dentro del entorno normal $\mathcal{D}(p_0, R)$, existe una \textbf{única geodésica radial} $\gamma_p: [0,1] \to S$ que une $p_0$ con $p$ y se mantiene dentro del entorno.
    
    Sea $\vec{v} = \gamma_p'(0)$. Por las propiedades de minimización de las geodésicas en entornos normales (consecuencia del Lema de Gauss), sabemos que la geodésica radial realiza la distancia:
    \[
    d(p_0, p) = L(\gamma_p) = |\vec{v}|.
    \]
    Como $d(p_0, p) < r$, entonces $|\vec{v}| < r$. Esto implica que el vector tangente está en el disco $D(\vec{0}, r) \subset T_{p_0}S$.
    Finalmente, como $p = \exp_{p_0}(\vec{v})$ y $|\vec{v}| < r$, concluimos que $p \in \mathcal{D}(p_0, r)$.
\end{proof}

\begin{ejemplo}{Plano Agujereado. ¿Por qué no coinciden siempre?}
    Si no estamos en un entorno normal (o si $r$ es muy grande), las topologías difieren.
    Consideremos el plano agujereado y un radio $r$ tal que la geodésica recta chocaría con el agujero.
    \begin{itemize}
        \item El disco geodésico $\mathcal{D}(p, r)$ \textbf{no está definido} en esa dirección (la exponencial no existe).
        \item La bola métrica $B_d(p, r)$ sí incluye puntos al otro lado del agujero, ya que el ínfimo de la distancia esquiva el agujero rodeándolo.
    \end{itemize}
    \end{ejemplo}


% ----------------------------------------------------------------------
% SECCIÓN 2: TEOREMA DE HOPF-RINOW
% ----------------------------------------------------------------------
\section{El Teorema de Hopf-Rinow}

\begin{definicion}{Curva minimizante}
Se dice que una curva $\alpha \in \Omega(p,q)$ es \textbf{\textcolor{mainred}{minimizante}} si minimiza la longitud entre todas las curvas que unen $p$ y $q$. En este caso diremos que \textbf{\textcolor{mainred}{$\alpha$ realiza la distancia}} entre esos dos puntos. 
En tal caso, $d(p,q) = L(\alpha)$.
\end{definicion}


\begin{lema}{1 (Regularidad de la curva minimizante)}
Sean $S$ una superficie regular y $\alpha : [a, b] \longrightarrow S$ una curva regular a trozos uniendo $p = \alpha(a)$ con $q = \alpha(b)$. Si $\alpha$ es minimizante, entonces $\alpha$ es un segmento de geodésica (salvo reparametrizaciones). En particular, $\alpha$ es diferenciable en todos sus puntos, esto es, no tiene vértices.
\end{lema}

\begin{proof}[Demostración Lema 1]
Sea $\alpha : [0, \ell] \longrightarrow S$ regular a trozos uniendo $p$ y $q$.
Sea $\phi : [0, \ell] \times (-\varepsilon, \varepsilon) \longrightarrow S$ una variación propia de $\alpha$.
Todas las curvas de la variación $\alpha_t$ unen $p$ y $q$.
Como $\alpha$ es minimizante, la función longitud $L(t) = L(\alpha_t)$ tiene un mínimo en $t=0$.
Esto implica que $L'(0) = 0$.
Por la caracterización variacional de las geodésicas, concluimos que $\alpha$ es una geodésica.
\end{proof}

\begin{lema}{2 (Existencia local de minimizantes)}
Sean $p_0 \in S$ un punto de una superficie regular $S$ y $r > 0$ de forma que la aplicación $\exp_{p_0}$ está definida en $D(0, r) \subset T_{p_0}S$. Entonces, todo punto $p \in \mathcal{D}(p_0, r)$ puede unirse con $p_0$ mediante (al menos) un segmento de geodésica minimizante.
\end{lema}

\begin{observacion}{Importante; Lema 2}
Fijamos $p_0 \in S$.
\begin{enumerate}
    \item Sea $V(p_0)$ un entorno normal de $p_0$. Para todo $p \in V$, la geodésica radial $\gamma_p$ es la curva más corta \textbf{entre las curvas de $V$} (es única). \textit{Nota: Y para que sea minimizante global tiene que ser la más corta en toda la superficie.}
    
    \item Sea $\mathcal{D}(p_0, r)$ un disco centrado en el centro del entorno normal. Si $\mathcal{D}(p_0, r) \subset V(p_0)$ (tienen que ser el mismo centro), entonces para $p \in \mathcal{D}(p_0, r)$, la radial $\gamma_p$ minimiza (es única). \textit{Nota: Y eso solo vale si parto de $p_0$ (centro del entorno normal).}
    
    \item \textbf{¡Esta es la situación del Lema 2!}
\end{enumerate}
\end{observacion}

% ----------------------------------------------------------------------
% TEOREMA DE HOPF-RINOW
% ----------------------------------------------------------------------
\subsection{El Teorema de Hopf-Rinow}

\begin{teorema}{de Hopf-Rinow}
Sea $S$ una superficie regular conexa. Las siguientes propiedades son equivalentes:
\begin{enumerate}[label=\roman*)]
    \item El espacio métrico $(S, d)$ es \textbf{completo}.
    \item $S$ es \textbf{geodésicamente completa}.
    \item $S$ es geodésicamente completa en un punto.
    \item Se verifica el teorema de \textbf{Heine-Borel}: los conjuntos cerrados (en $S$) y acotados (en la distancia intrínseca $d$) son compactos.
\end{enumerate}

Si se cumple una de las condiciones anteriores (y por tanto, todas ellas), diremos simplemente que la superficie $S$ es \textbf{completa}.

En tal caso, dos puntos cualesquiera de $S$ pueden unirse mediante un segmento de \textbf{geodésica minimizante}.
\end{teorema}

\begin{proof}
\textbf{i) $\implies$ ii)}
Supongamos que $(S, d)$ es completo.
Supongamos por reducción al absurdo que $S$ \textbf{no} es geodésicamente completa.
Entonces, existe un vector $\vec{v} \in T_pS$ tal que la geodésica $\gamma_v$ no está definida en todo $\mathbb{R}$.
Supongamos, por ejemplo, que $\gamma_v : [0, 1) \longrightarrow S$ y que no está definida en $t=1$.

Tomamos una sucesión $\{t_n\}_{n \in \mathbb{N}} \subset [0, 1)$ tal que $t_n \to 1$.
En particular, es una sucesión de Cauchy en $\mathbb{R}$. Dado $\varepsilon > 0$, existe $N$ tal que para $n, m \ge N$, $|t_n - t_m| < \varepsilon$.
Vamos a llevarnos esto a la superficie usando la distancia intrínseca.
Para $n, m \ge N$ (supongamos $t_m < t_n$):
\[
d(\gamma_v(t_n), \gamma_v(t_m)) \le L_{t_m}^{t_n}(\gamma_v) = \int_{t_m}^{t_n} |\gamma_v'(t)| \, dt = |\vec{v}|(t_n - t_m) < |\vec{v}|\varepsilon.
\]
Luego la sucesión $\{\gamma_v(t_n)\}_n$ es de \textbf{Cauchy} en $(S, d)$.
Como $(S, d)$ es un espacio métrico completo (por hipótesis), la sucesión es convergente:
\[
\exists p \in S \text{ tal que } \lim_{n \to \infty} \gamma_v(t_n) = p \quad (*).
\]
Por eso podemos extenderla. No perdamos de vista que queremos extender la geodésica al 1 para llegar a la contradicción.
Nos interesa que exista $\lim_{t \to 1} \gamma_v(t)$. Está bastante ligado a lo anterior. Lo único que necesitamos es que se cumpla $(*)$ para cualquier sucesión.
Basta con cambiar $t_n$ por otra sucesión cualquiera, por ejemplo $s_n \in [0, 1)$ con $s_n \to 1$, y repetir el mismo argumento.
\[
\exists p' \in S \text{ tal que } \lim_{n \to \infty} \gamma_v(s_n) = p'.
\]
Veamos que $p = p'$ (al estar en un espacio métrico esto ocurrirá si $d(p, p') = 0$).
\[
d(\gamma_v(t_n), \gamma_v(s_n)) \le L_{s_n}^{t_n}(\gamma_v) = |\vec{v}||t_n - s_n| \xrightarrow[n \to \infty]{} 0.
\]
Por tanto, $d(p, p') = 0 \implies p = p'$.
Consecuentemente, existe $\lim_{t \to 1} \gamma_v(t) = p$ y, por tanto, podemos definir de forma continua $\gamma_v(1) = p$.

\textit{¡Cuidado, no hemos terminado! Hemos extendido la curva a $t=1$ de forma continua, pero necesitamos extenderla como \textbf{geodésica} para tener la contradicción.}

Veamos que además es geodésica en $t=1$:
Sea $W$ un entorno uniformemente normal de $p = \gamma_v(1)$ (existe un resultado que asegura su existencia).
Sea $a < 1$ tal que $\gamma_v(a) \in W$ (cerca de 1). Como $W$ es entorno uniformemente normal de $p$, en particular será entorno normal de $\gamma_v(a)$.
Por tanto, por el teorema de existencia de geodésicas radiales, tenemos una \textbf{única} geodésica que parte de $\gamma_v(a)$ y llega a $p$, lo que demuestra que $\gamma_v$ es geodésica en $t=1$ y se puede extender más allá. \textbf{Contradicción} con que no estaba definida en 1.

\vspace{0.3cm}

\textbf{ii) $\implies$ iii)}
Trivial (si vale para todos los puntos, vale para uno).

\vspace{0.3cm}

\textbf{iii) $\implies$ iv)}
Sabemos que en cualquier espacio métrico, compacto implica cerrado y acotado. Luego, en particular, en $(S, d)$ también se cumple.
Comprobemos el recíproco: Sea $A \subset S$ cerrado y $d$-acotado.
Sabemos por hipótesis que existe $p_0 \in S$ tal que $S$ es geodésicamente completa en $p_0$. Esto implica que $\exp_{p_0} : T_{p_0}S \longrightarrow S$ está definida en todo el tangente.
Por tanto, para todo $p \in S$, existe una geodésica minimizante $\gamma_v$ uniendo $p_0 = \gamma_v(0)$ y $p = \gamma_v(1)$.
\textit{(Nota: Esto es porque la exponencial está definida en todo el tangente y los discos geodésicos cubren toda la superficie).}

Como $A$ es acotado, existe $M > 0$ tal que $A \subset B_d(p_0, M)$.
Sea $p \in A$. Existe $\gamma_v$ minimizante con $\gamma_v(0) = p_0$ y $\gamma_v(1) = p$.
Entonces $d(p_0, p) = |\vec{v}|$.
Como $p \in A \subset B_d(p_0, M)$, entonces $d(p_0, p) < M$, de donde $|\vec{v}| < M$.
Esto implica que el vector $\vec{v}$ pertenece al disco cerrado $\overline{D(\vec{0}, M)}$ del plano tangente. Y ahí sí se tiene que es compacto (en $\mathbb{R}^2$).
\[
p = \exp_{p_0}(\vec{v}) \in \exp_{p_0}(\overline{D(\vec{0}, M)}).
\]
El conjunto $\exp_{p_0}(\overline{D(\vec{0}, M)})$ es \textbf{compacto} por ser la imagen por una aplicación continua ($\exp_{p_0}$) de un compacto.
Como he cogido cualquier punto de $A$ y está en ese compacto, entonces $A$ está contenido en un compacto.
Como $A$ es cerrado (por hipótesis) y subconjunto de un compacto $\implies A$ es \textbf{compacto}.

\vspace{0.3cm}

\textbf{iv) $\implies$ i)}
¿Es $(S, d)$ completo? (Clase del 09/04)
Sea $\{p_n\}_n \subset (S, d)$ una sucesión de Cauchy.
Sea $A = \{p_n : n \in \mathbb{N}\}$. Como $\{p_n\}$ es de Cauchy, es acotada: dado $\varepsilon > 0$, existe $N$ tal que distancias pequeñas... fijamos $N$, acotamos los primeros $N$ términos (finitos) y el resto por la condición de Cauchy.
Entonces existe $r$ tal que $A \subset B_d(p_0, r)$.
Esto implica que $A$ es $d$-acotado.
Entonces su clausura $\overline{A}$ es $d$-acotada y cerrada.
Por hipótesis (Heine-Borel), $\overline{A}$ es \textbf{compacto}.
Toda sucesión en un compacto tiene una subsucesión convergente.
Consecuencia: Una sucesión de Cauchy con una subsucesión convergente es \textbf{convergente}.
Por tanto, $(S, d)$ es completo.
\end{proof}

\begin{corolario}{Consecuencias importantes}
\begin{itemize}
    \item Toda superficie regular y cerrada en $\mathbb{R}^3$ es completa.
    \item Toda superficie regular y compacta es completa.
\end{itemize}
\end{corolario}

\begin{proof}[Demostración de la primera consecuencia (Cerrada en $\mathbb{R}^3$)]
Supongamos que $S$ es cerrada en $\mathbb{R}^3$. Queremos ver que es completa.
Usamos la equivalencia (i): ver que el espacio métrico $(S, d)$ es completo.
Sea $\{p_n\}_n$ una sucesión de Cauchy en $(S, d)$.
Usamos el "truco" de comparar distancias: la distancia intrínseca es un ínfimo de longitudes sobre la superficie. La distancia euclídea (la cuerda) siempre es menor o igual que la intrínseca:
\[
|p_n - p_m|_{\mathbb{R}^3} \le d_S(p_n, p_m).
\]
Como $\{p_n\}$ es de Cauchy en $(S, d)$, entonces $\{p_n\}$ es de Cauchy en $(\mathbb{R}^3, |\cdot|)$.
Como $\mathbb{R}^3$ es completo, la sucesión es convergente en $\mathbb{R}^3$:
\[
\exists p \in \mathbb{R}^3 \text{ tal que } \lim_{n \to \infty} p_n = p.
\]
Como $S$ es un conjunto \textbf{cerrado} en $\mathbb{R}^3$, el límite $p$ debe pertenecer a $S$.
Finalmente, como sobre $S$ la topología usual y la intrínseca coinciden ($(S, \tau_u) = (S, \tau_d)$), la convergencia se da también en la métrica intrínseca.
Por tanto, $(S, d)$ es completo.
\end{proof}




% ----------------------------------------------------------------------
% SECCIÓN 3: TEOREMA DE BONNET
% ----------------------------------------------------------------------
\section{El Teorema de Bonnet}

\begin{definicion}{Diámetro}
Se define el diámetro intrínseco de una superficie regular y compacta $S$ como:
\[
D(S) := \max \{ d(p,q) : p, q \in S \}.
\]
\end{definicion}

\begin{teorema}{de Bonnet}
Sea $S$ una superficie regular y completa, tal que su curvatura de Gauss $K$ satisface la condición $K(p) \ge \delta > 0$ para todo $p \in S$.
Entonces $S$ es \textbf{compacta} y su diámetro verifica:
\[
D(S) \le \frac{\pi}{\sqrt{\delta}}.
\]
\end{teorema}

\begin{proof}[Demostración Álex]
Procedemos por reducción al absurdo. Supongamos que $D(S) > \frac{\pi}{\sqrt{\delta}}$.
Entonces existen $p_1, p_2 \in S$ tales que $d(p_1, p_2) = l > \frac{\pi}{\sqrt{\delta}}$.
Como $S$ es completa (Hopf-Rinow), existe una geodésica minimizante $\gamma : [0, l] \longrightarrow S$ uniendo $p_1$ y $p_2$.

Vamos a construir una variación que nos dé una contradicción (mostrando que $\gamma$ no sería minimizante).
Necesitamos un campo variacional $Z(s)$ adecuado.
\begin{enumerate}
    \item Tomamos un vector unitario $\vec{w} \in T_{p_1}S$ tal que $\vec{w} \perp \gamma'(0)$.
    \item Sea $W(s)$ el \textbf{transporte paralelo} de $\vec{w}$ a lo largo de $\gamma$.
    \item Propiedades de $W(s)$: es paralelo ($W'=0$), tiene norma 1 ($|W|=1$) y es ortogonal a la curva ($\langle W, \gamma' \rangle = 0$).
    \item Definimos el campo:
    \[
    Z(s) = \sin\left( \frac{\pi}{l}s \right) W(s).
    \]
\end{enumerate}
Comprobamos que es una variación propia:
$Z(0) = \sin(0)W(0) = 0$ y $Z(l) = \sin(\pi)W(l) = 0$.

Aplicamos la \textbf{Segunda Fórmula de Variación}:
\[
L''(0) = \int_0^l \left[ \left| \frac{DZ}{ds} \right|^2 - K(\gamma(s)) |Z(s)|^2 \right] ds.
\]
Calculamos los términos:
\begin{itemize}
    \item $|Z(s)|^2 = \sin^2(\frac{\pi}{l}s) |W(s)|^2 = \sin^2(\frac{\pi}{l}s)$.
    \item $\frac{DZ}{ds} = \frac{\pi}{l} \cos(\frac{\pi}{l}s) W(s) + \sin(\dots) \underbrace{W'(s)}_{0} \implies \left| \frac{DZ}{ds} \right|^2 = \frac{\pi^2}{l^2} \cos^2(\frac{\pi}{l}s)$.
\end{itemize}
Sustituyendo en la integral y usando que $K \ge \delta$:
\[
L''(0) = \int_0^l \left[ \frac{\pi^2}{l^2} \cos^2\left(\frac{\pi}{l}s\right) - K \sin^2\left(\frac{\pi}{l}s\right) \right] ds < \int_0^l \left[ \frac{\pi^2}{l^2} \cos^2 - \delta \sin^2 \right] ds.
\]
\textit{Nota: Como $\delta > \frac{\pi^2}{l^2}$ (por la hipótesis del absurdo $l > \pi/\sqrt{\delta}$), el término que resta es "muy grande".}
De hecho:
\[
L''(0) < \frac{\pi^2}{l^2} \int_0^l \left( \cos^2\left(\frac{\pi}{l}s\right) - \sin^2\left(\frac{\pi}{l}s\right) \right) ds = \frac{\pi^2}{l^2} \int_0^l \cos\left(\frac{2\pi}{l}s\right) ds = 0.
\]
Obtenemos $L''(0) < 0$. Esto implica que $L(t)$ tiene un \textbf{máximo} en $t=0$, lo cual contradice que $\gamma$ sea una geodésica minimizante (debería ser un mínimo).

\textbf{Conclusión:} $D(S) \le \frac{\pi}{\sqrt{\delta}}$. Como $S$ es acotada (diámetro finito) y completa, es compacta.
\end{proof}

\begin{tcolorbox}[colback=maingreen!5!white, colframe=maingreen!75!black, title=Observaciones Finales]
\begin{itemize}
    \item No basta suponer $K > 0$. Ejemplo: Paraboloide de revolución $z = x^2 + y^2$. Tiene $K > 0$ pero no es compacta (no se cumple que $K \ge \delta > 0$, ya que $K \to 0$ en el infinito).
    \item La cota es óptima. Para la esfera de radio $R$, $K = 1/R^2 = \delta$. Diámetro $= \pi R = \pi / \sqrt{\delta}$.
\end{itemize}
\end{tcolorbox}

    % Capítulo 5
    \chapter{Superficies Abstractas}
    %% Archivo: capitulos/tema5.tex

% -------------------------------------------------------------------------
\section{Superficies Abstractas. Introducción a la Geometría de Riemann}

Hasta ahora hemos estudiado superficies inmersas en $\mathbb{R}^3$. Sin embargo, el concepto de superficie puede definirse de manera intrínseca utilizando topología y cartas coordenadas, sin referencia a un espacio ambiente.

\subsection{Definiciones Topológicas}

\begin{definicion}{Superficie Abstracta}
Una superficie abstracta $S$ es un espacio topológico que cumple las siguientes condiciones:
\begin{enumerate}
    \item Es de \textbf{Hausdorff} (puntos distintos pueden separarse por abiertos disjuntos).
    \item Es \textbf{2AN} (Segundo Axioma de Numerabilidad: admite una base numerable de abiertos).
    \item Es localmente euclídeo: Para todo punto $p \in S$, existen un entorno abierto $U(p)$, un abierto $V \subset \mathbb{R}^2$ (con la topología usual $\mathcal{T}_u$) y un homeomorfismo:
    \[ \varphi: U(p) \longrightarrow V, \]
    llamado \textbf{carta} (o sistema de coordenadas locales).
\end{enumerate}
Si escribimos $\varphi(q) = (u(q), v(q))$, las funciones $u$ y $v$ se llaman funciones coordenadas.
\end{definicion}

\begin{proposicion}{Cambio de Cartas}
Dada una superficie abstracta $S$, sean $\varphi_1: U_1(p) \to V_1$ y $\varphi_2: U_2(p) \to V_2$ dos cartas que cubren un punto $p$. La aplicación de transición (o cambio de cartas):
\[
\varphi_2 \circ \varphi_1^{-1}: \varphi_1(U_1 \cap U_2) \longrightarrow \varphi_2(U_1 \cap U_2)
\]
es un homeomorfismo entre abiertos de $\mathbb{R}^2$.
\end{proposicion}

Para hacer cálculo diferencial, necesitamos pedir más regularidad en estos cambios de cartas.

\begin{definicion}{Superficie Abstracta Diferenciable}
Una superficie abstracta se dice \textbf{diferenciable} si los cambios de cartas $\varphi_j \circ \varphi_i^{-1}$ son \textbf{difeomorfismos} (diferenciables con inversa diferenciable).
\end{definicion}

% -------------------------------------------------------------------------
\section{El Plano Tangente Abstracto}

Como no estamos en $\mathbb{R}^3$, no podemos visualizar los vectores tangentes como flechas saliendo de la superficie. Debemos definirlos operacionalmente como objetos que derivan funciones.

\begin{definicion}{Vector Tangente}
Sean $S$ una superficie abstracta diferenciable, $p \in S$ y $\alpha: I \to S$ una curva con $\alpha(0)=p$. Un vector tangente en $p$ es un operador $\alpha'(0): \mathcal{C}^\infty(S) \to \mathbb{R}$ definido por:
\[
\alpha'(0)(f) := \frac{d}{dt}\Big|_{t=0} (f \circ \alpha)(t).
\]
\end{definicion}

\begin{proposicion}{Propiedades algebraicas}
Este operador verifica dos propiedades fundamentales para cualquier $a, b \in \mathbb{R}$ y $f, g \in \mathcal{C}^\infty(S)$:
\begin{enumerate}
    \item \textbf{Linealidad:} $\alpha'(0)(af+bg) = a\alpha'(0)(f) + b\alpha'(0)(g)$.
    \item \textbf{Regla de Leibniz:} $\alpha'(0)(f \cdot g) = f(p)\alpha'(0)(g) + g(p)\alpha'(0)(f)$.
\end{enumerate}
\end{proposicion}

\begin{teorema}{Espacio Tangente $T_pS$}
Consideremos los conjuntos de operadores lineales:
\begin{itemize}
    \item El conjunto de operadores $\{\omega: \mathcal{C}^\infty(S) \to \mathbb{R} \mid \omega \text{ es } \mathbb{R}\text{-lineal}\}$ tiene dimensión infinita.
    \item El conjunto de operadores $\{\omega: \mathcal{C}^\infty(S) \to \mathbb{R} \mid \omega \text{ es lineal y Leibniz}\}$ es un espacio vectorial real de \textbf{dimensión 2}.
\end{itemize}
A este último espacio se le denomina \textbf{Plano Tangente} a $S$ en $p$, denotado por $T_pS$.
\end{teorema}

% -------------------------------------------------------------------------
\section{Geometría Riemanniana}

Para recuperar conceptos geométricos (longitudes, ángulos, áreas) en este contexto abstracto, necesitamos dotar al plano tangente de un producto escalar.

\begin{definicion}{Métrica Riemanniana}
Una métrica riemanniana en $S$ es una elección, para cada $p \in S$, de un producto interior definido positivo $\inner{\cdot}{\cdot}_p$ en $T_pS$. Es decir, para cualesquiera $X, Y, Z \in T_pS$ y $a, b \in \mathbb{R}$:
\begin{itemize}
    \item \textbf{Bilinealidad:} $\inner{aX+bY}{Z}_p = a\inner{X}{Z}_p + b\inner{Y}{Z}_p$.
    \item \textbf{Simetría:} $\inner{X}{Y}_p = \inner{Y}{X}_p$.
    \item \textbf{Definida positiva:} $\inner{X}{X}_p \ge 0$, siendo 0 si y solo si $X=0$.
\end{itemize}
Esta elección debe variar de manera diferenciable con el punto $p$ .
\end{definicion}

% -------------------------------------------------------------------------
\section{Objetos Geométricos Intrínsecos}

Una vez fijada la métrica (los coeficientes $E, F, G$), todos los objetos geométricos se definen a partir de ella.

\begin{definicion}{Símbolos de Christoffel}
Se definen mediante el siguiente sistema de ecuaciones lineales, que involucra a la métrica y sus derivadas parciales:
\[
\begin{pmatrix} E & F \\ F & G \end{pmatrix}
\begin{pmatrix} \Gamma_{11}^1 & \Gamma_{12}^1 & \Gamma_{22}^1 \\ \Gamma_{11}^2 & \Gamma_{12}^2 & \Gamma_{22}^2 \end{pmatrix}
=
\begin{pmatrix} \frac{E_u}{2} & \frac{E_v}{2} & F_v - \frac{G_u}{2} \\ F_u - \frac{E_v}{2} & \frac{G_u}{2} & \frac{G_v}{2} \end{pmatrix}.
\]
\end{definicion}

\begin{definicion}{Curvatura de Gauss}
La curvatura de Gauss $K$ es un invariante intrínseco definido por:
\[
K := \frac{1}{E} \left[ \Gamma_{11}^1 \Gamma_{12}^2 + (\Gamma_{11}^2)_v + \Gamma_{11}^2 \Gamma_{22}^2 - \Gamma_{12}^1 \Gamma_{11}^2 - (\Gamma_{12}^2)_u - (\Gamma_{12}^2)^2 \right].
\]
\end{definicion}

\begin{definicion}{Geodésicas}
Las geodésicas de $(S, g)$ son las curvas que satisfacen el sistema de ecuaciones diferenciales :
\[
\begin{cases}
u'' + (u')^2\Gamma_{11}^1 + 2u'v'\Gamma_{12}^1 + (v')^2\Gamma_{22}^1 = 0, \\
v'' + (u')^2\Gamma_{11}^2 + 2u'v'\Gamma_{12}^2 + (v')^2\Gamma_{22}^2 = 0.
\end{cases}
\]
\end{definicion}
    \appendix 
    %\chapter*{Anexo: Ecuaciones Diferenciales}
\addcontentsline{toc}{chapter}{Anexo: Ecuaciones Diferenciales}


\section*{Ecuación lineal orden $n$ con coeficientes constantes}

Consideremos la EDO lineal homogénea de orden $n$ con coeficientes constantes,
\begin{equation}
    x^{n)} + a_{n-1}x^{n-1)} + \cdots + a_1 x' + a_0 x = 0, 
\end{equation}
con $a_{n-1}, \dots, a_0 \in \mathbb{R}$.

\begin{definicion}{Polinomio característico}
Dada la EDO lineal homogénea de orden $n$ se define su \textbf{polinomio característico} como el polinomio
\begin{equation}
    p(\lambda) = a_0 + a_1\lambda + a_2\lambda^2 + \cdots + a_{n-2}\lambda^{n-2} + a_{n-1}\lambda^{n-1} + \lambda^n. \tag{3.23}
\end{equation}

\end{definicion}


\begin{teorema}{Sistema fundamental de soluciones}
Un sistema fundamental de soluciones de (3.21) está formado por
\begin{equation}
    \{ t^k e^{at} \cos(bt), \, t^k e^{at} \sin(bt) \}, \tag{3.24}
\end{equation}
con $a+bi$, $b \ge 0$, recorriendo el conjunto de los valores propios y siendo $k = 0, \dots, m_a(\lambda) - 1$.

\end{teorema}


\subsection*{Ecuación lineal completa de orden $n$ con coeficientes constantes}

Queda sólo pendiente considerar de forma particular la EDO no homogénea
\begin{equation}
    x^{n)} + a_{n-1}x^{n-1)} + \cdots + a_1 x' + a_0 x = b(t). 
\end{equation}


\begin{teorema}{Método de coeficientes indeterminados}
Supongamos $b(t) = e^{at}(p(t)\cos(bt) + q(t)\sin(bt))$, donde $p, q$ son polinomios de grado a lo sumo $k \ge 0$. Sea $\lambda = a + bi$.

\begin{enumerate}
    \item Si $\lambda$ \textbf{no es raíz} del polinomio característico (3.23), entonces (3.25) tiene una solución particular de la forma
    \[
    e^{at}(r(t)\cos(bt) + s(t)\sin(bt)),
    \]
    con $r, s$ polinomios de grado a lo sumo $k$.

    \item Si $\lambda$ \textbf{es raíz} del polinomio característico (3.23) de multiplicidad $l$, entonces (3.25) tiene una solución particular de la forma
    \[
    t^l e^{at}(r(t)\cos(bt) + s(t)\sin(bt)),
    \]
    con $r, s$ polinomios de grado a lo sumo $k$.
\end{enumerate}

\end{teorema}

\begin{teorema}{Principio de superposición para EDOs}
Consideremos la EDO
\begin{equation}
    x^{n)} + a_{n-1}(t)x^{n-1)} + \cdots + a_1(t)x' + a_0(t)x = b_1(t) + b_2(t) + \cdots + b_r(t), \tag{3}
\end{equation}
con $b_1, \dots, b_r : I \subset \mathbb{R} \to \mathbb{R}$ continuas. Para cada $k = 1, \dots, r$, sea $x_k$ solución particular de
\[
    x^{n)} + a_{n-1}(t)x^{n-1)} + \cdots + a_1(t)x' + a_0(t)x = b_k(t).
\]
Entonces, $x_p = x_1 + \cdots + x_r$ es solución particular de (3).

\end{teorema} % Asumiendo que guardaste el archivo en la carpeta 'anexos'


\end{document}